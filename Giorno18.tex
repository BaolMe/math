% !TEX TS-program = latex
% !TEX spellcheck = it-IT
\documentclass[a4paper]{article}
\usepackage[utf8]{inputenc}
\usepackage[T1]{fontenc}
\usepackage[italian]{babel}
\usepackage{tikz} % LATEX
\usetikzlibrary {automata}


%%%%%%%%%%%%%%%%%%%%%%%%%%%%%%%%%%%%%%%%%%%%%%%%%%%%%%%%%%%%%%%%%%%%%%%%%%%%
%% Trim Size: 9.75in x 6.5in
%% Text Area: 8in (include Runningheads) x 5in
%% ws-ijgmmp.tex   :   2-9-08
%% Tex file to use with ws-ijgmmp.cls written in Latex2E.
%% The content, structure, format and layout of this style file is the
%% property of World Scientific Publishing Co. Pte. Ltd.
%% Copyright 1995, 2002 by World Scientific Publishing Co.
%% All rights are reserved.
%%%%%%%%%%%%%%%%%%%%%%%%%%%%%%%%%%%%%%%%%%%%%%%%%%%%%%%%%%%%%%%%%%%%%%%%%%%%
%%

%%%%%%%%%%%%%%%%%%%%%%%%%%%%%


\usepackage{pictexwd,dcpic}
\usepackage{csquotes}
\usepackage{nopageno}

\usepackage{amsmath, amssymb, amsthm}
\usepackage{pictex, dcpic}
\usepackage{color}
%\usepackage{graphicx}
%\numberwithin{equation}{section} % This line resets equation numbering when starting a new section.
%\renewcommand{\theequation}{Eq. \thesection.\arabic{equation}} % This line ads "Eq." in front of your equation numbering.




%
%    Macros.    Version 1.2.0.beta
%    The best use is to paste all of them into the papers
%     1/8/2005
%

%%%%%%%%%%%%%%%%%%%%%%%%%%%%%%
%%%%%%			Greek		 %%%%%%
%%%%%%%%%%%%%%%%%%%%%%%%%%%%%%

\def\al{\alpha}
\def\be{\beta}
\def\de{\delta}
\def\ga{\gamma}
\def\up{\upsilon}
\def\ep{\epsilon}
\def\io{\iota}
\def\te{\theta}
\def\la{\lambda}
\def\ze{\zeta}
\def\om{\omega}
\def\si{\sigma}
\def\vp{\varphi}
\def\vpi{\varpi}
\def\ka{\kappa}
\def\vs{\varsigma}
\def\vr{\varrho}

\def\De{\Delta}
\def\Ga{\Gamma}
\def\Te{\Theta}
\def\La{\Lambda}
\def\Om{\Omega}
\def\Si{\Sigma}
\def\Up{\Upsilon}

\def\boldDe{\bf\Delta}
\def\boldGa{\bf\Gamma}
\def\boldTe{\bf\Theta}
\def\boldLa{\bf\Lambda}
\def\boldOm{\bf\Omega}
\def\boldSi{\bf\Sigma}
\def\boldUp{\bf\Upsilon}
\def\boldXi{\bf\Xi}
\def\boldPi{\bf\Pi}
\def\boldPhi{\bf\Phi}
\def\boldPsi{\bf\Psi}

\def\boldal{\bf\al}
\def\boldbe{\bf\be}
\def\boldga{\bf\ga}
\def\boldde{\bf\de}
\def\boldep{\bf\ep}
\def\boldze{\bf\ze}
\def\boldeta{\bf\eta}
\def\boldte{\bf\te}
\def\boldio{\bf\io}
\def\boldka{\bf\ka}
\def\boldla{\bf\la}
\def\boldmu{\hbox{\greekbold\char "16}}
\def\boldnu{\hbox{\greekbold\char "17}}
\def\boldxi{\hbox{\greekbold\char "18}}
\def\boldpi{\hbox{\greekbold\char "19}}
\def\boldrho{\hbox{\greekbold\char "1A}}
\def\boldsi{\bf\si}
\def\boldtau{\hbox{\greekbold\char "1C}}
\def\boldup{\hbox{\greekbold\char "1D}}
\def\boldphi{\hbox{\greekbold\char "1E}}
\def\boldchi{\hbox{\greekbold\char "1F}}
\def\boldpsi{\hbox{\greekbold\char "20}}
\def\boldom{\hbox{\greekbold\char "21}}
\def\boldvep{\hbox{\greekbold\char "22}}
\def\boldvte{\hbox{\greekbold\char "23}}
\def\boldvpi{\hbox{\greekbold\char "24}}
\def\boldvrho{\hbox{\greekbold\char "25}}
\def\boldvsi{\hbox{\greekbold\char "26}}
\def\boldvphi{\hbox{\greekbold\char "27}}

%%%%%%%%%%%%%%%%%%%%%%%%%%%%%%
%%%%%%			Cal			 %%%%%%
%%%%%%%%%%%%%%%%%%%%%%%%%%%%%%
 \def\calA{{\hbox{\cal A}}}
 \def\calU{{\hbox{\cal U}}}
 \def\calB{{\hbox{\cal B}}}
 \def\calC{{\hbox{\cal C}}}
 \def\calI{{\hbox{\cal I}}}
 \def\calQ{{\hbox{\cal Q}}}
 \def\calP{{\hbox{\cal P}}}
 \def\calL{{\hbox{\cal L}}}
 \def\calE{{\hbox{\cal E}}}
 \def\calW{{\hbox{\cal W}}}
 \def\calV{{\hbox{\cal V}}}
 \def\calS{{\hbox{\cal S}}}
 \def\calF{{\hbox{\cal F}}}
 \def\calR{{\hbox{\cal R}}}
 \def\calO{{\hbox{\cal O}}}
 \def\calM{{\hbox{\cal M}}}

%%%%%%%%%%%%%%%%%%%%%%%%%%%%%%
%%%%%%			gothic		 %%%%%%
%%%%%%%%%%%%%%%%%%%%%%%%%%%%%%
 \def\su{{\mathfrak{su}}}
 \def\gl{{\mathfrak{gl}}}
 \def\gotg{{\mathfrak{g}}}
 \def\goth{{\mathfrak{h}}}
 \def\gotm{{\mathfrak{m}}}
 \def\gotk{{\mathfrak{k}}}
 \def\spin{{\mathfrak{spin}}}
 \def\slC{{\mathfrak{sl}}}
 \def\O{{\mathfrak{O}}}
 \def\Set{{\mathfrak{Set}}}

%%%%%%%%%%%%%%%%%%%%%%%%%%%%%%
%%%%%%			Bbb			 %%%%%%
%%%%%%%%%%%%%%%%%%%%%%%%%%%%%%
 \def\one{\mathbb{I}}
 \def\A{\mathbb{A}}
 \def\B{\mathbb{B}}
 \def\C{\mathbb{C}}
 \def\D{\mathbb{D}}
 \def\E{\mathbb{E}}
 \def\F{\mathbb{F}}
 \def\G{\mathbb{G}}
 \def\H{\mathbb{H}}
 \def\J{\mathbb{J}}
 \def\K{\mathbb{K}}
 \def\I{\mathbb{I}}
 \def\L{\mathbb{L}}
 \def\M{\mathbb{M}}
 \def\N{\mathbb{N}}
 \def\O{\mathbb{O}}
 \def\P{\mathbb{P}}
 \def\Q{\mathbb{Q}}
 \def\R{\mathbb{R}}
 \def\S{\mathbb{S}}
 \def\T{\mathbb{T}}
 \def\U{\mathbb{U}}
 \def\V{\mathbb{V}}
 \def\X{\mathbb{X}}
 \def\Y{\mathbb{Y}}
 \def\W{\mathbb{W}}
 \def\Z{\mathbb{Z}}
 

%%%%%%%%%%%%%%%%%%%%%%%%%%%%%%
%%%%%%		MathRoman		 %%%%%%
%%%%%%%%%%%%%%%%%%%%%%%%%%%%%%
\def\Tr{{\hbox{Tr}}}
\def\Con{{\hbox{Con}}}
\def\Aut{{\hbox{Aut}}}
\def\Div{{\hbox{Div}}}
\def\ad{{\hbox{ad}}}
\def\Ad{{\hbox{Ad}}}
\def\Re{{\hbox{Re}}}
\def\Im{{\hbox{Im}}}
\def\Card{{\hbox{Card}}}
\def\Iso{{\hbox{Iso}}}
\def\Geo{{\hbox{Geo}}}
\def\Int{{\hbox{Int}}}
\def\Inv{{\hbox{Inv}}}
\def\Spin{{\hbox{Spin}}}
\def\SO{{\hbox{SO}}}
\def\SU{{\hbox{SU}}}
\def\SL{{\hbox{SL}}}
\def\GL{{\hbox{GL}}}
\def\det{{\hbox{det}}}
\def\Hom{{\hbox{Hom}}}
\def\End{{\hbox{End}}}
\def\Euc{{\hbox{Euc}}}
\def\Lor{{\hbox{Lor}}}
\def\Diff{{\hbox{Diff}}}
\def\di{{\hbox{d}}}
\def\id{{\hbox{\rm id}}}
\def\diag{{\hbox{diag}}}
\def\rank{{\hbox{rank}}}
\def\span{{\hbox{span}}}
\def\Sim{{\hbox{Sim}}}

\def\Obj{{\hbox{Obj}}}

%%%%%%%%%%%%%%%%%%%%%%%%%%%%%%
%%%%%%		OtherSymbols		 %%%%%%
%%%%%%%%%%%%%%%%%%%%%%%%%%%%%%
\def\ip{\hbox to4pt{\leaders\hrule height0.3pt\hfill}\vbox to8pt{\leaders\vrule width0.3pt\vfill}\kern 2pt}

% inner product
\def\QDE{\hfill\hbox{\ }\vrule height4pt width4pt depth0pt} 
\def\del{\partial}
\def\na{\nabla}
\def\inter{\cap}
\def\Vec{\mathfrak{X}}
\def\Lie{\hbox{\LieFont \$}}

\def\arr{\rightarrow}
\def\larr{\longrightarrow}
\def\harr{\hookrightarrow}
\def\hlarr{\lhook\joinrel\longrightarrow}
\def\then{\Rightarrow}
\def\semidirect{\hbox{\Bbb \char111}}
\def\binomial#1#2{\left(\Matrix{#1\cr #2\cr}\right)}

\def\barJ{\bar J}

\def\sSU{{\hbox{SU}}}
\def\so{{so}}
%\def\N{{{\mathbb N}}}

\def\file#1{{\tt #1}}
\def\calH{{{\cal H}}}
\def\calJ{{{\cal J}}}
\def\calA{{{\cal A}}}
\def\calV{{{\cal V}}}
\def\calG{{{\cal G}}}
\def\calP{{{\cal P}}}
\def\calK{{{\cal K}}}
\def\calD{{{\cal D}}}
\def\scalD{{{\cal D}}}

\def\Loop{{\hbox{Loop}}}
\def\Hoop{{\hbox{Hoop}}}
\def\Sym{{\hbox{Sym}}}
\def\sR{{\R}}
\def\dCor{dCor}
\def\epm{\hbox{$^\pm$}}
\def\dunion{\coprod}

\def\nab#1{{\buildrel #1\over \na}}
\def\frac[#1/#2]{\hbox{$#1\over#2$}}
\def\Frac[#1/#2]{{#1\over#2}}
\def\({\left(}
\def\){\right)}
\def\[{\left[}
\def\]{\right]}
\def\^#1{{}^{#1}_{\>\cdot}}
\def\_#1{{}_{#1}^{\>\cdot}}
\def\Label=#1{{\buildrel {\hbox{\fiveSerif \ShowLabel{#1}}}\over =}}
\def\<{\kern -1pt}
\def\Bar{\>|\>}
\def\Dal{\hbox{\tenRelazioni  \char003}}

\def\uvec#1{\vbox{\hbox{$\scriptstyle\rightharpoonup$}\vskip-9pt\hbox{$#1$}}}
\def\dvec#1{\vtop{\hbox{$#1$}\vskip-10pt\hbox{$\scriptstyle\rightharpoondown$}}}
\def\Cprod{\diamond}
\def\obullet{\odot}



%%%%%%%%%%%%			frames 				%%%%%%%%%%%%%%%%%%%

\def\red#1{{\color{red}{#1}}}
\def\blue#1{{\color{blue}{#1}}}
\def\green#1{{\color{green}{#1}}}

\def\Red{\color{red}}
\def\Blue{\color{blue}}
\def\Green{\color{green}}


\def\frame#1{\vbox{\hrule\hbox{\vrule\vbox{\kern2pt\hbox{\kern2pt#1\kern2pt}\kern2pt}\vrule}\hrule\kern-4pt}} 
\def\redframe#1{\red{\frame{#1}}} 
\def\greenframe#1{\green{\frame{#1}}} 
\def\blueframe#1{\blue{\frame{#1}}} 

\def\uline#1{\underline{#1}}
\def\uuline#1{\underline{\underline{#1}}}
\def\Box to #1#2#3{\frame{\vtop{\hbox to #1{\hfill #2 \hfill}\hbox to #1{\hfill #3 \hfill}}}}



\def\ubal{\underline{\al}\kern1pt}
\def\obal{\overline{\al}\kern1pt}

\def\ubR{\underline{R}\kern1pt}
\def\obR{\overline{R}\kern1pt}
\def\ubom{\underline{\om}\kern1pt}
\def\obxi{\overline{\xi}\kern1pt}
\def\ubu{\underline{u}\kern1pt}
\def\ube{\underline{e}\kern1pt}
\def\obe{\overline{e}\kern1pt}
\def\Limit{\>{\buildrel{r\arr\infty}\over \longrightarrow}\,}
\def\union{\cup}
\def\Emptyset{\varnothing}




\def\Uvec#1{\vbox{\mathsurround=0pt\ialign{##\crcr
     $\scriptscriptstyle\rightharpoonup$\crcr\noalign{\kern1pt\nointerlineskip}
     $\hfil\displaystyle{#1}\hfil$\crcr}}}
\def\Dvec#1{\vbox{\mathsurround=0pt\ialign{##\crcr
     $\scriptscriptstyle\rightharpoondown$\crcr\noalign{\kern-7pt\nointerlineskip}
     $\hfil\displaystyle{#1}\hfil$\crcr}}}


%   u 
\def\uvecu{\vbox{\mathsurround=0pt\ialign{##\crcr
     $\scriptscriptstyle\rightharpoonup$\crcr\noalign{\kern1pt\nointerlineskip}
     $\hfil\displaystyle{u}\hfil$\crcr}}}
\def\dvecu{\vbox{\mathsurround=0pt\ialign{##\crcr
     $\scriptscriptstyle\rightharpoondown$\crcr\noalign{\kern-7pt\nointerlineskip}
     $\hfil\displaystyle{u}\hfil$\crcr}}}
%\def\uvecu{\Uvec{u}}
%\def\dvecu{\Dvec{u}}

%   be
\def\uvecbe{\vbox{\mathsurround=0pt\ialign{##\crcr
     \kern3pt$\scriptscriptstyle\rightharpoonup$\crcr\noalign{\kern1pt\nointerlineskip}
     $\hfil\displaystyle{\be}\hfil$\crcr}}}
\def\dvecbe{\vbox{\mathsurround=0pt\ialign{##\crcr
     \kern1pt$\scriptscriptstyle\rightharpoondown$\crcr\noalign{\kern-10pt\nointerlineskip}
     $\hfil\displaystyle{\be}\hfil$\crcr}}}

%   n
\def\uvecn{\vbox{\mathsurround=0pt\ialign{##\crcr
     $\scriptscriptstyle\rightharpoonup$\crcr\noalign{\kern1pt\nointerlineskip}
     $\hfil\displaystyle{n}\hfil$\crcr}}}
\def\dvecn{\vbox{\mathsurround=0pt\ialign{##\crcr
     $\scriptscriptstyle\rightharpoondown$\crcr\noalign{\kern-7pt\nointerlineskip}
     $\hfil\displaystyle{n}\hfil$\crcr}}}

%   m
\def\uvecm{\vbox{\mathsurround=0pt\ialign{##\crcr
     $\scriptscriptstyle\rightharpoonup$\crcr\noalign{\kern1pt\nointerlineskip}
     $\hfil\displaystyle{m}\hfil$\crcr}}}
\def\dvecm{\vbox{\mathsurround=0pt\ialign{##\crcr
     $\scriptscriptstyle\rightharpoondown$\crcr\noalign{\kern-7pt\nointerlineskip}
     $\hfil\displaystyle{m}\hfil$\crcr}}}

%   N
\def\uvecN{\vbox{\mathsurround=0pt\ialign{##\crcr
     \kern3pt$\scriptscriptstyle\rightharpoonup$\crcr\noalign{\kern1pt\nointerlineskip}
     $\hfil\displaystyle{N}\hfil$\crcr}}}
\def\dvecN{\vbox{\mathsurround=0pt\ialign{##\crcr
     \kern0pt$\scriptscriptstyle\rightharpoondown$\crcr\noalign{\kern-10pt\nointerlineskip}
     $\hfil\displaystyle{N}\hfil$\crcr}}}

%   u
\def\uvecu{\vbox{\mathsurround=0pt\ialign{##\crcr
     $\scriptscriptstyle\rightharpoonup$\crcr\noalign{\kern1pt\nointerlineskip}
     $\hfil\displaystyle{u}\hfil$\crcr}}}
\def\dvecu{\vbox{\mathsurround=0pt\ialign{##\crcr
     $\scriptscriptstyle\rightharpoondown$\crcr\noalign{\kern-7pt\nointerlineskip}
     $\hfil\displaystyle{u}\hfil$\crcr}}}

%   w
\def\uvecw{\vbox{\mathsurround=0pt\ialign{##\crcr
     $\scriptscriptstyle\rightharpoonup$\crcr\noalign{\kern1pt\nointerlineskip}
     $\hfil\displaystyle{w}\hfil$\crcr}}}
\def\dvecw{\vbox{\mathsurround=0pt\ialign{##\crcr
     $\scriptscriptstyle\rightharpoondown$\crcr\noalign{\kern-7pt\nointerlineskip}
     $\hfil\displaystyle{w}\hfil$\crcr}}}

%   v
\def\uvecv{\vbox{\mathsurround=0pt\ialign{##\crcr
     $\scriptscriptstyle\rightharpoonup$\crcr\noalign{\kern1pt\nointerlineskip}
     $\hfil\displaystyle{v}\hfil$\crcr}}}
\def\dvecv{\vbox{\mathsurround=0pt\ialign{##\crcr
     $\scriptscriptstyle\rightharpoondown$\crcr\noalign{\kern-7pt\nointerlineskip}
     $\hfil\displaystyle{v}\hfil$\crcr}}}

\def\astA{{}^\ast A}
\def\circA{{}^\circ A}
\def\astk{{}^\ast k}
\def\circk{{}^\circ k}
\def\astK{{}^\ast K}
\def\circK{{}^\circ K}
\def\astL{{}^\ast L}
\def\circL{{}^\circ L}
\def\astal{{}^\ast \al}
\def\circal{{}^\circ \al}
\def\astsi{{}^\ast \si}
\def\circsi{{}^\circ \si}
\def\aste{{}^\ast e}
\def\circe{{}^\circ e}
\def\astte{{}^\ast \te}
\def\circte{{}^\circ \te}
\def\astGa{{}^\ast \Ga}
\def\circGa{{}^\circ \Ga}
\def\Lie{\pounds}

\long\def\Hide#1{}
\long\def\HideMarked#1{\hfill{$\triangleright$}}
\def\rtau{\tau}


%%%%%%%%%%%%%%%%%%%%%%%%%%%%%%%%%%%%%%%%%%%%%%%%%%%%%%%%%%%%%%%
%\def\NormalStyle{\leftskip=2cm\rightskip=0cm\normalsize\parindent=5pt\parskip=3pt\normalbaselineskip=14pt\baselineskip=\normalbaselineskip}
%\def\AbstractStyle{\leftskip=3cm\rightskip=1cm\scriptsize\parindent=0pt\parskip=0pt\normalbaselineskip=11pt\baselineskip=\normalbaselineskip}
%\def\NoteStyle{\leftskip=3cm\rightskip=1cm\scriptsize\parindent=0pt\parskip=0pt\normalbaselineskip=11pt\baselineskip=\normalbaselineskip}

\def\ShowLabel#1{\ref{#1}}

\def\Bibliography{\begin{thebibliography}{199}\footnotesize\References}
\def\EndBibliography{\end{thebibliography}}
\def\bib#1#2{\bibitem{#1}#2}
\def\Ref#1{\cite{#1}}


\def\NewSection#1{\section{#1}}
\def\NewSubSection#1{\subsection*{#1}}
\def\NewAppendix#1#2{\section*{Appendix #1: #2}}
\def\Acknowledgements{\section*{Acknowledgements}}
\def\bs{\bigskip}
\def\ms{\medskip}
\def\ss{\smallskip}
\def\ni{\noindent}

\long\def\Note#1{\blockquote{\footnotesize {\bf Nota:}~#1\par}}
\long\def\Ex#1{\blockquote{\footnotesize {\bf Esercizio:}~#1\par}}

\def\eq#1{\begin{equation}#1\end{equation}}
\def\eqLabel#1#2{\begin{equation}#1\label{#2}\end{equation}}


\def\Cases#1{\begin{cases}#1\end{cases}}
\def\Matrix#1{\begin{matrix}#1\end{matrix}}
\def\Align#1{\begin{aligned}#1\end{aligned}}

\def\eqs#1{\eq{\Align{#1}}}
\def\eqsLabel#1#2{\eq{\Align{#1}\label{#2}}}




\def\Abstract{\AbstractStyle{\bf Abstract. }}
\def\EndAbstract{\par\NormalStyle}

\def\TitleScript{}
\def\TitleLine#1{\def\TitleScript{#1}}
\def\MoreTitleLine#1{\edef\TitleScript{\TitleScript\\#1}}

\def\AuthorScript{}
\def\AuthorLine#1{\def\AuthorScript{#1}}
\def\MoreAuthorLine#1{\edef\AuthorScript{\AuthorScript\\#1}}

\def\AddressScript{}
\def\AddressLine#1{\def\AddressScript{#1}}
\def\MoreAddressLine#1{\edef\AddressScript{\AddressScript\\#1}}

\date{}

\def\BeginDocument{\begin{document}}

\def\MakeTitle{
\begin{center}
{\Large\bf\sffamily\baselineskip=2pt\TitleScript}\\
\vskip 10pt
{\small by {\it \AuthorScript}}\\
\vskip 10pt
{\small \AddressScript}
\end{center}
\ms
}

\def\EndDocument{\end{document}}

\def\Figure[#1]#2{\begin{figure}[htbp] %  figure placement: here, top, bottom, or page
   \centering
   \includegraphics[#1]{#2} }
\def\Caption#1{\caption{#1}}
\def\EndFigure{\end{figure}}

\def\Diagram#1{\eq{
\begindc{\commdiag}[10]
#1
\enddc%
}%
}

\def\Itemize#1{\begin{itemize}#1\end{itemize}}
\def\Item[#1]{\item[#1]}

\def\AllReferences{}



\def\Shrink{\small}
%%%%%%%%%%%%%%%%%%%%%%%%%%%%%%%%%%%%%%%%%%%%%%%%%
%%%%%%%%%%%%%%%%%%%%%%%%%%%%%%%%%%%%%%%%%%%%%%%%%%%

%\BeginDocument


%%%%%%%%%%%%%%%%%%%%%%%%%%%%%%%%%%%%%%%%%%%%%%%%%%%






















\parindent=5pt
\baselineskip=13pt
\parskip=5pt





\begin{document}

\AllReferences


%%%%%%%%%%%%%%%%%%%%% Publisher's Area please ignore %%%%%%%%%%%%%%%
%
%
%%%%%%%%%%%%%%%%%%%%%%%%%%%%%%%%%%%%%%%%%%%%%%%%%%%%%%%%%%%%%%%%%%%%

\section*{Giorno 18: logica del prim'ordine}

Anche se non lo sappiamo, abbiamo un esempio di sistema assiomatico (gli assiomi di Peano per i numeri naturali) di un modello (i numeri naturali come cardinalità degli insiemi finiti).
Forse vale la pena di discutere un po' la semantica di tutto ciò.

\Note{Alcune cose che dirò sulla semantica della matematica sono opinioni personali. Quando quello che dico non è necessariamente condiviso provo a segnalarlo.}

Prima di dire cosa è un sistema assiomatico, serve un ambiente in cui sistemarlo. 
E questo ambiente, che chiamiamo {\it sistema formale}, è piuttosto familiare a chi fa informatica.
Poi specializziamo i sistemi formali in particolare al calcolo proposizionale (o logica proposizionale) e alla logica del prim'ordine.

\Note{Entrambi questi sistemi formali sono stati sviluppati per automatizzare il ragionamento nella seconda metà dell'1800.
Ci sono tante varianti ma alla fine in genera si usa quello che presentiamo come logica del prim'ordine.
}

Un {\it sistema formale} è fatto di:

un insieme finito chiamato {\it alfabeto} $\calA$ di caratteri, 

di liste di caratteri dette {\it parole}, che formano un vocabolario $\calV$, 

di frasi formate di parole secondo una ({\it context free}) grammatica.

\ms

Le frasi conformi alla grammatica si chiamano {\it predicati} o {\it proposizioni} a seconda del sistema formale che utilizziamo (vedi sotto).

Infine, abbiamo una lista di {\it regole di produzione}. 
Ogni regola di produzione è formata da $(n+1)$-predicati $(f_0, f_1, f_2, \dots, f_n)$.
I predicati $(f_1, f_2, \dots, f_n)$ si chiamano {\it premesse} o {\it ipotesi} della regola, mentre $f_0$ è detta {\it conseguenza} oppure {\it tesi}.
Una regola di produzione si indica come $[ f_1, f_2, \dots, f_n\arr f_0]$.

Le regole di produzione sono scelte in modo da automatizzare il ragionamento seguendo i principi della logica classica 
(o talvolta pure variazioni ad esempio la logica a più valori o altre varianti).

\Note{Un insieme di proposizioni è detto {\it chiuso} se quando contiene tutte le premesse di una regola di produzione, contengono pure la sua tesi.
Quindi se uno fornisce un insieme $\calP$ di proposizioni (assiomi), si può cercare di chiudere l'insieme $[\calP]$ 
applicando le regole di produzioni e aggiungendo le corrispondenti tesi.}

A noi interessano principalmente 2 esempi, che spiegano in modo semplice la definizione di sistema formale. 
Prima consideriamo la {\it logica proposizionale} e poi la {\it logica del prim'ordine} che in un certo senso è lo standard usato di solito.

Cominciamo con la logica proposizionale. 
Si intende che le proposizioni siano frasi (ad esempio nel linguaggio naturale) a cui si può attribuire un valore di verità, cioè sono o vere o false.
Non tutte le frasi sono così, {\it vai a cagare} o {\it vuoi più bene a mamma o a papà?} non sono proposizioni, 
{\it 10=3} o {\it Socrate è un uomo} sono proposizioni. Le proposizioni semplici (atomiche, che non contengono sotto-proposizioni sensate) si indicano con lettere latine maiuscole $A, B, \dots, P, Q, \dots$. Le proposizioni si possono poi combinare con i connettori logici, se $A$ e $B$ sono proposizioni, lo sono pure
$$
(A).and.(B)
\qquad
.not.(A)
\qquad
(A).or.(B)
\qquad
\dots
$$
Come sanno gli informatici tutti i connettori logici si possono definire utilizzando solo $(A).nand.(B) :=.not.( (A).and.(B))$.
Ad esempio, possiamo definire $.not.$ e $.or.$ come
$$
.not.(A)=(A).nand.(A)
\qquad
(A).or.(B)=   ( .not.( A )) .nand. ( .not. (B) )
$$
\Note{Potete controllare che questa definizione definisce $.or.$ davvero.
\begin{center}
\begin{tabular}{ccc}
$A$  & $B$ &  $(A).or.(B)$ \\
v & v &   v \\
v & f & v \\
f & v &  v \\
f & f &  f 
\end{tabular}
\end{center}

Forse bisogna specificare che esiste pure un operatore $(A)\then(B)$ che è definito come $(.not.(A)).or.(B)$.
Va detto perché siamo abituati a vederlo come una cosa logica che invece è definito come un operatore logico
\begin{center}
\begin{tabular}{ccc}
$A$  & $B$ &  $(A)\then(B)$ \\
v & v &   v \\
v & f & v \\
f & v &  f \\
f & f &  v
\end{tabular}
\end{center}

In pratica $(A)\then(B)$ è falso solo se $B$ è vero senza che lo sia $A$ che se ci pensate è l'unico caso in cui possiamo 
escludere che $B$ sia implicato da $A$ (se $A$ è vero deve essere vero pure $B$, se $A$ è falso allora $B$ è libero di fare quello che vuole).

}
 
Il riassunto è che potete creare tutti gli operatori logici che volete a partire dal $.nand.$ solamente.
Quindi se la grammatica specifica che 
$$
[Proposition] \arr [Proposition] .nand. [Proposition] 
$$ 
allora possiamo combinare proposizioni con ogni operatore logico.

Per noi $.not.(A)$, $(A).and.(B)$, $(A).or.(B)$, $(A)\then(B)$, \dots sono proposizioni se $A$ e $B$ lo sono.  


Le regole di produzione sono 
$$
[ (P)\then(Q), P  \arr Q]
\hbox to 6cm{\hfill (modus ponens)}
$$
da leggere: se $(P)\then(Q)$ e $P$ sono in $[\calP]$ allora è $Q$ in $[\calP]$.

Esempi delle altre regole sono
$$
\Align{
[ (P)\then(Q), .not.(Q)  \arr .not.(P)]&   \hbox to 5cm{\hfill (modus tollens)}\cr
[ (P).or.(Q), .not.(P)  \arr Q]&   \hbox to 5cm{\hfill}\cr
[P, Q \arr (P).and.(Q)] &\cr
\dots&
}
$$

Le regole di produzione sono abbastanza numerose (in fondo definiscono implicitamente tutti i connettori logici che sono $2^4=16$ tanti quante le tabelle di verità).
Ma il punto è la proprietà che si usa per sceglierle. Controllate pure, ma tutte sono valide indipendentemente dai valori di verità delle proposizioni che contengono.
In fondo se per giudicare se un ragionamento è corretto dovessimo sapere si cosa sta parlando non servirebbe la logica basterebbe applicare gli operatori logici.

\Note{Prendiamo come esempio il modus tollens $[ (P)\then(Q), .not.(Q)  \arr .not.(P)]$. Compaiono 2 proposizioni $P$ e $Q$ che possono essere entrambe, indipendentemente, vere o false.
\begin{center}
\begin{tabular}{cc}
$P$  	& $Q$		\\
v 	& v 		\\
v 	& f 		\\
f 	& v 		\\
f 	& f 		
\end{tabular}
\qquad\qquad
\begin{tabular}{ccc}
 $(P)\then(Q)$	& 	$.not.(Q)$	& $.not.(P)$	\\
  v 			&		f	&	f\\
  f 			&		v	&	f\\
  v 			&		f	&	v\\
  v 			&		v	&	v
\end{tabular}
\end{center}
Vedete che se le 2 premesse sono vere allora è vera pure la tesi.

Proviamo pure con $[P, Q \arr (P).and.(Q)]$
\begin{center}
\begin{tabular}{cc}
$P$  	& $Q$		\\
v 	& v 			\\
v 	& f 			\\
f 	& v 			\\
f 	& f 					
\end{tabular}
\qquad\qquad
\begin{tabular}{ccc}
$P$  	& $Q$	& $(P).and.(Q)$	\\
v 	& v 		&  v 				\\
v 	& f 		&  f 				\\
f 	& v 		&  f 				\\
f 	& f 		&  f 				
\end{tabular}
\end{center}
Di nuovo se le premesse sono vere allora la tesi è vera.

Come ho detto, non ho bisogno di sapere se le proposizioni sono vere o false. Diciamo che una regola è {\it valida} se e solo se la conseguenza è vera quando sono vere tutte le premesse
(non necessariamente quando sono vere $P$ e $Q$ che sono contenute nelle premesse).
}

Essere un ragionamento valido quindi non dice nulla del valore delle proposizioni che vi compaiono. 
In particolare, se abbiamo una premessa certamente falsa ($(A).and.(.non.(A))$) possiamo ricavare da questa in modo valido qualunque proposizione.


Se consideriamo un insieme chiuso $[\calP]$, se abbiamo che $A$ e $.not.(A)$ sono in $[\calP]$ allora l'insieme (o gli assiomi $\calP$ da cui deriva) è detto incoerente.
Un sistema di assiomi  è quindi  {\it coerente} se non consente di ricavare due proposizioni contraddittorie.


Se di ogni proposizione $A$ possiamo ricavare $A$ o $.not.(A)$ allora il sistema si dice {\it completo}.

Se nessun assioma può essere derivato dagli altri, il sistema si dice {\it indipendente}.

\ms

La {\it logica del prim'ordine} è un'estensione della logica proposizionale. Ha 2 ingredienti in più:

le variabili (indicate con lettere minuscole $x, y, \dots$) da cui i predicati $P(x)$ possono dipendere,

i quantificatori (per ogni $x$ ed esiste un $x$) che si può anteporre a un predicato per ottenere proposizioni.


\Note{
($P$: {\it 7 è primo}) è una proposizione, ($P(x)$: {\it x è primo}) è un predicato. Non possiamo dire se $P(x)$ è vera o falsa finché non sappiamo chi è $x$.
$P=P(7)$ è una proposizione, ma anche $\exists x: P(x)$ è una proposizione che afferma che esiste (almeno) un numero primo, mentre $\forall x:P(x)$
è una proposizione (falsa) che afferma che tutti i numeri naturali sono primi.
In genere le variabili $x, y$ sono elementi di un insieme (e.g.~$x\in \N$) che è l'ambito in cui stiamo lavorando.
}

Sia la logica proposizionale che la logica del prim'ordine hanno 2 scopi, quello di rappresentare i ragionamenti (come lo spartito rappresenta la musica)
e quello di definire i ragionamenti validi, cioè quelli che garantiscono che {\it se sono vere le ipotesi allora è vera pure la tesi}.

Il prossimo passo è specificare quest'ambito (la logica del prim'ordine) i sistemi assiomatici e i modelli.



\EndDocument
\end


