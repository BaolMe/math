% !TEX TS-program = latex
% !TEX spellcheck = it-IT
\documentclass[a4paper]{article}
\usepackage[utf8]{inputenc}
\usepackage[T1]{fontenc}
\usepackage[italian]{babel}
\usepackage{tikz} % LATEX
\usetikzlibrary {automata}


%%%%%%%%%%%%%%%%%%%%%%%%%%%%%%%%%%%%%%%%%%%%%%%%%%%%%%%%%%%%%%%%%%%%%%%%%%%%
%% Trim Size: 9.75in x 6.5in
%% Text Area: 8in (include Runningheads) x 5in
%% ws-ijgmmp.tex   :   2-9-08
%% Tex file to use with ws-ijgmmp.cls written in Latex2E.
%% The content, structure, format and layout of this style file is the
%% property of World Scientific Publishing Co. Pte. Ltd.
%% Copyright 1995, 2002 by World Scientific Publishing Co.
%% All rights are reserved.
%%%%%%%%%%%%%%%%%%%%%%%%%%%%%%%%%%%%%%%%%%%%%%%%%%%%%%%%%%%%%%%%%%%%%%%%%%%%
%%

%%%%%%%%%%%%%%%%%%%%%%%%%%%%%


\usepackage{pictexwd,dcpic}
\usepackage{csquotes}
\usepackage{nopageno}

\usepackage{amsmath, amssymb, amsthm}
\usepackage{pictex, dcpic}
\usepackage{color}
%\usepackage{graphicx}
%\numberwithin{equation}{section} % This line resets equation numbering when starting a new section.
%\renewcommand{\theequation}{Eq. \thesection.\arabic{equation}} % This line ads "Eq." in front of your equation numbering.




%
%    Macros.    Version 1.2.0.beta
%    The best use is to paste all of them into the papers
%     1/8/2005
%

%%%%%%%%%%%%%%%%%%%%%%%%%%%%%%
%%%%%%			Greek		 %%%%%%
%%%%%%%%%%%%%%%%%%%%%%%%%%%%%%

\def\al{\alpha}
\def\be{\beta}
\def\de{\delta}
\def\ga{\gamma}
\def\up{\upsilon}
\def\ep{\epsilon}
\def\io{\iota}
\def\te{\theta}
\def\la{\lambda}
\def\ze{\zeta}
\def\om{\omega}
\def\si{\sigma}
\def\vp{\varphi}
\def\vpi{\varpi}
\def\ka{\kappa}
\def\vs{\varsigma}
\def\vr{\varrho}

\def\De{\Delta}
\def\Ga{\Gamma}
\def\Te{\Theta}
\def\La{\Lambda}
\def\Om{\Omega}
\def\Si{\Sigma}
\def\Up{\Upsilon}

\def\boldDe{\bf\Delta}
\def\boldGa{\bf\Gamma}
\def\boldTe{\bf\Theta}
\def\boldLa{\bf\Lambda}
\def\boldOm{\bf\Omega}
\def\boldSi{\bf\Sigma}
\def\boldUp{\bf\Upsilon}
\def\boldXi{\bf\Xi}
\def\boldPi{\bf\Pi}
\def\boldPhi{\bf\Phi}
\def\boldPsi{\bf\Psi}

\def\boldal{\bf\al}
\def\boldbe{\bf\be}
\def\boldga{\bf\ga}
\def\boldde{\bf\de}
\def\boldep{\bf\ep}
\def\boldze{\bf\ze}
\def\boldeta{\bf\eta}
\def\boldte{\bf\te}
\def\boldio{\bf\io}
\def\boldka{\bf\ka}
\def\boldla{\bf\la}
\def\boldmu{\hbox{\greekbold\char "16}}
\def\boldnu{\hbox{\greekbold\char "17}}
\def\boldxi{\hbox{\greekbold\char "18}}
\def\boldpi{\hbox{\greekbold\char "19}}
\def\boldrho{\hbox{\greekbold\char "1A}}
\def\boldsi{\bf\si}
\def\boldtau{\hbox{\greekbold\char "1C}}
\def\boldup{\hbox{\greekbold\char "1D}}
\def\boldphi{\hbox{\greekbold\char "1E}}
\def\boldchi{\hbox{\greekbold\char "1F}}
\def\boldpsi{\hbox{\greekbold\char "20}}
\def\boldom{\hbox{\greekbold\char "21}}
\def\boldvep{\hbox{\greekbold\char "22}}
\def\boldvte{\hbox{\greekbold\char "23}}
\def\boldvpi{\hbox{\greekbold\char "24}}
\def\boldvrho{\hbox{\greekbold\char "25}}
\def\boldvsi{\hbox{\greekbold\char "26}}
\def\boldvphi{\hbox{\greekbold\char "27}}

%%%%%%%%%%%%%%%%%%%%%%%%%%%%%%
%%%%%%			Cal			 %%%%%%
%%%%%%%%%%%%%%%%%%%%%%%%%%%%%%
 \def\calA{{\hbox{\cal A}}}
 \def\calU{{\hbox{\cal U}}}
 \def\calB{{\hbox{\cal B}}}
 \def\calC{{\hbox{\cal C}}}
 \def\calI{{\hbox{\cal I}}}
 \def\calQ{{\hbox{\cal Q}}}
 \def\calP{{\hbox{\cal P}}}
 \def\calL{{\hbox{\cal L}}}
 \def\calE{{\hbox{\cal E}}}
 \def\calW{{\hbox{\cal W}}}
 \def\calV{{\hbox{\cal V}}}
 \def\calS{{\hbox{\cal S}}}
 \def\calF{{\hbox{\cal F}}}
 \def\calR{{\hbox{\cal R}}}
 \def\calO{{\hbox{\cal O}}}
 \def\calM{{\hbox{\cal M}}}

%%%%%%%%%%%%%%%%%%%%%%%%%%%%%%
%%%%%%			gothic		 %%%%%%
%%%%%%%%%%%%%%%%%%%%%%%%%%%%%%
 \def\su{{\mathfrak{su}}}
 \def\gl{{\mathfrak{gl}}}
 \def\gotg{{\mathfrak{g}}}
 \def\goth{{\mathfrak{h}}}
 \def\gotm{{\mathfrak{m}}}
 \def\gotk{{\mathfrak{k}}}
 \def\spin{{\mathfrak{spin}}}
 \def\slC{{\mathfrak{sl}}}
 \def\O{{\mathfrak{O}}}
 \def\Set{{\mathfrak{Set}}}

%%%%%%%%%%%%%%%%%%%%%%%%%%%%%%
%%%%%%			Bbb			 %%%%%%
%%%%%%%%%%%%%%%%%%%%%%%%%%%%%%
 \def\one{\mathbb{I}}
 \def\A{\mathbb{A}}
 \def\B{\mathbb{B}}
 \def\C{\mathbb{C}}
 \def\D{\mathbb{D}}
 \def\E{\mathbb{E}}
 \def\F{\mathbb{F}}
 \def\G{\mathbb{G}}
 \def\H{\mathbb{H}}
 \def\J{\mathbb{J}}
 \def\K{\mathbb{K}}
 \def\I{\mathbb{I}}
 \def\L{\mathbb{L}}
 \def\M{\mathbb{M}}
 \def\N{\mathbb{N}}
 \def\O{\mathbb{O}}
 \def\P{\mathbb{P}}
 \def\Q{\mathbb{Q}}
 \def\R{\mathbb{R}}
 \def\S{\mathbb{S}}
 \def\T{\mathbb{T}}
 \def\U{\mathbb{U}}
 \def\V{\mathbb{V}}
 \def\X{\mathbb{X}}
 \def\Y{\mathbb{Y}}
 \def\W{\mathbb{W}}
 \def\Z{\mathbb{Z}}
 

%%%%%%%%%%%%%%%%%%%%%%%%%%%%%%
%%%%%%		MathRoman		 %%%%%%
%%%%%%%%%%%%%%%%%%%%%%%%%%%%%%
\def\Tr{{\hbox{Tr}}}
\def\Con{{\hbox{Con}}}
\def\Aut{{\hbox{Aut}}}
\def\Div{{\hbox{Div}}}
\def\ad{{\hbox{ad}}}
\def\Ad{{\hbox{Ad}}}
\def\Re{{\hbox{Re}}}
\def\Im{{\hbox{Im}}}
\def\Card{{\hbox{Card}}}
\def\Iso{{\hbox{Iso}}}
\def\Geo{{\hbox{Geo}}}
\def\Int{{\hbox{Int}}}
\def\Inv{{\hbox{Inv}}}
\def\Spin{{\hbox{Spin}}}
\def\SO{{\hbox{SO}}}
\def\SU{{\hbox{SU}}}
\def\SL{{\hbox{SL}}}
\def\GL{{\hbox{GL}}}
\def\det{{\hbox{det}}}
\def\Hom{{\hbox{Hom}}}
\def\End{{\hbox{End}}}
\def\Euc{{\hbox{Euc}}}
\def\Lor{{\hbox{Lor}}}
\def\Diff{{\hbox{Diff}}}
\def\di{{\hbox{d}}}
\def\id{{\hbox{\rm id}}}
\def\diag{{\hbox{diag}}}
\def\rank{{\hbox{rank}}}
\def\span{{\hbox{span}}}
\def\Sim{{\hbox{Sim}}}

\def\Obj{{\hbox{Obj}}}

%%%%%%%%%%%%%%%%%%%%%%%%%%%%%%
%%%%%%		OtherSymbols		 %%%%%%
%%%%%%%%%%%%%%%%%%%%%%%%%%%%%%
\def\ip{\hbox to4pt{\leaders\hrule height0.3pt\hfill}\vbox to8pt{\leaders\vrule width0.3pt\vfill}\kern 2pt}

% inner product
\def\QDE{\hfill\hbox{\ }\vrule height4pt width4pt depth0pt} 
\def\del{\partial}
\def\na{\nabla}
\def\inter{\cap}
\def\Vec{\mathfrak{X}}
\def\Lie{\hbox{\LieFont \$}}

\def\arr{\rightarrow}
\def\larr{\longrightarrow}
\def\harr{\hookrightarrow}
\def\hlarr{\lhook\joinrel\longrightarrow}
\def\then{\Rightarrow}
\def\semidirect{\hbox{\Bbb \char111}}
\def\binomial#1#2{\left(\Matrix{#1\cr #2\cr}\right)}

\def\barJ{\bar J}

\def\sSU{{\hbox{SU}}}
\def\so{{so}}
%\def\N{{{\mathbb N}}}

\def\file#1{{\tt #1}}
\def\calH{{{\cal H}}}
\def\calJ{{{\cal J}}}
\def\calA{{{\cal A}}}
\def\calV{{{\cal V}}}
\def\calG{{{\cal G}}}
\def\calP{{{\cal P}}}
\def\calK{{{\cal K}}}
\def\calD{{{\cal D}}}
\def\scalD{{{\cal D}}}

\def\Loop{{\hbox{Loop}}}
\def\Hoop{{\hbox{Hoop}}}
\def\Sym{{\hbox{Sym}}}
\def\sR{{\R}}
\def\dCor{dCor}
\def\epm{\hbox{$^\pm$}}
\def\dunion{\coprod}

\def\nab#1{{\buildrel #1\over \na}}
\def\frac[#1/#2]{\hbox{$#1\over#2$}}
\def\Frac[#1/#2]{{#1\over#2}}
\def\({\left(}
\def\){\right)}
\def\[{\left[}
\def\]{\right]}
\def\^#1{{}^{#1}_{\>\cdot}}
\def\_#1{{}_{#1}^{\>\cdot}}
\def\Label=#1{{\buildrel {\hbox{\fiveSerif \ShowLabel{#1}}}\over =}}
\def\<{\kern -1pt}
\def\Bar{\>|\>}
\def\Dal{\hbox{\tenRelazioni  \char003}}

\def\uvec#1{\vbox{\hbox{$\scriptstyle\rightharpoonup$}\vskip-9pt\hbox{$#1$}}}
\def\dvec#1{\vtop{\hbox{$#1$}\vskip-10pt\hbox{$\scriptstyle\rightharpoondown$}}}
\def\Cprod{\diamond}
\def\obullet{\odot}
\def\cbrt{{}^3\kern-5pt\sqrt}



%%%%%%%%%%%%			frames 				%%%%%%%%%%%%%%%%%%%

\def\red#1{{\color{red}{#1}}}
\def\blue#1{{\color{blue}{#1}}}
\def\green#1{{\color{green}{#1}}}

\def\Red{\color{red}}
\def\Blue{\color{blue}}
\def\Green{\color{green}}


\def\frame#1{\vbox{\hrule\hbox{\vrule\vbox{\kern2pt\hbox{\kern2pt#1\kern2pt}\kern2pt}\vrule}\hrule\kern-4pt}} 
\def\redframe#1{\red{\frame{#1}}} 
\def\greenframe#1{\green{\frame{#1}}} 
\def\blueframe#1{\blue{\frame{#1}}} 

\def\uline#1{\underline{#1}}
\def\uuline#1{\underline{\underline{#1}}}
\def\Box to #1#2#3{\frame{\vtop{\hbox to #1{\hfill #2 \hfill}\hbox to #1{\hfill #3 \hfill}}}}



\def\ubal{\underline{\al}\kern1pt}
\def\obal{\overline{\al}\kern1pt}

\def\ubR{\underline{R}\kern1pt}
\def\obR{\overline{R}\kern1pt}
\def\ubom{\underline{\om}\kern1pt}
\def\obxi{\overline{\xi}\kern1pt}
\def\ubu{\underline{u}\kern1pt}
\def\ube{\underline{e}\kern1pt}
\def\obe{\overline{e}\kern1pt}
\def\Limit{\>{\buildrel{r\arr\infty}\over \longrightarrow}\,}
\def\union{\cup}
\def\Emptyset{\varnothing}




\def\Uvec#1{\vbox{\mathsurround=0pt\ialign{##\crcr
     $\scriptscriptstyle\rightharpoonup$\crcr\noalign{\kern1pt\nointerlineskip}
     $\hfil\displaystyle{#1}\hfil$\crcr}}}
\def\Dvec#1{\vbox{\mathsurround=0pt\ialign{##\crcr
     $\scriptscriptstyle\rightharpoondown$\crcr\noalign{\kern-7pt\nointerlineskip}
     $\hfil\displaystyle{#1}\hfil$\crcr}}}


%   u 
\def\uvecu{\vbox{\mathsurround=0pt\ialign{##\crcr
     $\scriptscriptstyle\rightharpoonup$\crcr\noalign{\kern1pt\nointerlineskip}
     $\hfil\displaystyle{u}\hfil$\crcr}}}
\def\dvecu{\vbox{\mathsurround=0pt\ialign{##\crcr
     $\scriptscriptstyle\rightharpoondown$\crcr\noalign{\kern-7pt\nointerlineskip}
     $\hfil\displaystyle{u}\hfil$\crcr}}}
%\def\uvecu{\Uvec{u}}
%\def\dvecu{\Dvec{u}}

%   be
\def\uvecbe{\vbox{\mathsurround=0pt\ialign{##\crcr
     \kern3pt$\scriptscriptstyle\rightharpoonup$\crcr\noalign{\kern1pt\nointerlineskip}
     $\hfil\displaystyle{\be}\hfil$\crcr}}}
\def\dvecbe{\vbox{\mathsurround=0pt\ialign{##\crcr
     \kern1pt$\scriptscriptstyle\rightharpoondown$\crcr\noalign{\kern-10pt\nointerlineskip}
     $\hfil\displaystyle{\be}\hfil$\crcr}}}

%   n
\def\uvecn{\vbox{\mathsurround=0pt\ialign{##\crcr
     $\scriptscriptstyle\rightharpoonup$\crcr\noalign{\kern1pt\nointerlineskip}
     $\hfil\displaystyle{n}\hfil$\crcr}}}
\def\dvecn{\vbox{\mathsurround=0pt\ialign{##\crcr
     $\scriptscriptstyle\rightharpoondown$\crcr\noalign{\kern-7pt\nointerlineskip}
     $\hfil\displaystyle{n}\hfil$\crcr}}}

%   m
\def\uvecm{\vbox{\mathsurround=0pt\ialign{##\crcr
     $\scriptscriptstyle\rightharpoonup$\crcr\noalign{\kern1pt\nointerlineskip}
     $\hfil\displaystyle{m}\hfil$\crcr}}}
\def\dvecm{\vbox{\mathsurround=0pt\ialign{##\crcr
     $\scriptscriptstyle\rightharpoondown$\crcr\noalign{\kern-7pt\nointerlineskip}
     $\hfil\displaystyle{m}\hfil$\crcr}}}

%   N
\def\uvecN{\vbox{\mathsurround=0pt\ialign{##\crcr
     \kern3pt$\scriptscriptstyle\rightharpoonup$\crcr\noalign{\kern1pt\nointerlineskip}
     $\hfil\displaystyle{N}\hfil$\crcr}}}
\def\dvecN{\vbox{\mathsurround=0pt\ialign{##\crcr
     \kern0pt$\scriptscriptstyle\rightharpoondown$\crcr\noalign{\kern-10pt\nointerlineskip}
     $\hfil\displaystyle{N}\hfil$\crcr}}}

%   u
\def\uvecu{\vbox{\mathsurround=0pt\ialign{##\crcr
     $\scriptscriptstyle\rightharpoonup$\crcr\noalign{\kern1pt\nointerlineskip}
     $\hfil\displaystyle{u}\hfil$\crcr}}}
\def\dvecu{\vbox{\mathsurround=0pt\ialign{##\crcr
     $\scriptscriptstyle\rightharpoondown$\crcr\noalign{\kern-7pt\nointerlineskip}
     $\hfil\displaystyle{u}\hfil$\crcr}}}

%   w
\def\uvecw{\vbox{\mathsurround=0pt\ialign{##\crcr
     $\scriptscriptstyle\rightharpoonup$\crcr\noalign{\kern1pt\nointerlineskip}
     $\hfil\displaystyle{w}\hfil$\crcr}}}
\def\dvecw{\vbox{\mathsurround=0pt\ialign{##\crcr
     $\scriptscriptstyle\rightharpoondown$\crcr\noalign{\kern-7pt\nointerlineskip}
     $\hfil\displaystyle{w}\hfil$\crcr}}}

%   v
\def\uvecv{\vbox{\mathsurround=0pt\ialign{##\crcr
     $\scriptscriptstyle\rightharpoonup$\crcr\noalign{\kern1pt\nointerlineskip}
     $\hfil\displaystyle{v}\hfil$\crcr}}}
\def\dvecv{\vbox{\mathsurround=0pt\ialign{##\crcr
     $\scriptscriptstyle\rightharpoondown$\crcr\noalign{\kern-7pt\nointerlineskip}
     $\hfil\displaystyle{v}\hfil$\crcr}}}

\def\astA{{}^\ast A}
\def\circA{{}^\circ A}
\def\astk{{}^\ast k}
\def\circk{{}^\circ k}
\def\astK{{}^\ast K}
\def\circK{{}^\circ K}
\def\astL{{}^\ast L}
\def\circL{{}^\circ L}
\def\astal{{}^\ast \al}
\def\circal{{}^\circ \al}
\def\astsi{{}^\ast \si}
\def\circsi{{}^\circ \si}
\def\aste{{}^\ast e}
\def\circe{{}^\circ e}
\def\astte{{}^\ast \te}
\def\circte{{}^\circ \te}
\def\astGa{{}^\ast \Ga}
\def\circGa{{}^\circ \Ga}
\def\Lie{\pounds}

\long\def\Hide#1{}
\long\def\HideMarked#1{\hfill{$\triangleright$}}
\def\rtau{\tau}


%%%%%%%%%%%%%%%%%%%%%%%%%%%%%%%%%%%%%%%%%%%%%%%%%%%%%%%%%%%%%%%
%\def\NormalStyle{\leftskip=2cm\rightskip=0cm\normalsize\parindent=5pt\parskip=3pt\normalbaselineskip=14pt\baselineskip=\normalbaselineskip}
%\def\AbstractStyle{\leftskip=3cm\rightskip=1cm\scriptsize\parindent=0pt\parskip=0pt\normalbaselineskip=11pt\baselineskip=\normalbaselineskip}
%\def\NoteStyle{\leftskip=3cm\rightskip=1cm\scriptsize\parindent=0pt\parskip=0pt\normalbaselineskip=11pt\baselineskip=\normalbaselineskip}

\def\ShowLabel#1{\ref{#1}}

\def\Bibliography{\begin{thebibliography}{199}\footnotesize\References}
\def\EndBibliography{\end{thebibliography}}
\def\bib#1#2{\bibitem{#1}#2}
\def\Ref#1{\cite{#1}}


\def\NewSection#1{\section{#1}}
\def\NewSubSection#1{\subsection*{#1}}
\def\NewAppendix#1#2{\section*{Appendix #1: #2}}
\def\Acknowledgements{\section*{Acknowledgements}}
\def\bs{\bigskip}
\def\ms{\medskip}
\def\ss{\smallskip}
\def\ni{\noindent}

\long\def\Note#1{\blockquote{\footnotesize {\bf Nota:}~#1\par}}
\long\def\Ex#1{\blockquote{\footnotesize {\bf Esercizio:}~#1\par}}

\def\eq#1{\begin{equation}#1\end{equation}}
\def\eqLabel#1#2{\begin{equation}#1\label{#2}\end{equation}}


\def\Cases#1{\begin{cases}#1\end{cases}}
\def\Matrix#1{\begin{matrix}#1\end{matrix}}
\def\Align#1{\begin{aligned}#1\end{aligned}}

\def\eqs#1{\eq{\Align{#1}}}
\def\eqsLabel#1#2{\eq{\Align{#1}\label{#2}}}




\def\Abstract{\AbstractStyle{\bf Abstract. }}
\def\EndAbstract{\par\NormalStyle}

\def\TitleScript{}
\def\TitleLine#1{\def\TitleScript{#1}}
\def\MoreTitleLine#1{\edef\TitleScript{\TitleScript\\#1}}

\def\AuthorScript{}
\def\AuthorLine#1{\def\AuthorScript{#1}}
\def\MoreAuthorLine#1{\edef\AuthorScript{\AuthorScript\\#1}}

\def\AddressScript{}
\def\AddressLine#1{\def\AddressScript{#1}}
\def\MoreAddressLine#1{\edef\AddressScript{\AddressScript\\#1}}

\date{}

\def\BeginDocument{\begin{document}}

\def\MakeTitle{
\begin{center}
{\Large\bf\sffamily\baselineskip=2pt\TitleScript}\\
\vskip 10pt
{\small by {\it \AuthorScript}}\\
\vskip 10pt
{\small \AddressScript}
\end{center}
\ms
}

\def\EndDocument{\end{document}}

\def\Figure[#1]#2{\begin{figure}[htbp] %  figure placement: here, top, bottom, or page
   \centering
   \includegraphics[#1]{#2} }
\def\Caption#1{\caption{#1}}
\def\EndFigure{\end{figure}}

\def\Diagram#1{\eq{
\begindc{\commdiag}[10]
#1
\enddc%
}%
}

\def\Itemize#1{\begin{itemize}#1\end{itemize}}
\def\Item[#1]{\item[#1]}

\def\AllReferences{}



\def\Shrink{\small}
%%%%%%%%%%%%%%%%%%%%%%%%%%%%%%%%%%%%%%%%%%%%%%%%%
%%%%%%%%%%%%%%%%%%%%%%%%%%%%%%%%%%%%%%%%%%%%%%%%%%%

%\BeginDocument


%%%%%%%%%%%%%%%%%%%%%%%%%%%%%%%%%%%%%%%%%%%%%%%%%%%






















\parindent=5pt
\baselineskip=13pt
\parskip=5pt





\begin{document}

\AllReferences


%%%%%%%%%%%%%%%%%%%%% Publisher's Area please ignore %%%%%%%%%%%%%%%
%
%
%%%%%%%%%%%%%%%%%%%%%%%%%%%%%%%%%%%%%%%%%%%%%%%%%%%%%%%%%%%%%%%%%%%%

\section*{Giorno 27: serie, successioni e limiti}


Una {\it successione} (nei numeri razionali) è una funzione $a:\N\arr \Q: k\mapsto a_k$. 
Ad esempio, dato $z\in \Q$, $a:k\mapsto z^k$ è una successione, $b:k\mapsto \frac[z/k+1]$ pure.

Un {\it intervallo} di $\Q$ è un sottoinsieme $B\in \Q$ tale che se $q_0, q_1\in B$ allora per ogni $q\in \Q$ tale che $q_0<q<q_1$ abbiamo che $q\in B$.
Ad esempio, $(q_0, q_1)=\{q\in \Q: q_0<q<q_1\}$ sono intervalli.

Una {\it intorno di $q\in\Q$} è un intervallo che contiene $q\in \Q$. 

Un {\it aperto} in $\Q$ è un sottoinsieme $U\subset \Q$ tale che per ogni $q\in U$ esiste un intorno $B_q$ di $q$ contenuto in $B_q\subset U$.

\Note{Si noti che $\Emptyset$ è un aperto di $\Q$, e che $\Q$ è pure un aperto di $\Q$.

L'intersezione di 2 aperti $U_1\inter U_2$ è un aperto.

L'unione di 2 aperti $U_1\union U_2$ è un aperto.

Data una famiglia $U_\al$ con $\al\in I$ di aperti, magari una famiglia infinita (numerabile o non numerabile), allora
\eq{
U=\union_{\al\in I} U_\al = \{q\in \Q: \exists \al\in I: q\in U_\al\}
}
è pure un aperto.

Infatti se $q\in U$, allora $q\in U_\al$ per qualche $\al\in I$. Ma $U_\al$ è aperto, quindi per ogni $q_\al\in U_\al$ esiste un intorno $V_\al \subset U_\al$.
Ma se $V_\al \subset U_\al$, allora $V_\al \subset U_\al\subset U$.
In particolare esiste un intorno $V$ di $q$ contenuto in $U_\al\subset U$.
Quindi $U$ contiene un intorno di di ogni suo punto e, per definizione, è aperto.

[Se lo leggete 1 parola al minuto invece che i soliti 1 parole al secondo, credo dovrebbe essere sufficientemente convincente.]

Questo definisce una topologia su $\Q$, detta la {\it topologia standard}, o la {\it topologia delle palle}.
Per ora non sappiamo cosa sia una topologia (che corrisponde a definire gli aperti di un insieme). 
Per ora prendetelo come un modello di quello che diventerà un sistema assiomatico dopo. 
Ma nel modello possiamo dimostrare le proprietà come teoremi.
Le proprietà enunciate sopra diventeranno gli assiomi della topologia, che però si applica a insiemi molto più generali di $\Q$.

Poi questo è un modello di topologia è infatti ha anche teoremi che sono dimostrati in questo caso che non sono veri in generale ma solo per specifici tipi di topologie.
Ad esempio questo modello è una topologia metrica, ma esistono topologie non metriche che possono non condividere tutte le proprietà di quelle metriche. 
}

Possiamo fare circa la stessa cosa in $\N$. Un {\it intorno di $\infty$} è un sottoinsieme $I_n=\{N\in \N: n<N\}\subset \N$.

Ora data una successione $a:\N\arr \Q$ diciamo che 
\eq{
\lim_{k\arr \infty} a_k = q\in \Q
} e diciamo che il limite all'infinito della successione è uguale a $q\in \Q$ se
per ogni intorno $B_q$ esiste un intorno $I_n$ tale che per ogni $n\in I_k$ abbiamo che $a_n\in B_q$.

\Note{Ricordate quando vi dicevo che i quantificatori sono importanti? nella definizione di limite di una successione ce ne sono 3 e se non li ricordate come sono la definizione viene compromessa.
Per la stessa ragione, un testo matematico è ridondante per quel che riguarda il quantificatori rispetto ai testi non matematici.
I lettori tendono a saltare le cose che non si aspettano e quindi a non capire una mazza di quello che c'è scritto. Prendetevi il tempo perché il 90\% delle difficoltà viene dal voler leggere un testo matematico come se fosse una poesia.
}

Ad esempio, consideriamo la successione $a:k\mapsto a_k=1-\frac[3/k]$ e prendiamo $q=1$.
Vogliamo mostrare che $\lim_{k\arr \infty} a_k = 1$. 
Dobbiamo quindi mostrare che per ogni intorno $(q_0, q_1)$ (che è un intorno di 1 se e solo se $q_0<1<q_1$)
esiste un intorno di infinito $I_n$ tale che $a_N \in (q_0, q_1)$  quando $N\in I_n$, cioè ogni qual volta che $n<N$.

\Note{Se deve essere $a_k \in (q_0, q_1)$, significa che deve essere $q_0<1-\frac[3/k]<q_1$.
Siccome per ogni $k\in \N$ abbiamo che $a_k=1-\frac[3/k]<1<q_1$ allora $a_k<q_1$ è sempre verificata.

Se deve essere $q_0<a_k= 1-\frac[3/k]$, questo equivale a chiedere
\eqs{
q_0< 1-\frac[3/k]
\qquad
 \frac[3/k]< 1-q_0
\qquad
 \frac[3/k]< 1-q_0
\qquad\qquad
k> \frac[3/1-q_0]
}
Siccome $q_0<1$, allora $1-q_0>0$, allora $\frac[3/1-q_0]>0$. Qualunque sia il valore di $\frac[3/1-q_0] \in \Q$ esistono sempre interi positivi $k> \frac[3/1-q_0]$
Basta quindi prendere $n> \frac[3/1-q_0]$ come uno di questi interi per vere che $N\in I_n$ implichi che $a_N\in (q_0, q_1)$.
Questo è quanto dovevamo mostrare per affermare che $\lim_{k\arr \infty} a_k = 1$.

\ms
Se invece proviamo a mostrare che $\lim_{k\arr \infty} a_k = 0$, dovremmo mostrare che esiste un intorno di $I_0= (q_0, q_1)$ con $q_0<0<q_1$ tale che 
esiste un intorno di infinito $I_n$ tale che $N\in I_n$ implica che $a_N\in I_0$, cioè $q_0<a_N<q_1$.
come sopra $a_N>q_0$ è sempre vero, visto che $a_N>0$ e $q_0<0$.

Invece per avere $a_N<q_1$ dovrebbe essere $k< \frac[3/1-q_1]$. 
Ma all'aumentare di $k$ questo limite viene violato, e quindi presto o tardi $a_N\ge q_1$ e così non possiamo concludere che $\lim_{k\arr \infty} a_k = 0$.

In generale possiamo mostrare che se $\lim_{k\arr \infty} a_k = q$ allora non può essere $\lim_{k\arr \infty} a_k = q'\not=q$.
}

Quando diciamo che $\lim_{k\arr \infty} a_k = q$ stiamo dicendo che al crescere di $k$ il valore di $a_k$ non può che restare vicino (a piacere) a $q$,
dove stare vicino significa stare in ogni intorno piccolo a piacere di $q$.

\ms

Data una successione $a:N\arr \Q$ possiamo definire una {\it serie}, che denotiamo con $\sum_{k=0}^\infty a_k$, che corrisponde alla somma dei termini della successione $a_n$.
Ad esempio la serie 
\eq{
1+z+z^2+\dots+z^k+ \dots = \sum_{k=0}^\infty z^k
}
corrisponde alla serie associata alla successione $a:\N\arr \Q: k\mapsto z^k$.

Viceversa data una serie $\sum_{k=0}^\infty a_k$ possiamo associare ad essa la {\it successione delle ridotte} $R:\N\arr \Q: n\mapsto R_n= \sum_{k=0}^n a_k$.
Se la successione delle ridotte ha un limite $\lim_{n\arr \infty} R_n= q$ allora diciamo per definizione che la serie converge e poniamo $\sum_{k=0}^\infty a_k= q$.

Ovviamente non tutte le serie convergono a un valore finito. Ad esempio la serie $\sum_{k=0}^\infty a_k= 1+1+1+\dots$ non converge ad un numero razionale finito, cioè la successione delle ridotte cresce arbitrariamente (e diciamo che diverge).
La  serie $\sum_{k=0}^\infty a_k= 1-1+1-1+\dots+ (-1)^k+\dots$ non converge perché la successione delle ridotte resta finita ma non ammette limite (e si dice che la serie {\it oscilla} in questo caso tra 1 e 0).

Quindi abbiamo serie che divergono, serie che oscillano ma esistono pure serie (ad esempio $\sum_{k=0}^\infty z^k $) che convergono ad un numero razionale preciso (in questo caso $ \Frac[1/1-z]$ se $-1<z<1$ e $z\not=0$).

Per ora basta così. 
La situazione necessita di più attenzione però. Dobbiamo imparare quando le serie e le successioni convergono e possibilmente saper calcolare il loro limite $q$.
Ma per ora è chiaro che l'argomento di Zenone si fonda sulla convinzione (non dimostrata e qui contraddetta) che ogni serie diverga solo perché suona bene che la somma di infiniti numeri positivi cresce sempre quindi di sicuro arriva ad infinito.

Tra l'altro qui stiamo facendo serie e successioni in $\Q$ e in $\Q$ mostreremo che ci sono anche serie che rallentano abbastanza velocemente per non divergere ma lo stesso non convergono a un numero razionale. Questo definirà i numeri reali. Infatti si possono scrivere successioni di numeri razionali che convergerebbero a $\sqrt{2}$ che non essendo un numero razionale significa che oscillano (non esiste il limite in $\Q$).





\EndDocument
\end


