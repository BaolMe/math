% !TEX TS-program = latex
% !TEX spellcheck = it-IT
\documentclass[a4paper]{article}
\usepackage[utf8]{inputenc}
\usepackage[T1]{fontenc}
\usepackage[italian]{babel}

%%%%%%%%%%%%%%%%%%%%%%%%%%%%%%%%%%%%%%%%%%%%%%%%%%%%%%%%%%%%%%%%%%%%%%%%%%%%
%% Trim Size: 9.75in x 6.5in
%% Text Area: 8in (include Runningheads) x 5in
%% ws-ijgmmp.tex   :   2-9-08
%% Tex file to use with ws-ijgmmp.cls written in Latex2E.
%% The content, structure, format and layout of this style file is the
%% property of World Scientific Publishing Co. Pte. Ltd.
%% Copyright 1995, 2002 by World Scientific Publishing Co.
%% All rights are reserved.
%%%%%%%%%%%%%%%%%%%%%%%%%%%%%%%%%%%%%%%%%%%%%%%%%%%%%%%%%%%%%%%%%%%%%%%%%%%%
%%

%%%%%%%%%%%%%%%%%%%%%%%%%%%%%


\usepackage{pictexwd,dcpic}
\usepackage{csquotes}
\usepackage{nopageno}

\usepackage{amsmath, amssymb, amsthm}
\usepackage{pictex, dcpic}
\usepackage{color}
%\usepackage{graphicx}
%\numberwithin{equation}{section} % This line resets equation numbering when starting a new section.
%\renewcommand{\theequation}{Eq. \thesection.\arabic{equation}} % This line ads "Eq." in front of your equation numbering.




%
%    Macros.    Version 1.2.0.beta
%    The best use is to paste all of them into the papers
%     1/8/2005
%

%%%%%%%%%%%%%%%%%%%%%%%%%%%%%%
%%%%%%			Greek		 %%%%%%
%%%%%%%%%%%%%%%%%%%%%%%%%%%%%%

\def\al{\alpha}
\def\be{\beta}
\def\de{\delta}
\def\ga{\gamma}
\def\up{\upsilon}
\def\ep{\epsilon}
\def\io{\iota}
\def\te{\theta}
\def\la{\lambda}
\def\ze{\zeta}
\def\om{\omega}
\def\si{\sigma}
\def\vp{\varphi}
\def\vpi{\varpi}
\def\ka{\kappa}
\def\vs{\varsigma}
\def\vr{\varrho}

\def\De{\Delta}
\def\Ga{\Gamma}
\def\Te{\Theta}
\def\La{\Lambda}
\def\Om{\Omega}
\def\Si{\Sigma}
\def\Up{\Upsilon}

\def\boldDe{\bf\Delta}
\def\boldGa{\bf\Gamma}
\def\boldTe{\bf\Theta}
\def\boldLa{\bf\Lambda}
\def\boldOm{\bf\Omega}
\def\boldSi{\bf\Sigma}
\def\boldUp{\bf\Upsilon}
\def\boldXi{\bf\Xi}
\def\boldPi{\bf\Pi}
\def\boldPhi{\bf\Phi}
\def\boldPsi{\bf\Psi}

\def\boldal{\bf\al}
\def\boldbe{\bf\be}
\def\boldga{\bf\ga}
\def\boldde{\bf\de}
\def\boldep{\bf\ep}
\def\boldze{\bf\ze}
\def\boldeta{\bf\eta}
\def\boldte{\bf\te}
\def\boldio{\bf\io}
\def\boldka{\bf\ka}
\def\boldla{\bf\la}
\def\boldmu{\hbox{\greekbold\char "16}}
\def\boldnu{\hbox{\greekbold\char "17}}
\def\boldxi{\hbox{\greekbold\char "18}}
\def\boldpi{\hbox{\greekbold\char "19}}
\def\boldrho{\hbox{\greekbold\char "1A}}
\def\boldsi{\bf\si}
\def\boldtau{\hbox{\greekbold\char "1C}}
\def\boldup{\hbox{\greekbold\char "1D}}
\def\boldphi{\hbox{\greekbold\char "1E}}
\def\boldchi{\hbox{\greekbold\char "1F}}
\def\boldpsi{\hbox{\greekbold\char "20}}
\def\boldom{\hbox{\greekbold\char "21}}
\def\boldvep{\hbox{\greekbold\char "22}}
\def\boldvte{\hbox{\greekbold\char "23}}
\def\boldvpi{\hbox{\greekbold\char "24}}
\def\boldvrho{\hbox{\greekbold\char "25}}
\def\boldvsi{\hbox{\greekbold\char "26}}
\def\boldvphi{\hbox{\greekbold\char "27}}

%%%%%%%%%%%%%%%%%%%%%%%%%%%%%%
%%%%%%			Cal			 %%%%%%
%%%%%%%%%%%%%%%%%%%%%%%%%%%%%%
 \def\calA{{\hbox{\cal A}}}
 \def\calU{{\hbox{\cal U}}}
 \def\calB{{\hbox{\cal B}}}
 \def\calC{{\hbox{\cal C}}}
 \def\calI{{\hbox{\cal I}}}
 \def\calQ{{\hbox{\cal Q}}}
 \def\calP{{\hbox{\cal P}}}
 \def\calL{{\hbox{\cal L}}}
 \def\calE{{\hbox{\cal E}}}
 \def\calW{{\hbox{\cal W}}}
 \def\calS{{\hbox{\cal S}}}
 \def\calF{{\hbox{\cal F}}}
 \def\calR{{\hbox{\cal R}}}
 \def\calO{{\hbox{\cal O}}}
 \def\calM{{\hbox{\cal M}}}

%%%%%%%%%%%%%%%%%%%%%%%%%%%%%%
%%%%%%			gothic		 %%%%%%
%%%%%%%%%%%%%%%%%%%%%%%%%%%%%%
 \def\su{{\mathfrak{su}}}
 \def\gl{{\mathfrak{gl}}}
 \def\gotg{{\mathfrak{g}}}
 \def\goth{{\mathfrak{h}}}
 \def\gotm{{\mathfrak{m}}}
 \def\gotk{{\mathfrak{k}}}
 \def\spin{{\mathfrak{spin}}}
 \def\slC{{\mathfrak{sl}}}
 \def\O{{\mathfrak{O}}}
 \def\Set{{\mathfrak{Set}}}

%%%%%%%%%%%%%%%%%%%%%%%%%%%%%%
%%%%%%			Bbb			 %%%%%%
%%%%%%%%%%%%%%%%%%%%%%%%%%%%%%
 \def\one{\mathbb{I}}
 \def\A{\mathbb{A}}
 \def\B{\mathbb{B}}
 \def\C{\mathbb{C}}
 \def\D{\mathbb{D}}
 \def\E{\mathbb{E}}
 \def\F{\mathbb{F}}
 \def\G{\mathbb{G}}
 \def\H{\mathbb{H}}
 \def\J{\mathbb{J}}
 \def\K{\mathbb{K}}
 \def\I{\mathbb{I}}
 \def\L{\mathbb{L}}
 \def\M{\mathbb{M}}
 \def\N{\mathbb{N}}
 \def\O{\mathbb{O}}
 \def\P{\mathbb{P}}
 \def\Q{\mathbb{Q}}
 \def\R{\mathbb{R}}
 \def\S{\mathbb{S}}
 \def\T{\mathbb{T}}
 \def\U{\mathbb{U}}
 \def\V{\mathbb{V}}
 \def\X{\mathbb{X}}
 \def\Y{\mathbb{Y}}
 \def\W{\mathbb{W}}
 \def\Z{\mathbb{Z}}
 

%%%%%%%%%%%%%%%%%%%%%%%%%%%%%%
%%%%%%		MathRoman		 %%%%%%
%%%%%%%%%%%%%%%%%%%%%%%%%%%%%%
\def\Tr{{\hbox{Tr}}}
\def\Con{{\hbox{Con}}}
\def\Aut{{\hbox{Aut}}}
\def\Div{{\hbox{Div}}}
\def\ad{{\hbox{ad}}}
\def\Ad{{\hbox{Ad}}}
\def\Re{{\hbox{Re}}}
\def\Im{{\hbox{Im}}}
\def\Card{{\hbox{Card}}}
\def\Iso{{\hbox{Iso}}}
\def\Geo{{\hbox{Geo}}}
\def\Int{{\hbox{Int}}}
\def\Inv{{\hbox{Inv}}}
\def\Spin{{\hbox{Spin}}}
\def\SO{{\hbox{SO}}}
\def\SU{{\hbox{SU}}}
\def\SL{{\hbox{SL}}}
\def\GL{{\hbox{GL}}}
\def\det{{\hbox{det}}}
\def\Hom{{\hbox{Hom}}}
\def\End{{\hbox{End}}}
\def\Euc{{\hbox{Euc}}}
\def\Lor{{\hbox{Lor}}}
\def\Diff{{\hbox{Diff}}}
\def\di{{\hbox{d}}}
\def\id{{\hbox{\rm id}}}
\def\diag{{\hbox{diag}}}
\def\rank{{\hbox{rank}}}
\def\span{{\hbox{span}}}
\def\Sim{{\hbox{Sim}}}

\def\Obj{{\hbox{Obj}}}

%%%%%%%%%%%%%%%%%%%%%%%%%%%%%%
%%%%%%		OtherSymbols		 %%%%%%
%%%%%%%%%%%%%%%%%%%%%%%%%%%%%%
\def\ip{\hbox to4pt{\leaders\hrule height0.3pt\hfill}\vbox to8pt{\leaders\vrule width0.3pt\vfill}\kern 2pt}

% inner product
\def\QDE{\hfill\hbox{\ }\vrule height4pt width4pt depth0pt} 
\def\del{\partial}
\def\na{\nabla}
\def\inter{\cap}
\def\Vec{\mathfrak{X}}
\def\Lie{\hbox{\LieFont \$}}

\def\arr{\rightarrow}
\def\larr{\longrightarrow}
\def\harr{\hookrightarrow}
\def\hlarr{\lhook\joinrel\longrightarrow}
\def\then{\Rightarrow}
\def\semidirect{\hbox{\Bbb \char111}}
\def\binomial#1#2{\left(\Matrix{#1\cr #2\cr}\right)}

\def\barJ{\bar J}

\def\sSU{{\hbox{SU}}}
\def\so{{so}}
%\def\N{{{\mathbb N}}}

\def\file#1{{\tt #1}}
\def\calH{{{\cal H}}}
\def\calJ{{{\cal J}}}
\def\calG{{{\cal G}}}
\def\calK{{{\cal K}}}
\def\calD{{{\cal D}}}
\def\scalD{{{\cal D}}}

\def\Loop{{\hbox{Loop}}}
\def\Hoop{{\hbox{Hoop}}}
\def\Sym{{\hbox{Sym}}}
\def\sR{{\R}}
\def\dCor{dCor}
\def\epm{\hbox{$^\pm$}}
\def\dunion{\coprod}

\def\nab#1{{\buildrel #1\over \na}}
\def\frac[#1/#2]{\hbox{$#1\over#2$}}
\def\Frac[#1/#2]{{#1\over#2}}
\def\({\left(}
\def\){\right)}
\def\[{\left[}
\def\]{\right]}
\def\^#1{{}^{#1}_{\>\cdot}}
\def\_#1{{}_{#1}^{\>\cdot}}
\def\Label=#1{{\buildrel {\hbox{\fiveSerif \ShowLabel{#1}}}\over =}}
\def\<{\kern -1pt}
\def\Bar{\>|\>}
\def\Dal{\hbox{\tenRelazioni  \char003}}

\def\uvec#1{\vbox{\hbox{$\scriptstyle\rightharpoonup$}\vskip-9pt\hbox{$#1$}}}
\def\dvec#1{\vtop{\hbox{$#1$}\vskip-10pt\hbox{$\scriptstyle\rightharpoondown$}}}
\def\Cprod{\diamond}
\def\obullet{\odot}



%%%%%%%%%%%%			frames 				%%%%%%%%%%%%%%%%%%%

\def\red#1{{\color{red}{#1}}}
\def\blue#1{{\color{blue}{#1}}}
\def\green#1{{\color{green}{#1}}}

\def\Red{\color{red}}
\def\Blue{\color{blue}}
\def\Green{\color{green}}


\def\frame#1{\vbox{\hrule\hbox{\vrule\vbox{\kern2pt\hbox{\kern2pt#1\kern2pt}\kern2pt}\vrule}\hrule\kern-4pt}} 
\def\redframe#1{\red{\frame{#1}}} 
\def\greenframe#1{\green{\frame{#1}}} 
\def\blueframe#1{\blue{\frame{#1}}} 

\def\uline#1{\underline{#1}}
\def\uuline#1{\underline{\underline{#1}}}
\def\Box to #1#2#3{\frame{\vtop{\hbox to #1{\hfill #2 \hfill}\hbox to #1{\hfill #3 \hfill}}}}



\def\ubal{\underline{\al}\kern1pt}
\def\obal{\overline{\al}\kern1pt}

\def\ubR{\underline{R}\kern1pt}
\def\obR{\overline{R}\kern1pt}
\def\ubom{\underline{\om}\kern1pt}
\def\obxi{\overline{\xi}\kern1pt}
\def\ubu{\underline{u}\kern1pt}
\def\ube{\underline{e}\kern1pt}
\def\obe{\overline{e}\kern1pt}
\def\Limit{\>{\buildrel{r\arr\infty}\over \longrightarrow}\,}
\def\union{\cup}
\def\Emptyset{\varnothing}




\def\Uvec#1{\vbox{\mathsurround=0pt\ialign{##\crcr
     $\scriptscriptstyle\rightharpoonup$\crcr\noalign{\kern1pt\nointerlineskip}
     $\hfil\displaystyle{#1}\hfil$\crcr}}}
\def\Dvec#1{\vbox{\mathsurround=0pt\ialign{##\crcr
     $\scriptscriptstyle\rightharpoondown$\crcr\noalign{\kern-7pt\nointerlineskip}
     $\hfil\displaystyle{#1}\hfil$\crcr}}}


%   u 
\def\uvecu{\vbox{\mathsurround=0pt\ialign{##\crcr
     $\scriptscriptstyle\rightharpoonup$\crcr\noalign{\kern1pt\nointerlineskip}
     $\hfil\displaystyle{u}\hfil$\crcr}}}
\def\dvecu{\vbox{\mathsurround=0pt\ialign{##\crcr
     $\scriptscriptstyle\rightharpoondown$\crcr\noalign{\kern-7pt\nointerlineskip}
     $\hfil\displaystyle{u}\hfil$\crcr}}}
%\def\uvecu{\Uvec{u}}
%\def\dvecu{\Dvec{u}}

%   be
\def\uvecbe{\vbox{\mathsurround=0pt\ialign{##\crcr
     \kern3pt$\scriptscriptstyle\rightharpoonup$\crcr\noalign{\kern1pt\nointerlineskip}
     $\hfil\displaystyle{\be}\hfil$\crcr}}}
\def\dvecbe{\vbox{\mathsurround=0pt\ialign{##\crcr
     \kern1pt$\scriptscriptstyle\rightharpoondown$\crcr\noalign{\kern-10pt\nointerlineskip}
     $\hfil\displaystyle{\be}\hfil$\crcr}}}

%   n
\def\uvecn{\vbox{\mathsurround=0pt\ialign{##\crcr
     $\scriptscriptstyle\rightharpoonup$\crcr\noalign{\kern1pt\nointerlineskip}
     $\hfil\displaystyle{n}\hfil$\crcr}}}
\def\dvecn{\vbox{\mathsurround=0pt\ialign{##\crcr
     $\scriptscriptstyle\rightharpoondown$\crcr\noalign{\kern-7pt\nointerlineskip}
     $\hfil\displaystyle{n}\hfil$\crcr}}}

%   m
\def\uvecm{\vbox{\mathsurround=0pt\ialign{##\crcr
     $\scriptscriptstyle\rightharpoonup$\crcr\noalign{\kern1pt\nointerlineskip}
     $\hfil\displaystyle{m}\hfil$\crcr}}}
\def\dvecm{\vbox{\mathsurround=0pt\ialign{##\crcr
     $\scriptscriptstyle\rightharpoondown$\crcr\noalign{\kern-7pt\nointerlineskip}
     $\hfil\displaystyle{m}\hfil$\crcr}}}

%   N
\def\uvecN{\vbox{\mathsurround=0pt\ialign{##\crcr
     \kern3pt$\scriptscriptstyle\rightharpoonup$\crcr\noalign{\kern1pt\nointerlineskip}
     $\hfil\displaystyle{N}\hfil$\crcr}}}
\def\dvecN{\vbox{\mathsurround=0pt\ialign{##\crcr
     \kern0pt$\scriptscriptstyle\rightharpoondown$\crcr\noalign{\kern-10pt\nointerlineskip}
     $\hfil\displaystyle{N}\hfil$\crcr}}}

%   u
\def\uvecu{\vbox{\mathsurround=0pt\ialign{##\crcr
     $\scriptscriptstyle\rightharpoonup$\crcr\noalign{\kern1pt\nointerlineskip}
     $\hfil\displaystyle{u}\hfil$\crcr}}}
\def\dvecu{\vbox{\mathsurround=0pt\ialign{##\crcr
     $\scriptscriptstyle\rightharpoondown$\crcr\noalign{\kern-7pt\nointerlineskip}
     $\hfil\displaystyle{u}\hfil$\crcr}}}

%   w
\def\uvecw{\vbox{\mathsurround=0pt\ialign{##\crcr
     $\scriptscriptstyle\rightharpoonup$\crcr\noalign{\kern1pt\nointerlineskip}
     $\hfil\displaystyle{w}\hfil$\crcr}}}
\def\dvecw{\vbox{\mathsurround=0pt\ialign{##\crcr
     $\scriptscriptstyle\rightharpoondown$\crcr\noalign{\kern-7pt\nointerlineskip}
     $\hfil\displaystyle{w}\hfil$\crcr}}}

%   v
\def\uvecv{\vbox{\mathsurround=0pt\ialign{##\crcr
     $\scriptscriptstyle\rightharpoonup$\crcr\noalign{\kern1pt\nointerlineskip}
     $\hfil\displaystyle{v}\hfil$\crcr}}}
\def\dvecv{\vbox{\mathsurround=0pt\ialign{##\crcr
     $\scriptscriptstyle\rightharpoondown$\crcr\noalign{\kern-7pt\nointerlineskip}
     $\hfil\displaystyle{v}\hfil$\crcr}}}

\def\astA{{}^\ast A}
\def\circA{{}^\circ A}
\def\astk{{}^\ast k}
\def\circk{{}^\circ k}
\def\astK{{}^\ast K}
\def\circK{{}^\circ K}
\def\astL{{}^\ast L}
\def\circL{{}^\circ L}
\def\astal{{}^\ast \al}
\def\circal{{}^\circ \al}
\def\astsi{{}^\ast \si}
\def\circsi{{}^\circ \si}
\def\aste{{}^\ast e}
\def\circe{{}^\circ e}
\def\astte{{}^\ast \te}
\def\circte{{}^\circ \te}
\def\astGa{{}^\ast \Ga}
\def\circGa{{}^\circ \Ga}
\def\Lie{\pounds}

\long\def\Hide#1{}
\long\def\HideMarked#1{\hfill{$\triangleright$}}
\def\rtau{\tau}


%%%%%%%%%%%%%%%%%%%%%%%%%%%%%%%%%%%%%%%%%%%%%%%%%%%%%%%%%%%%%%%
%\def\NormalStyle{\leftskip=2cm\rightskip=0cm\normalsize\parindent=5pt\parskip=3pt\normalbaselineskip=14pt\baselineskip=\normalbaselineskip}
%\def\AbstractStyle{\leftskip=3cm\rightskip=1cm\scriptsize\parindent=0pt\parskip=0pt\normalbaselineskip=11pt\baselineskip=\normalbaselineskip}
%\def\NoteStyle{\leftskip=3cm\rightskip=1cm\scriptsize\parindent=0pt\parskip=0pt\normalbaselineskip=11pt\baselineskip=\normalbaselineskip}

\def\ShowLabel#1{\ref{#1}}

\def\Bibliography{\begin{thebibliography}{199}\footnotesize\References}
\def\EndBibliography{\end{thebibliography}}
\def\bib#1#2{\bibitem{#1}#2}
\def\Ref#1{\cite{#1}}


\def\NewSection#1{\section{#1}}
\def\NewSubSection#1{\subsection*{#1}}
\def\NewAppendix#1#2{\section*{Appendix #1: #2}}
\def\Acknowledgements{\section*{Acknowledgements}}
\def\bs{\bigskip}
\def\ms{\medskip}
\def\ss{\smallskip}
\def\ni{\noindent}

\long\def\Note#1{\blockquote{\footnotesize {\bf Nota:}~#1\par}}
\long\def\Ex#1{\blockquote{\footnotesize {\bf Esercizio:}~#1\par}}

\def\eq#1{\begin{equation}#1\end{equation}}
\def\eqLabel#1#2{\begin{equation}#1\label{#2}\end{equation}}


\def\Cases#1{\begin{cases}#1\end{cases}}
\def\Matrix#1{\begin{matrix}#1\end{matrix}}
\def\Align#1{\begin{aligned}#1\end{aligned}}

\def\eqs#1{\eq{\Align{#1}}}
\def\eqsLabel#1#2{\eq{\Align{#1}\label{#2}}}




\def\Abstract{\AbstractStyle{\bf Abstract. }}
\def\EndAbstract{\par\NormalStyle}

\def\TitleScript{}
\def\TitleLine#1{\def\TitleScript{#1}}
\def\MoreTitleLine#1{\edef\TitleScript{\TitleScript\\#1}}

\def\AuthorScript{}
\def\AuthorLine#1{\def\AuthorScript{#1}}
\def\MoreAuthorLine#1{\edef\AuthorScript{\AuthorScript\\#1}}

\def\AddressScript{}
\def\AddressLine#1{\def\AddressScript{#1}}
\def\MoreAddressLine#1{\edef\AddressScript{\AddressScript\\#1}}

\date{}

\def\BeginDocument{\begin{document}}

\def\MakeTitle{
\begin{center}
{\Large\bf\sffamily\baselineskip=2pt\TitleScript}\\
\vskip 10pt
{\small by {\it \AuthorScript}}\\
\vskip 10pt
{\small \AddressScript}
\end{center}
\ms
}

\def\EndDocument{\end{document}}

\def\Figure[#1]#2{\begin{figure}[htbp] %  figure placement: here, top, bottom, or page
   \centering
   \includegraphics[#1]{#2} }
\def\Caption#1{\caption{#1}}
\def\EndFigure{\end{figure}}

\def\Diagram#1{\eq{
\begindc{\commdiag}[10]
#1
\enddc%
}%
}

\def\Itemize#1{\begin{itemize}#1\end{itemize}}
\def\Item[#1]{\item[#1]}

\def\AllReferences{}



\def\Shrink{\small}
%%%%%%%%%%%%%%%%%%%%%%%%%%%%%%%%%%%%%%%%%%%%%%%%%
%%%%%%%%%%%%%%%%%%%%%%%%%%%%%%%%%%%%%%%%%%%%%%%%%%%

%\BeginDocument


%%%%%%%%%%%%%%%%%%%%%%%%%%%%%%%%%%%%%%%%%%%%%%%%%%%






















\parindent=5pt
\baselineskip=13pt
\parskip=5pt





\begin{document}

\AllReferences


%%%%%%%%%%%%%%%%%%%%% Publisher's Area please ignore %%%%%%%%%%%%%%%
%
%
%%%%%%%%%%%%%%%%%%%%%%%%%%%%%%%%%%%%%%%%%%%%%%%%%%%%%%%%%%%%%%%%%%%%

\section*{Giorno 12: prodotto}

Pure il prodotto è definito usando gli assiomi di Peano. 
Ma quando pensiamo ai numeri come insiemi possiamo definire il prodotto come la cardinalità del prodotto cartesiamo.
$$
2*3 = \Card(\{0,1\}\times \{0,1,2\})
=\Card(\{(0,0), (0,1), (0,2), (1,0), (1,1), (1,2)\})=6
$$
Anche se $\{0,1\}\times \{0,1,2\}$ e $\{0,1,2\}\times \{0,1\}$ sono insiemi diversi, la loro cardinalità è la stessa (sotto la mappa biettiva $(a, b)\mapsto (b, a)$).
Quindi il prodotto è commutativo

Stessa cosa per $A\times (B\times C)$ e $(A,\times B)\times C$, il primo contiene roba tipo $(a, (b, c))$ il secondo roba tipo $((a, b), c)$.
Ma possiamo definire una mappa biettiva $(a, (b, c))\mapsto ((a, b), c)$ che dimostra che $\Card(A\times (B\times C)) =\Card((A,\times B)\times C)$
e quindi la proprietà associativa del prodotto $(ab)c= a(bc)$.

Poi se prendiamo $1=\{0\}$, quando facciamo 
$$1*3=\Card(\{0\}\times \{0,1,2\})=\Card( \{(0,0),(0,1),(0,2)\})
$$
Di nuovo, $\{0\}\times A$ e $A$ sono insiemi diversi ma hanno la stessa cardinalità, quindi $1*n=n*1=n$.
Quindi $1$ è elemento neutro per il prodotto.

\Ex{dimostrate che $0*n=0$, per qualunque $n\in \N$.

Suggerimento: scrivete il prodotto cartesiano $\Emptyset\times A$}


In più avete la proprietà distributiva $(a+b)c= ac+bc$.

\Note{
che si dimostra trovando una mappa biettiva tra $(A\dunion B)\times C$ e $(A\times C)\dunion (B\times C)$...}

Fine della storia sulle operazioni definite in $\N$. O quasi.
Il punto è che da qui in avanti la rappresentazione di numeri e operazioni in termini di insiemi non ci serve più possiamo astrarre e procedere solo usando le proprietà delle operazioni.

Ad esempio, dalla proprietà distributiva, mettiamo $a=1=b$ e otteniamo $2n=n+n$ che è la definizione che si dà alle elementari. 
Mettiamo $a=1$ e $b=2$  e otteniamo $3n=n+n+n$ e avanti così.

Un altro giochino  carino è il seguente teorema: l'elemento neutro 0 della somma è unico.

\Note{
Supponiamo che esista in altro elemento neutro $a$.
Se scriviamo $a+0$, siccome 0 `è elemento neutro abbiamo $a+0=a$.
Ma siccome pure $a$ è elemento neutro abbiamo anche $a+0=0$.
Quindi abbiamo $a=a+0=0$, cioè $a=0$, cioè ogni ``altro'' elemento neutro coincide con quello vecchio.
}

Potremmo pure dare una dimostrazione basata sugli insiemi dello stesso fatto, ma questa dimostrazione è più intelligente. Perché?

Perché, siccome usa solo la proprietà dell'elemento neutro $a+0=0+a=a$ e questa proprietà è condivisa da tutti gli elementi neutri (1 del prodotto, la funzione $f(x)=0$ rispetto alla somma delle funzioni $f:A\arr \R$) la stessa dimostrazione dimostra che 1 è l'unico elemento neutro del prodotto, e l'unicità dell'elemento neutro del gruppo delle rotazioni, dello 0 in ogni spazio vettoriale,
di $0+i0\in \C$.



\NewSubSection{Algoritmo per il prodotto}

Alle elementari abbiamo fatto pure l'algoritmo per eseguire il prodotto in $\N$. 
Se dobbiamo fare $1238*326$ in pratica usiamo questo trucco (di nuovo basato sulla notazione posizionale)
$$
1238*326= 1238(3*100+2*10+6)
=1238*3*100+1238*2*10+1238*6
$$
Quindi se sappiamo moltiplicare un numero naturale per una cifra (e sappiamo che moltiplicare per 100 significa aggiungere 00 alla coda si un numero) sappiamo calcolarlo.

Per la moltiplicazione di  un numero naturale per una cifra usiamo lo stesso trucco
$$
\Align{1238*3=& 1*3*1000+ 2*3*100 + 3*3*10+8*3
=\cr
=& 3*1000+ 6*100 + 9*10+24= 3690+24= 3714
}
$$

In sostanza l'algoritmo che utilizziamo alle elementari è solo per minimizzare le cose che bisogna ricordare per tanto tempo durante l'esecuzione (vi dice dove scrivere i riporti mentre calcolate le somme parziali). A parte questo state facendo quello che fate qui sopra.

Di nuovo, può essere laborioso ma potete moltiplicare qualunque coppia di numeri naturali, basta che sappiate moltiplicare le cifre (tabelline) e fare le somme.




\NewSubSection{Divisione e resto}

Come per la sottrazione la divisione in $\N$ ha un ruolo ancillare.
Due numeri naturali $n, k\in \N$ si possono dividere solo se $n$ è multiplo di $k$, cioè se esiste in $q\in\N$ tale che $n=qk$ e allora $k$ divide $n$ (abbiamo convenuto di scrivere $k|n$)
e scriviamo $n:k=q$ (con resto $r=0$).

In pratica se scriviamo che $n:k=q$ stiamo solo dicendo che $q\in \N$ è quel numero che moltiplicato per $k$ dà $n$, cioè che $q*k=n$. 
In altre parole la divisione esatta è inversa al prodotto.

Possiamo essere più liberali e mostrare che per ogni coppia di numeri naturali $n, k\in \N$, esistono (e sono unicamente determinati) 2 numeri $q\in\N$ e $r\in \{0, 1, \dots, k-1\}$ tale che
$n=qk+r$. 
Lo chiamiamo il {\it teorema dello Zecchino d'oro} ({\it 44 gatti in fila per 6 col resto di 2}) e $q$ lo chiamiamo il quoziente di $n:k$, $r$ lo chiamiamo il resto di $n:k$ che indichiamo anche con 
$r=n \mod k$ (che leggiamo $n$ modulo $k$).

\Note{L'operazione che associa a $(n, k)$ i numeri $(q, r)$ è ben definita. Se chiedete a me i microprocessori dovrebbero implementare questa, non i floating points.}


\NewSubSection{Algoritmo di divisione}
Se ho 2 numeri $n$ e $k$ e voglio dividere $n:k$ comincio a cercare il più grande $q_1$ tale che $q_1*k\le n$.
Quindi definisco $n_1= n-q_1*k$. Se $n_1\in \{0, 1, \dots, k-1\}$ abbiamo finito se no
cerchiamo il più grande $q_2$ tale che $q_2*k\le n_1$. Quindi definiamo $n_2=n_1-q_2*k$ e avanti così finché $r=n_{l}= n_{l-1}-q_l*k\in \{0, 1, \dots, k-1\}$ che è quindi il resto.
Abbiamo che 
$$
\Align{
n=& q_1*k+n_1
=  q_1*k+q_2*k + n_2
=\dots =\cr
=& q_1*k+q_2*k +\dots + q_l*k+r
= (q_1+q_2 +\dots + q_l)*k+r
= q*k+r
}
$$
quindi il quoziente è $q=q_1+q_2 +\dots + q_l$ e il resto è $r$.

Come per la sottrazione l'algoritmo delle elementari è solo un modo per organizzare il calcolo senza dover ricordare (o capire) troppe cose.

\NewSubSection{Divisibilità e numeri primi}

Ora che sappiamo cosa significa dividere esattamente $k|n$ ($k$ divide $n$) possiamo notare che certi numeri $2, 3, 5, 7, \dots$ hanno esattamente 2 divisori ($1$ e loro stessi).
Li chiamiamo {\it numeri primi}.
Un'altra definizione equivalente è che se un numero primo $p$ divide un prodotto $ab$, siccome non potete spezzare $p$ in un prodotto, allora $p$ o divide $a$ o divide $b$.
Quindi possiamo dire che $p\not =1$ è primo se e solo se
$$
p|ab \then p|a \hbox{ .or. } p|b
$$
Notate che 1 non è un primo perché non ha esattamente 2 divisori ne ha 1 solo.

Fate una bella cosa, scrivere i numeri da 2 a 10 su un foglio e poi disegnare una freccia tra $a$ e $b$ se $a|b$.
Ottenete dei grafi disconnessi, uno per ogni numero primo tra 2 a 10.

Ogni numero naturale $n$ si può scrivere in maniera unica come prodotto di numeri primi (per avere l'unicità della decomposizione abbiamo escluso 1 dai primi).

I numeri primi sono infiniti.

\Note{
Supponiamo che i numeri primi siano in numero finito cioè $\{2, 3, 5, ...., p_n\}$.
Quindi $p_n$ dovrebbe essere il primo più grande di tutti.
Sfortunatamente $p= 2*3*5*\dots* p_n+1$ è primo ($p \mod p_k =1$, quindi $p_k$ non divide $p$) ed è più grande di $p_k$ che quindi non è il primo più grande di tutti.
}

\EndDocument
\end


