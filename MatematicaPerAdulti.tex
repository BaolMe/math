% !TEX TS-program = latex
% !TEX spellcheck = it-IT
\documentclass[a4paper]{article}
\usepackage[utf8]{inputenc}
\usepackage[T1]{fontenc}
\usepackage[italian]{babel}

%%%%%%%%%%%%%%%%%%%%%%%%%%%%%%%%%%%%%%%%%%%%%%%%%%%%%%%%%%%%%%%%%%%%%%%%%%%%
%% Trim Size: 9.75in x 6.5in
%% Text Area: 8in (include Runningheads) x 5in
%% ws-ijgmmp.tex   :   2-9-08
%% Tex file to use with ws-ijgmmp.cls written in Latex2E.
%% The content, structure, format and layout of this style file is the
%% property of World Scientific Publishing Co. Pte. Ltd.
%% Copyright 1995, 2002 by World Scientific Publishing Co.
%% All rights are reserved.
%%%%%%%%%%%%%%%%%%%%%%%%%%%%%%%%%%%%%%%%%%%%%%%%%%%%%%%%%%%%%%%%%%%%%%%%%%%%
%%

%%%%%%%%%%%%%%%%%%%%%%%%%%%%%


\usepackage{pictexwd,dcpic}
\usepackage{csquotes}
\usepackage{nopageno}

\usepackage{amsmath, amssymb, amsthm}
\usepackage{pictex, dcpic}
\usepackage{color}
%\usepackage{graphicx}
%\numberwithin{equation}{section} % This line resets equation numbering when starting a new section.
%\renewcommand{\theequation}{Eq. \thesection.\arabic{equation}} % This line ads "Eq." in front of your equation numbering.




%
%    Macros.    Version 1.2.0.beta
%    The best use is to paste all of them into the papers
%     1/8/2005
%

%%%%%%%%%%%%%%%%%%%%%%%%%%%%%%
%%%%%%			Greek		 %%%%%%
%%%%%%%%%%%%%%%%%%%%%%%%%%%%%%

\def\al{\alpha}
\def\be{\beta}
\def\de{\delta}
\def\ga{\gamma}
\def\up{\upsilon}
\def\ep{\epsilon}
\def\io{\iota}
\def\te{\theta}
\def\la{\lambda}
\def\ze{\zeta}
\def\om{\omega}
\def\si{\sigma}
\def\vp{\varphi}
\def\vpi{\varpi}
\def\ka{\kappa}
\def\vs{\varsigma}
\def\vr{\varrho}

\def\De{\Delta}
\def\Ga{\Gamma}
\def\Te{\Theta}
\def\La{\Lambda}
\def\Om{\Omega}
\def\Si{\Sigma}
\def\Up{\Upsilon}

\def\boldDe{\bf\Delta}
\def\boldGa{\bf\Gamma}
\def\boldTe{\bf\Theta}
\def\boldLa{\bf\Lambda}
\def\boldOm{\bf\Omega}
\def\boldSi{\bf\Sigma}
\def\boldUp{\bf\Upsilon}
\def\boldXi{\bf\Xi}
\def\boldPi{\bf\Pi}
\def\boldPhi{\bf\Phi}
\def\boldPsi{\bf\Psi}

\def\boldal{\bf\al}
\def\boldbe{\bf\be}
\def\boldga{\bf\ga}
\def\boldde{\bf\de}
\def\boldep{\bf\ep}
\def\boldze{\bf\ze}
\def\boldeta{\bf\eta}
\def\boldte{\bf\te}
\def\boldio{\bf\io}
\def\boldka{\bf\ka}
\def\boldla{\bf\la}
\def\boldmu{\hbox{\greekbold\char "16}}
\def\boldnu{\hbox{\greekbold\char "17}}
\def\boldxi{\hbox{\greekbold\char "18}}
\def\boldpi{\hbox{\greekbold\char "19}}
\def\boldrho{\hbox{\greekbold\char "1A}}
\def\boldsi{\bf\si}
\def\boldtau{\hbox{\greekbold\char "1C}}
\def\boldup{\hbox{\greekbold\char "1D}}
\def\boldphi{\hbox{\greekbold\char "1E}}
\def\boldchi{\hbox{\greekbold\char "1F}}
\def\boldpsi{\hbox{\greekbold\char "20}}
\def\boldom{\hbox{\greekbold\char "21}}
\def\boldvep{\hbox{\greekbold\char "22}}
\def\boldvte{\hbox{\greekbold\char "23}}
\def\boldvpi{\hbox{\greekbold\char "24}}
\def\boldvrho{\hbox{\greekbold\char "25}}
\def\boldvsi{\hbox{\greekbold\char "26}}
\def\boldvphi{\hbox{\greekbold\char "27}}

%%%%%%%%%%%%%%%%%%%%%%%%%%%%%%
%%%%%%			Cal			 %%%%%%
%%%%%%%%%%%%%%%%%%%%%%%%%%%%%%
 \def\calA{{\hbox{\cal A}}}
 \def\calU{{\hbox{\cal U}}}
 \def\calB{{\hbox{\cal B}}}
 \def\calC{{\hbox{\cal C}}}
 \def\calI{{\hbox{\cal I}}}
 \def\calQ{{\hbox{\cal Q}}}
 \def\calP{{\hbox{\cal P}}}
 \def\calL{{\hbox{\cal L}}}
 \def\calE{{\hbox{\cal E}}}
 \def\calW{{\hbox{\cal W}}}
 \def\calS{{\hbox{\cal S}}}
 \def\calF{{\hbox{\cal F}}}
 \def\calR{{\hbox{\cal R}}}
 \def\calO{{\hbox{\cal O}}}
 \def\calM{{\hbox{\cal M}}}

%%%%%%%%%%%%%%%%%%%%%%%%%%%%%%
%%%%%%			gothic		 %%%%%%
%%%%%%%%%%%%%%%%%%%%%%%%%%%%%%
 \def\su{{\mathfrak{su}}}
 \def\gl{{\mathfrak{gl}}}
 \def\gotg{{\mathfrak{g}}}
 \def\goth{{\mathfrak{h}}}
 \def\gotm{{\mathfrak{m}}}
 \def\gotk{{\mathfrak{k}}}
 \def\spin{{\mathfrak{spin}}}
 \def\slC{{\mathfrak{sl}}}
 \def\O{{\mathfrak{O}}}
 \def\Set{{\mathfrak{Set}}}

%%%%%%%%%%%%%%%%%%%%%%%%%%%%%%
%%%%%%			Bbb			 %%%%%%
%%%%%%%%%%%%%%%%%%%%%%%%%%%%%%
 \def\one{\mathbb{I}}
 \def\A{\mathbb{A}}
 \def\B{\mathbb{B}}
 \def\C{\mathbb{C}}
 \def\D{\mathbb{D}}
 \def\E{\mathbb{E}}
 \def\F{\mathbb{F}}
 \def\G{\mathbb{G}}
 \def\H{\mathbb{H}}
 \def\J{\mathbb{J}}
 \def\K{\mathbb{K}}
 \def\I{\mathbb{I}}
 \def\L{\mathbb{L}}
 \def\M{\mathbb{M}}
 \def\N{\mathbb{N}}
 \def\O{\mathbb{O}}
 \def\P{\mathbb{P}}
 \def\Q{\mathbb{Q}}
 \def\R{\mathbb{R}}
 \def\S{\mathbb{S}}
 \def\T{\mathbb{T}}
 \def\U{\mathbb{U}}
 \def\V{\mathbb{V}}
 \def\X{\mathbb{X}}
 \def\Y{\mathbb{Y}}
 \def\W{\mathbb{W}}
 \def\Z{\mathbb{Z}}
 

%%%%%%%%%%%%%%%%%%%%%%%%%%%%%%
%%%%%%		MathRoman		 %%%%%%
%%%%%%%%%%%%%%%%%%%%%%%%%%%%%%
\def\Tr{{\hbox{Tr}}}
\def\Con{{\hbox{Con}}}
\def\Aut{{\hbox{Aut}}}
\def\Div{{\hbox{Div}}}
\def\ad{{\hbox{ad}}}
\def\Ad{{\hbox{Ad}}}
\def\Re{{\hbox{Re}}}
\def\Im{{\hbox{Im}}}
\def\Iso{{\hbox{Iso}}}
\def\Geo{{\hbox{Geo}}}
\def\Int{{\hbox{Int}}}
\def\Inv{{\hbox{Inv}}}
\def\Spin{{\hbox{Spin}}}
\def\SO{{\hbox{SO}}}
\def\SU{{\hbox{SU}}}
\def\SL{{\hbox{SL}}}
\def\GL{{\hbox{GL}}}
\def\det{{\hbox{det}}}
\def\Hom{{\hbox{Hom}}}
\def\End{{\hbox{End}}}
\def\Euc{{\hbox{Euc}}}
\def\Lor{{\hbox{Lor}}}
\def\Diff{{\hbox{Diff}}}
\def\di{{\hbox{d}}}
\def\id{{\hbox{\rm id}}}
\def\diag{{\hbox{diag}}}
\def\rank{{\hbox{rank}}}
\def\span{{\hbox{span}}}
\def\Sim{{\hbox{Sim}}}

\def\Obj{{\hbox{Obj}}}

%%%%%%%%%%%%%%%%%%%%%%%%%%%%%%
%%%%%%		OtherSymbols		 %%%%%%
%%%%%%%%%%%%%%%%%%%%%%%%%%%%%%
\def\ip{\hbox to4pt{\leaders\hrule height0.3pt\hfill}\vbox to8pt{\leaders\vrule width0.3pt\vfill}\kern 2pt}

% inner product
\def\QDE{\hfill\hbox{\ }\vrule height4pt width4pt depth0pt} 
\def\del{\partial}
\def\na{\nabla}
\def\inter{\cap}
\def\Vec{\mathfrak{X}}
\def\Lie{\hbox{\LieFont \$}}

\def\arr{\rightarrow}
\def\larr{\longrightarrow}
\def\harr{\hookrightarrow}
\def\hlarr{\lhook\joinrel\longrightarrow}
\def\then{\Rightarrow}
\def\semidirect{\hbox{\Bbb \char111}}
\def\binomial#1#2{\left(\Matrix{#1\cr #2\cr}\right)}

\def\barJ{\bar J}

\def\sSU{{\hbox{SU}}}
\def\so{{so}}
%\def\N{{{\mathbb N}}}

\def\file#1{{\tt #1}}
\def\calH{{{\cal H}}}
\def\calJ{{{\cal J}}}
\def\calG{{{\cal G}}}
\def\calK{{{\cal K}}}
\def\calD{{{\cal D}}}
\def\scalD{{{\cal D}}}

\def\Loop{{\hbox{Loop}}}
\def\Hoop{{\hbox{Hoop}}}
\def\Sym{{\hbox{Sym}}}
\def\sR{{\R}}
\def\dCor{dCor}
\def\epm{\hbox{$^\pm$}}
\def\dunion{\coprod}

\def\nab#1{{\buildrel #1\over \na}}
\def\frac[#1/#2]{\hbox{$#1\over#2$}}
\def\Frac[#1/#2]{{#1\over#2}}
\def\({\left(}
\def\){\right)}
\def\[{\left[}
\def\]{\right]}
\def\^#1{{}^{#1}_{\>\cdot}}
\def\_#1{{}_{#1}^{\>\cdot}}
\def\Label=#1{{\buildrel {\hbox{\fiveSerif \ShowLabel{#1}}}\over =}}
\def\<{\kern -1pt}
\def\Bar{\>|\>}
\def\Dal{\hbox{\tenRelazioni  \char003}}

\def\uvec#1{\vbox{\hbox{$\scriptstyle\rightharpoonup$}\vskip-9pt\hbox{$#1$}}}
\def\dvec#1{\vtop{\hbox{$#1$}\vskip-10pt\hbox{$\scriptstyle\rightharpoondown$}}}
\def\Cprod{\diamond}
\def\obullet{\odot}



%%%%%%%%%%%%			frames 				%%%%%%%%%%%%%%%%%%%

\def\red#1{{\color{red}{#1}}}
\def\blue#1{{\color{blue}{#1}}}
\def\green#1{{\color{green}{#1}}}

\def\Red{\color{red}}
\def\Blue{\color{blue}}
\def\Green{\color{green}}


\def\frame#1{\vbox{\hrule\hbox{\vrule\vbox{\kern2pt\hbox{\kern2pt#1\kern2pt}\kern2pt}\vrule}\hrule\kern-4pt}} 
\def\redframe#1{\red{\frame{#1}}} 
\def\greenframe#1{\green{\frame{#1}}} 
\def\blueframe#1{\blue{\frame{#1}}} 

\def\uline#1{\underline{#1}}
\def\uuline#1{\underline{\underline{#1}}}
\def\Box to #1#2#3{\frame{\vtop{\hbox to #1{\hfill #2 \hfill}\hbox to #1{\hfill #3 \hfill}}}}



\def\ubal{\underline{\al}\kern1pt}
\def\obal{\overline{\al}\kern1pt}

\def\ubR{\underline{R}\kern1pt}
\def\obR{\overline{R}\kern1pt}
\def\ubom{\underline{\om}\kern1pt}
\def\obxi{\overline{\xi}\kern1pt}
\def\ubu{\underline{u}\kern1pt}
\def\ube{\underline{e}\kern1pt}
\def\obe{\overline{e}\kern1pt}
\def\Limit{\>{\buildrel{r\arr\infty}\over \longrightarrow}\,}
\def\union{\cup}
\def\Emptyset{\varnothing}




\def\Uvec#1{\vbox{\mathsurround=0pt\ialign{##\crcr
     $\scriptscriptstyle\rightharpoonup$\crcr\noalign{\kern1pt\nointerlineskip}
     $\hfil\displaystyle{#1}\hfil$\crcr}}}
\def\Dvec#1{\vbox{\mathsurround=0pt\ialign{##\crcr
     $\scriptscriptstyle\rightharpoondown$\crcr\noalign{\kern-7pt\nointerlineskip}
     $\hfil\displaystyle{#1}\hfil$\crcr}}}


%   u 
\def\uvecu{\vbox{\mathsurround=0pt\ialign{##\crcr
     $\scriptscriptstyle\rightharpoonup$\crcr\noalign{\kern1pt\nointerlineskip}
     $\hfil\displaystyle{u}\hfil$\crcr}}}
\def\dvecu{\vbox{\mathsurround=0pt\ialign{##\crcr
     $\scriptscriptstyle\rightharpoondown$\crcr\noalign{\kern-7pt\nointerlineskip}
     $\hfil\displaystyle{u}\hfil$\crcr}}}
%\def\uvecu{\Uvec{u}}
%\def\dvecu{\Dvec{u}}

%   be
\def\uvecbe{\vbox{\mathsurround=0pt\ialign{##\crcr
     \kern3pt$\scriptscriptstyle\rightharpoonup$\crcr\noalign{\kern1pt\nointerlineskip}
     $\hfil\displaystyle{\be}\hfil$\crcr}}}
\def\dvecbe{\vbox{\mathsurround=0pt\ialign{##\crcr
     \kern1pt$\scriptscriptstyle\rightharpoondown$\crcr\noalign{\kern-10pt\nointerlineskip}
     $\hfil\displaystyle{\be}\hfil$\crcr}}}

%   n
\def\uvecn{\vbox{\mathsurround=0pt\ialign{##\crcr
     $\scriptscriptstyle\rightharpoonup$\crcr\noalign{\kern1pt\nointerlineskip}
     $\hfil\displaystyle{n}\hfil$\crcr}}}
\def\dvecn{\vbox{\mathsurround=0pt\ialign{##\crcr
     $\scriptscriptstyle\rightharpoondown$\crcr\noalign{\kern-7pt\nointerlineskip}
     $\hfil\displaystyle{n}\hfil$\crcr}}}

%   m
\def\uvecm{\vbox{\mathsurround=0pt\ialign{##\crcr
     $\scriptscriptstyle\rightharpoonup$\crcr\noalign{\kern1pt\nointerlineskip}
     $\hfil\displaystyle{m}\hfil$\crcr}}}
\def\dvecm{\vbox{\mathsurround=0pt\ialign{##\crcr
     $\scriptscriptstyle\rightharpoondown$\crcr\noalign{\kern-7pt\nointerlineskip}
     $\hfil\displaystyle{m}\hfil$\crcr}}}

%   N
\def\uvecN{\vbox{\mathsurround=0pt\ialign{##\crcr
     \kern3pt$\scriptscriptstyle\rightharpoonup$\crcr\noalign{\kern1pt\nointerlineskip}
     $\hfil\displaystyle{N}\hfil$\crcr}}}
\def\dvecN{\vbox{\mathsurround=0pt\ialign{##\crcr
     \kern0pt$\scriptscriptstyle\rightharpoondown$\crcr\noalign{\kern-10pt\nointerlineskip}
     $\hfil\displaystyle{N}\hfil$\crcr}}}

%   u
\def\uvecu{\vbox{\mathsurround=0pt\ialign{##\crcr
     $\scriptscriptstyle\rightharpoonup$\crcr\noalign{\kern1pt\nointerlineskip}
     $\hfil\displaystyle{u}\hfil$\crcr}}}
\def\dvecu{\vbox{\mathsurround=0pt\ialign{##\crcr
     $\scriptscriptstyle\rightharpoondown$\crcr\noalign{\kern-7pt\nointerlineskip}
     $\hfil\displaystyle{u}\hfil$\crcr}}}

%   w
\def\uvecw{\vbox{\mathsurround=0pt\ialign{##\crcr
     $\scriptscriptstyle\rightharpoonup$\crcr\noalign{\kern1pt\nointerlineskip}
     $\hfil\displaystyle{w}\hfil$\crcr}}}
\def\dvecw{\vbox{\mathsurround=0pt\ialign{##\crcr
     $\scriptscriptstyle\rightharpoondown$\crcr\noalign{\kern-7pt\nointerlineskip}
     $\hfil\displaystyle{w}\hfil$\crcr}}}

%   v
\def\uvecv{\vbox{\mathsurround=0pt\ialign{##\crcr
     $\scriptscriptstyle\rightharpoonup$\crcr\noalign{\kern1pt\nointerlineskip}
     $\hfil\displaystyle{v}\hfil$\crcr}}}
\def\dvecv{\vbox{\mathsurround=0pt\ialign{##\crcr
     $\scriptscriptstyle\rightharpoondown$\crcr\noalign{\kern-7pt\nointerlineskip}
     $\hfil\displaystyle{v}\hfil$\crcr}}}

\def\astA{{}^\ast A}
\def\circA{{}^\circ A}
\def\astk{{}^\ast k}
\def\circk{{}^\circ k}
\def\astK{{}^\ast K}
\def\circK{{}^\circ K}
\def\astL{{}^\ast L}
\def\circL{{}^\circ L}
\def\astal{{}^\ast \al}
\def\circal{{}^\circ \al}
\def\astsi{{}^\ast \si}
\def\circsi{{}^\circ \si}
\def\aste{{}^\ast e}
\def\circe{{}^\circ e}
\def\astte{{}^\ast \te}
\def\circte{{}^\circ \te}
\def\astGa{{}^\ast \Ga}
\def\circGa{{}^\circ \Ga}
\def\Lie{\pounds}

\long\def\Hide#1{}
\long\def\HideMarked#1{\hfill{$\triangleright$}}
\def\rtau{\tau}


%%%%%%%%%%%%%%%%%%%%%%%%%%%%%%%%%%%%%%%%%%%%%%%%%%%%%%%%%%%%%%%
\def\NormalStyle{\leftskip=0cm\rightskip=0cm\normalsize\parindent=5pt\parskip=3pt\normalbaselineskip=14pt\baselineskip=\normalbaselineskip}
\def\AbstractStyle{\leftskip=1cm\rightskip=1cm\scriptsize\parindent=0pt\parskip=0pt\normalbaselineskip=11pt\baselineskip=\normalbaselineskip}\def\NoteStyle{\leftskip=1cm\rightskip=1cm\scriptsize\parindent=0pt\parskip=0pt\normalbaselineskip=11pt\baselineskip=\normalbaselineskip}

\def\ShowLabel#1{\ref{#1}}

\def\Bibliography{\begin{thebibliography}{199}\footnotesize\References}
\def\EndBibliography{\end{thebibliography}}
\def\bib#1#2{\bibitem{#1}#2}
\def\Ref#1{\cite{#1}}


\def\NewSection#1{\section{#1}}
\def\NewSubSection#1{\subsection*{#1}}
\def\NewAppendix#1#2{\section*{Appendix #1: #2}}
\def\Acknowledgements{\section*{Acknowledgements}}
\def\bs{\bigskip}
\def\ms{\medskip}
\def\ss{\smallskip}
\def\ni{\noindent}

\long\def\Note#1{\blockquote{\footnotesize {\bf Nota:}~#1\par}}
\long\def\Ex#1{\blockquote{\footnotesize {\bf Esercizio:}~#1\par}}

\def\eq#1{\begin{equation}#1\end{equation}}
\def\eqLabel#1#2{\begin{equation}#1\label{#2}\end{equation}}


\def\Cases#1{\begin{cases}#1\end{cases}}
\def\Matrix#1{\begin{matrix}#1\end{matrix}}
\def\Align#1{\begin{aligned}#1\end{aligned}}

\def\eqs#1{\eq{\Align{#1}}}
\def\eqsLabel#1#2{\eq{\Align{#1}\label{#2}}}




\def\Abstract{\AbstractStyle{\bf Abstract. }}
\def\EndAbstract{\par\NormalStyle}

\def\TitleScript{}
\def\TitleLine#1{\def\TitleScript{#1}}
\def\MoreTitleLine#1{\edef\TitleScript{\TitleScript\\#1}}

\def\AuthorScript{}
\def\AuthorLine#1{\def\AuthorScript{#1}}
\def\MoreAuthorLine#1{\edef\AuthorScript{\AuthorScript\\#1}}

\def\AddressScript{}
\def\AddressLine#1{\def\AddressScript{#1}}
\def\MoreAddressLine#1{\edef\AddressScript{\AddressScript\\#1}}

\date{}

\def\BeginDocument{\begin{document}}

\def\MakeTitle{
\begin{center}
{\Large\bf\sffamily\baselineskip=2pt\TitleScript}\\
\vskip 10pt
{\small by {\it \AuthorScript}}\\
\vskip 10pt
{\small \AddressScript}
\end{center}
\ms
}

\def\EndDocument{\end{document}}

\def\Figure[#1]#2{\begin{figure}[htbp] %  figure placement: here, top, bottom, or page
   \centering
   \includegraphics[#1]{#2} }
\def\Caption#1{\caption{#1}}
\def\EndFigure{\end{figure}}

\def\Diagram#1{\eq{
\begindc{\commdiag}[10]
#1
\enddc%
}%
}

\def\Itemize#1{\begin{itemize}#1\end{itemize}}
\def\Item[#1]{\item[#1]}

\def\AllReferences{}



\def\Shrink{\small}
%%%%%%%%%%%%%%%%%%%%%%%%%%%%%%%%%%%%%%%%%%%%%%%%%
%%%%%%%%%%%%%%%%%%%%%%%%%%%%%%%%%%%%%%%%%%%%%%%%%%%

%\BeginDocument


%%%%%%%%%%%%%%%%%%%%%%%%%%%%%%%%%%%%%%%%%%%%%%%%%%%






















\parindent=5pt
\baselineskip=13pt
\parskip=5pt





\begin{document}

\AllReferences


%%%%%%%%%%%%%%%%%%%%% Publisher's Area please ignore %%%%%%%%%%%%%%%
%
%
%%%%%%%%%%%%%%%%%%%%%%%%%%%%%%%%%%%%%%%%%%%%%%%%%%%%%%%%%%%%%%%%%%%%

\title{Matematica per adulti}

\maketitle

\NewSection{Prima settimana}

intro


\section*{Giorno 1: quanti numeri?}



Non so ci avete pensato, che possiamo contare per sempre. Dato un numero, posso sempre scrivere e dare un nome al successivo.
Curiosamente non è sempre stato così.

Io ricordo distintamente quando da piccolo ho realizzato la cosa. Scherzando, ma poi non tanto, dico spesso agli studenti che quella è stata la prima e unica
esperienza spirituale della mia vita. Per questo mi ha stupito realizzare che non tutti da piccoli sono passati per quell'esperienza.
L'ho chiesto al forum e alcuni hanno stentato a arrivare al punto, qualcuno ha dovuto essere convinto che sa il nome delnumero 66 354 427 745 367.
In fondo è sempre bello quando il mondo ti stupisce.

I romani avevano un numero finito di simboli con cui potevano scrivere un numero finito di numeri.
Per esempio con I e V potete scrivere
I
II
III
IV
V
VI
VII
VIII.

Poi per scrivere 9 dovete inventare un nuovo simbolo X.

Loro avevano (a seconda del secolo) fino a un simbolo $M$ per 1000
(poi aggiungevano barre sopra $\bar M$, $\bar{\bar M}$ per iterare la cosa ma senza cambiare il fatto 
che arrivavano a un certo numero) e poi si dovevano fermare.

Noi abbiamo lo zero, la notazione posizionale e la lingua (a parte i numeri 3, 11 e 20)
è costruita in modo analogo.
Conti fino a 10 poi 
10-n arrivi fino a  20, 
20-n arrivi fino a 30, 40, 50, ..., cento.

Poi si ricomincia con 1cento-nn, 2-cento--nn, ..., 9-cento--99.

Poi viene 1000 e si ricomincia 1-mille-nnn, 2-mille-nnn, ... nnn-mille-nnn.
Poi viene 1 milione e si ricomincia, 1-milione-nnnnnn, fino a nnn-mille-nnn-milioni-nnnnnn,
poi viene 1miliardo e si ricomincia fino a  nnn-milioni-nnn-mila-nnn-miliardi-nnn-milioni-nnn-mila-nnn.

Voi potete dire che dieci cento mille milione miliardi funzionano come i romani.
E forse è questo che trae in inganno i bambini, in fondo chi ha bisogno di contare fino a 1 miliardo?

Ma qui viene la novità, 1000-milioni-di-miliardi non hanno un nome vero 
(ce l'hanno perché la barbarie non ha limite ma non ne abbiamo bisogno).

Si chiamano 
1 miliardo-di-miliardi 

E si ricomincia. E poi 
1-milione di miliardi di miliardi-nnnnnnnnn
e così finché avete voglia.



\section*{Giorno 2: cardinali, ordinali, numerali}



\Abstract
Esistono infiniti numeri naturali.
\EndAbstract

\bs

Che differenza c'è tra numeri {\it cardinali}, {\it ordinali} e {\it numerali}?

I cardinali servono per contare gli oggetti (zero, uno, due, ...),
gli ordinali servono per l'ordine (primo, secondo, terzo, ...) definiscono maggiore e minore, le operazioni di somma e prodotto in $\N$.

I numerali sono diversi modi di rappresentare un numero: 11 si può scrivere 11 in base 10, 1011 in binario, $B$ in esadecimale, $X$ in romano.
Quando i carcerati contano i giorni sul muro della cella le 4 stanghette verticali tagliate da una stanghetta obliqua sono un numerale per 5.

Non ci occupiamo di numerali, sono solo rappresentazioni.
Esistono cardinali (uno, due, tre, ...) e ordinali (primo, secondo, terzo, ...).

Il problema è che finché parliamo di numeri finiti, cardinali e ordinali sono in corrispondenza 1-a-1.
Ok dovremo dare definizioni per bene. Lo faremo.
Ma se è così, serve uno solo dei 2 l'altro non serve.

MA quando consideriamo insiemi infiniti i numeri ordinali sono molti di più.
Quindi gli ordinali sono molto meglio.
Ci torniamo in modo più preciso ma ci vuole pazienza che ci va un po'.

Per ora stiamo ancora a partire! 
Per essere sinceri questi primi giorni servono a far capire che ci sono un sacco di problemi, anzi che più si scava e peggio è.
Ancora non abbiamo detto COSA è un numero naturale.

Qui i matematici si dividono in 2. Metà predilige le proprietà, le enuncia
e se non si cura di cosa sia un numero. L'altra metà definisce cosa è un numero naturale
(in genere come classe di equivalenza di insiemi finiti)
e poi usando questa rappresentazione definire le operazioni (eg somma e prodotto)
e per dimostrare le proprietà (eg commutativa, associativa, ...).

Siccome stiamo facendo turismo e nessuno ha fretta di arrivare, percorriamo entrambe le strade, ok?

Prima strada.
Nell'800 Peano ha dato i seguenti assiomi per i numeri naturali.

\quad
1) 0 è un numero naturale.

\quad
2) esiste una funzione $s: \N\arr \N$ e diciamo che $s(n)$ è il successore di $n$.

\quad
3) se $x$ diverso da $y$ allora $s(x)$ è diverso da $s(y)$

\quad
4) $s(x)$ non è mai 0 qualunque sia $x$

\quad
5) [Principio di induzione]

\qquad
   se $U$ è un sottoinsieme di $\N$ e
  
\qquad\quad
        a) $0 \in U$
  
\qquad\quad
        b) se $n\in U$ allora $s(n) \in U$
  
\qquad
   allora $U=\N$

Questi assiomi, con un bel po' di lavoro permettono di dimostrare OGNI proprietà dei numeri naturali (che sappiamo dimostrare altrimenti).



Se conveniamo che non c'è un numero più grande di tutti (no "$\infty$" non è un numero naturale se lo fosse, sapreste fare "$\infty+1$"), 
l'insieme dei numeri è infinito (in-finito) perché non posso finire di contare?

Per ora lasciate perdere cosa sia un numero, quello che conta è che sono infiniti, che hanno un nome, 
e che potremo fare operazioni coi numeri.



\section*{Giorno 3: somme e prodotto}



Definiamo i numeri in $\N$

$0$

$1=s(0)$

$2=s(1)= s(s(0))$

$3=s(2)= s(s(s(0)))$

$4=s(3)= s(s(s(s(0))))$

$5=s(4)= s(s(s(s(s(0)))))$

$6=s(5)= s(s(s(s(s(s(0))))))$

$7=s(6)= s(s(s(s(s(s(s(0)))))))$

$8=s(7)= s(s(s(s(s(s(s(s(0))))))))$

e avanti così.

Definiamo la {\it somma} $a+b$ con le proprietà

\quad    $a+0=a$

\quad    $a+(s(b)) = s(a+b) = (a+b)+1$

Se devo fare
 $$
    5+3=5+s(s(s(0)))= s(5+s(s(0)))=s(s(5+s(0)))=s(s(s(5))) + 0 = s(s(s(5)))
$$
che è definito come 8.

Definiamo la {\it prodotto} $ab$:

\quad   $a\cdot 1 = a$

\quad   $a\cdot s(b) = a\cdot b+a$

Se devo fare
$$
   3\cdot 2 = 3\cdot s(1) = 3\cdot 1+3 = 3+3 = 6
$$
$$
   2\cdot 3 = 2\cdot s(s(1)) = 2\cdot s(1)+2 = (2\cdot 1+2)+2 = (2+2)+2 = 4+2 = 6
$$


Poi ci mettiamo con santa pazienza e dimostriamo le proprietà.

$$
\Align{
&  	a+0 = 0+a =a \cr                
&  	(a+b)+c = a+(b+c)\cr           
&  	a+b = b+a\cr                   
&  	a1=1a=a  \cr                 
&  	(ab)c = a(bc)\cr           
&  	ab = ba \cr                  
& 	(a+b)c=ac + bc\cr
}
\qquad
\Align{
&  	\hbox{ 0 elemento neutro della somma}\cr                
&  	\hbox{ associativa della somma}\cr           
&  	\hbox{ commutativa della somma}\cr                   
&  	\hbox{ 1 elemento neutro del prodotto}\cr                 
&  	\hbox{ associativa del prodotto}\cr           
&  	\hbox{ commutativa del prodotto }\cr                  
& 	\hbox{ distributiva della somma rispetto al prodotto}\cr 	
}
$$          

Non credo siano importanti i dettagli ma se vuoi ne dimostriamo qualcuna.



\section*{Giorno 4: insiemi}


\Abstract
Fin qui non abbiamo ancora fatto nulla. Abbiamo solo scoperto che:

1) i numeri naturali sono infiniti 
 (quindi anche in prima elementare non si può parlare solo di roba finita anche se faccio 2+3).
 Cioè si può ma secondo me non si capisce cosa si lascia fuori e si lascia fuori il meglio.

2) esisteranno 2 tipi di numeri (cardinali e ordinali). 
Finché consideriamo numeri finiti non fa differenza, uno vale l'altro. 
Se consideriamo gli infiniti, invece, gli ordinali sono più fondamentali e "di più" dei cardinali.

3) i cardinali sono per contare le cose, gli ordinali per essere messi in fila (0, 1, 2, 3, ...).
Coi numeri finiti fate entrambe le cose, con quelli infiniti molti ordinali diversi hanno la stessa cardinalità.

4) Abbiamo fatto gli assiomi di Peano per i numeri naturali.
Questi permettono di definire le operazioni di somma e prodotto, di dimostrare un sacco di cose,
ma non ci dicono cosa sono i numeri naturali.
Negli approcci assiomatici uno non dice cosa sono le cose, dice le proprietà e usa solo quelle.

5) per un approccio non-assiomatico abbiamo bisogno di dire cosa è un insieme (e poi funzioni e relazioni).
Questo è un vero casino (sempre a causa degli insiemi infiniti).
\EndAbstract

\bs

Insiemi:
prima cosa da dire è che gli insiemi possono avere {\it elementi}.
Per dire che 3 è un numero naturale, chiamiamo $\N$ l'insieme dei numeri naturali e diciamo $3 \in \N$, che si legge 3 appartiene a $\N$.

Seconda cosa esiste un insieme senza elementi. 
Si chiama l'insieme vuoto, lo chiamiamo $\Emptyset$.
Siccome 2 insiemi sono uguali a meno che non produca un elemento che sta in uno ma non nell'altro, 
esiste un solo insieme vuoto o se preferite ogni insieme vuoto e uguale all'altro.

Come ho detto un insieme finito è la lista senza ripetizione dei suoi elementi.
I numeri primi minori di 10 sono l'insieme $P=\{2, 3, 5, 7\}$.
Quanti elementi ha? (ah! non potete rispondere perché ancora non abbiamo definito la cardinalità di un insieme 
e pure coi numeri andiamo ancora maluccio.)

Quando si passa agli insiemi infiniti, si è pensato (direi nell'800) di definirli dando una proprietà $P(x)$
che quando è vera per $x$ allora $x$ appartiene a $A$ ($x\in A$), quando è falsa, allora $x$ non appartiene all'insieme $A$ ($x\not\in A$).

Ad esempio i numeri pari corrispondono alla proprietà    $P(n)$: esiste $k\in N$ tale che $n=2k$.

per $n=0$, esiste $k=0$ tale che $2\cdot0=0$, quindi 0 è pari.

per $n=1$, non esiste $k\in \N$ tale che $2k=1$, quindi 1 non è pari. 

(ovviamente bisognerebbe dimostrare che non esiste, mi credete?)

per $n=2$, esiste $k=1$ tale che $2\cdot 1=2$, quindi 2 è pari.

per $n=6$, esiste $k=3$ tale che $2\cdot 3=6$, quindi 6 è pari.

[sto usando i numeri naturali di Peano per fare gli esempi]

L'insieme dei numeri pari $A=\{n\in \N: P(n)\}$ lo indichiamo pure in modo più impreciso come $A=\{0,2, 4, ..., 2k, ...\}$ come se fosse un elenco.
Lo indichiamo pure con $2\N$, ma sono solo syntax sugar.

L'unico piccola crepa nel nostro castello di buone intenzioni è che ci sono delle proprietà (antinomie) 
che non definiscono dei buoni insiemi.



\Note{
per gli informatici, quello che faccio è che definisco un linguaggio formale e una grammatica per scrivere enunciati e proposizioni.
Dico proprio con una context free grammar.
L'idea è che poi voglio definire un insieme come gli elementi che rendono vera una proposizione.
Purtroppo non si riesce ad escludere le antinomie a livello sintattico. 
Cioè si può ma si è costretti a tipizzare fortemente il linguaggio e bisogna gerarchizzare i tipi molto più rigorosamente di come fate voi nei linguaggi di oggi.
A quel punto dichiaro equivalenti 2 proposizioni che sono sempre vere o false sugli stessi oggetti e un insieme è una proposizione modulo equivalenti.
Gli assiomi di Peano in buona sostanza sono una tale proposizione e definiscono l'insieme $\N$ 
(quando scrivo  "per ogni $k \in \N$", sto dichiarando il tipo di $k$).
}



\section*{Giorno 5: operazioni tra insiemi}


Ora che sappiamo cosa è un insieme (o almeno facciamo finta) possiamo definire un po' di operazioni.

Dati 2 insiemi $A$ e $B$ definiamo l'{\it unione}, scriviamo $A \union B$ che è l'insieme 
che contiene tutti gli elementi $x$ che sono in $A$ o in $B$ 
(o in entrambi ma sono elementi dell'unione una volta sola visto che $A \union B$ è un insieme 
e nessun insieme può contenere 2 volte uno stesso elemento).

Dati 2 insiemi $A$ e $B$ definiamo l'{\it intersezione}, scriviamo $A \inter B$ che è l'insieme 
che contiene tutti gli elementi $x$ che sono in $A$ e in $B$ 
(cioè gli elementi comuni a $A$ e $B$).

Il {\it prodotto} $A \times B$ è l'insieme delle coppie $(a, b)$ con il primo elemento $a \in A$ e 
il secondo $b \in B$. In pratica
$$
A \times B = \{(a, b): a\in A, b\in B\}
$$

\Note{come puoi immaginare ora che sai come funziona la testa dei matematici prima o poi vorremo 
fare operazioni tra infiniti insiemi. Per ora ci accontentiamo.}


\Note{un altro casino è che possiamo fare $(A \times B) \times C$ che in teoria avrebbe come elementi 
coppie $((a, b), c)$ ma facciamo finta che non ce ne accorgiamo e che $A \times B \times C$ contenga terne $(a, b, c)$.
sia $A \times B \times C$, che $(A \times B) \times C$ che $A \times (B \times C)$ contengono le terne $(a, b, c)$.}

\Ex{Cosa contiene $\N \times \Emptyset$?}

Se ogni elemento di A è anche elemento di B allora A è un {\it sottoinsieme} di B e scriviamo $A \subset B$.
Ovviamente $\Emptyset$ è sottoinsieme di ogni insieme $B$ ($\Emptyset \subset B$ per qualunque $B$).
Ovviamente per ogni insieme $B$, $B \subset B$, cioè $B$ è sempre sottoinsieme di se stesso.

Dato $A$ un sottoinsieme di $B$ ($A \subset B$) possiamo definire il complemento di $A$ in $B$,
che è l'insieme $B-A$ che contiene tutti gli elementi $x \in B$ tale che non sono elementi di $A$,
cioè:
 $$
    B-A = \{x\in B: x\not\in A\subset B\}  
 $$
    
\Note{sui libri lo trovi anche definito quanto $A$ non è sottoinsieme ma a me piace di più così al momento.

}

\Note{qui è uno dei posti dove i tipi importano. Se ho $P$, insieme dei primi minori di 10, 
cioè $P=\{2, 3, 5, 7\}$ e prendessi l'insieme delle cose che non stanno in $P$ oltre a 6, in $-P$ ci troverei pure una mela, hookii, me e te.
invece faccio $\N-P$ e ci trovo tutti {\bf i numeri} che non sono in $P$. $\N$ funziona come un tipo.}


Dato un insieme $A$, $P(A)$ è l'insieme di tutti i sui sottoinsiemi, si chiama l'{\it insieme delle parti}.
Se $A = \{0, 1, 2\}$ allora 
$$
P(A) = \{\Emptyset, \{0\}, \{1\}, \{2\}, \{0,1\}, \{0,2\}, \{1,2\}, A\}
$$
ha 8 elementi, ops {\it avrà} 8 elementi.

\Note{$\{1\}$ e 1 sono due cose diverse. $1 \in A$ è un elemento di $A$. $\{1\}$ invece è un sottinsieme di $A$ che contiene un solo elemento.
Abbiamo $1\in \{1\} \subset A$.  Non si può scrivere né $1 \subset A$ né $\{1\} \in A$.

}

\Note{anche che $\Emptyset$ è un oggetto, un elemento dell'insieme delle parti $P(A)$.

}

\Ex{quanti elementi ha $P(\Emptyset)$?}

\Note{siccome gli insiemi sono definiti con delle proposizioni $P(x)$ non ti sfuggirà che l'unione intersezione e complemento
di insiemi corrisponde agli operatori logici  $.or.$, $.and.$, $.not.$ tra le corrispondenti proposizioni.
[In buona sostanza logica booleana e insiemistica sono la stessa cosa.]

}


\NewSection{Seconda settimana}

intro

\section*{Giorno 6: relazioni e funzioni}



Introduciamo 2 concetti molto importanti per dopo.

Dati 2 insiemi $A$ e $B$,
chiamiamo {\it relazione} da $A$ a $B$ un sottoinsieme $R\subset A\times B$.
Diciamo che $a\in A$ è in relazione con $b\in B$ se e solo se $(a, b)\in R$

Esempio: definiamo una relazione su $\N$ (che significa da $\N$ a $\N$)
che descrive la divisibilità. 
Diciamo che $(k, n)\in R\subset \N\times \N$ (che "$k$ divide $n$" e scriviamo $k|n$) se il numero naturale $k$ divide esattamente $n$.
Ad esempio abbiamo $1|10$, $2|10$, $5|10$, $10|10$, ma non $3|10$.
Possiamo quindi pensare alla relazione $|$ come il sottoinsieme 
$$
I=\{(k, n)\in \N\times \N: \hbox{ esiste $q\in \N: n=qk$}\}
$$

Ci interessano 3 tipi di relazioni:

1) relazioni di equivalenza su un insieme $A$

2) relazioni di ordine su un insieme $A$

3) le funzioni da un insieme $A$ a un insieme $B$ (che può essere uguale a o diverso da $A$)

Una relazione di equivalenza $\sim$ è una relazione con le seguenti proprietà (li\-be\-ra\-mente ispirate da =)

1) $x\in A: x\sim x$  (ogni elemento di $A$ è in relazione con se stesso)

2) $x, y\in A:$ se $x\sim y$ allora anche $y\sim x$ (se $x$ è un relazione con $y$ allora anche $y$ è in relazione con $x$)

3) $x, y\in A:$ se  $x\sim y$ e  $y\sim z$ allora anche $x\sim z$.

\Note{La relazione di divisibilità gode della proprietà 1 (per ogni $n\in \N: n|n$).
Gode pure della proprietà 3)  ($k|n|m$ allora $k|m$).
Ma non gode della proprietà 2) ($3|6$ na non $6|3$).
Quindi non è una relazione di equivalenza.
}

\Ex{definiamo la relazione di equivalenza su $\N$. Diciamo che $n=_{(3)}k$ 
se esistono $q, p\in \N$ e $r\in \{0, 1, 2\}$ tale che  $n= 3q+r$ e $k= 3p+r$. 
Ad esempio $7/3$ ha resto 1, $10/3$ ha resto 1, quindi abbiamo che $7=_{(3)} 10$.
Siccome $12/3$ ha resto 0, non $12\not=_{(3)} 10$.
È facile convincersi che questa relazione d'equivalenza separa $\N$ in tre sottoinsiemi:

$R_0=\{n\in \N: \hbox{resto di $n/3$ è $0$}\}$, cioè dei numeri che si dividono esattamente per $3$.

$R_1=\{n\in \N: \hbox{resto di $n/3$ è $1$}\}$, cioè dei numeri che hanno resto 1 quando li dividete per $3$.

$R_2=\{n\in \N: \hbox{resto di $n/3$ è $2$}\}$, cioè dei numeri che hanno resto 2 quando li dividete per $3$.

Ogni numero naturale sta in uno e solo uno tra $R_0$, $R_1$, e $R_2$.
In altre parole, l'unione $R_0 \union R_1\union R_2 = \N$ e
$R_0\inter R_1=\Emptyset$,
$R_0\inter R_2=\Emptyset$,
$R_1\inter R_2=\Emptyset$.


Questi sottoinsiemi $R_0$, $R_1$, $R_2$, si chiamano le classi di equivalenza della relazione $k=_{(3)}n$.
Potete rimpiazzare $3$ con ogni numero naturale $k$ che definisce $k$ classi di equivalenza a seconda del resto
$r=0,1, ..., k-1$.}

\Note{Questa costruzione è piuttosto generica. Ogni volta che voglio dividere un insieme $A$ in classi di equivalenza,
lo fate definendo una relazione di equivalenza in modo da spezzare l'insieme $A$ nelle classi di equivalenza desiderate.
Le 3 classi di equivalenza $R_0$, $R_1$, $R_2$, tengono conto dei possibili resti e trascurano il rapporto.
Abbiamo che $10= 3\cdot 3+1$ e $7=3\cdot2+1$, quindi $10=_{(3)}7$ perché hanno lo stesso resto anche se il rapporto con 3 è diverso
(3 e 2, rispettivamente). 
Le relazioni di equivalenza sono di preciso un modo di considerare equivalenti elementi diversi che però hanno 
in comune alcune caratteristiche (il resto), e considerare ininfluenti le altre (il rapporto).
}


Delle relazioni di ordine ce ne occupiamo dopo, che ce ne sono tipi sottilmente diversi.

Invece definiamo una funzione $f:A\arr B$ una relazione tra $A$ e $B$ tale che 
per ogni $a\in A$ esiste uno e in solo elemento $b\in B$ tale che $(a, b)$ sono in relazione.
Se $(a, b)$ sono un relazione scriviamo $b=f(a)$ e diciamo che $b$ è l'immagine di $a$ attraverso la funzione $f$.
Per riassumere tutto ciò, scriviamo una funzione $f: A\arr B: a \mapsto f(a)$
per dire pure che la funzione $f$ associa a $a$ l'elemento $f(a)=b$.

 
\Note{A questo punto probabilmente vi chiedete perché uno deve fare le cose così complicate invece di dire che
una funzione è $y=3x+1$, tu mi dai un valore $x$ e io uso la funzione per calcolarmi il valore di $y$.
Il problema è che poi uso funzioni in contesti diversi (coi numeri ma pure tra insiemi di spazi, di equazioni, ...).
In questo modo, facciamo il lavoro una volta sola anche se serve un po' di sforzo iniziale ad abituarsi useremo molto di più questi concetti che i numeri!}

Esempi di funzioni $f:\N\arr \N$ considerate la funzione $s:\N\arr \N: n\mapsto n+1$ 
che ad ogni numero naturale associa il suo successivo. Questo è la funzione che compare negli assiomi di Peano.
Solo bisogna sapere fare $n+1$ cioè aver definito la somma in $\N$. Gli assiomi di Peano sono emunciati prima di dire come si fa la somma.
Abbiamo definito la somma su $\N$ usando la funzione $s:\N\arr \N$ invece che usare la somma per definire $s$.

Sono anche funzioni $f:\N\arr\N: n\mapsto 2n$, $f:\N\arr\N: n\mapsto n^2$, mentre non è una funzione
$\sqrt{\ }:\N\arr \N: n\mapsto \sqrt{n}$. Perché non è una buona funzione?

\Note{Guardate che a questo tipo di domande si risponde facilmente andando a prendere la definizione e capendo perché non funziona.
Una funzione per un matematico è esattamente quello che c'è scritto nella definizione. Se un matematico definisce un cane come un quadrupede, 
allora per lui un gatto è un cane.
}

Una funzione $f:A\arr B$ è detta biettiva se oltre al fatto che ad ogni elemento $a\in A$ corrisponde una e una sola immagine $f(a)\in B$
(se no $f$ non è neanche una funzione) vale pure che per ogni $b\in B$ esiste un solo elemento $a\in A$ tale che $f(a)=b$ (cioè se esiste una e una sola {\it controimmagine} di $b$)


\Ex{
$f:\N\arr\N: n\mapsto 2n$, $s:\N\arr \N: n\mapsto n+1$, $f:\N\arr\N: n\mapsto n^2$   
non sono biettive. Perché?

Per rispondere dovete trovare un numero naturale $b$ che non è prodotto come immagine da nessun $a\in \N$
oppure che è prodotto come immagine da più di un numero. }


\Ex{definiamo i numeri pari $P=\{n\in \N: \hbox{ esiste $k\in \N: n=2k$}\}$.
La mappa $f: \N\arr P: n\mapsto 2n$ è una funzione? È biettiva?
}


Ora abbiamo gli ingredienti per definire la cardinalità degli insiemi e dire cosa significa contare.
La prossima volta.



\section*{Giorno 7: che stai a contare?}



Se ti do una scatola di caramelle e ti chiedo quante caramelle contiene tu che fai?

Tiro a indivinare. Apri la scatola prendi una caramella e conti 1, prendi un'altra caramella e conti 2, ...,
prendi la 13ma caramella e conti 13. Non ci sono più caramelle e dici che nella scatola c'erano 13 caramelle.

Ora lasciami fare una domandina. Che hai fatto in tutto ciò se non istituire una funzione biettiva tra le caramelle e il sottoinsieme $I_{13}=\{1, 2, 3, \dots, 12, 13\}\subset \N$?

Contare {\it significa} stabilire una funziona bilineare tra un insieme da contare (la scatola di caramelle) e un sottoinsieme finito di $\N$.

Definizione: dati 2 insiemi $A$ e $B$, diciamo che hanno la stessa {\it cardinalità} se esiste una funzione biettiva $f:A\arr B$.

Avere la stessa cardinalità è una relazione di equivalenza sugli insiemi (e pure sugli insiemi infiniti).
Le classi di equivalenza rispetto a questa relazione di equivalenza sono, ad esempio, tutti gli insiemi con 17 elementi.
Esiste una classe di equivalenza ogni $n\in\N$, fatta di tutti gli insiemi finiti con $n$ elementi.

Le classi di equivalenza {\it sono} una rappresentazione dei numeri naturali. Il numero $17\in\N$ è identificato con tutti gli insiemi finiti con 17 elementi.

Dobbiamo notare 2 cose: primo, il numero è per definizione astratto, non importa se conti mele, pere, colori o unicorni.
Secondo, la definizione di avere la stessa cardinalità si estende per costruzione anche agli insiemi infiniti.

Gli insiemi che hanno la stessa cadinalità di $\N$ sono detti {\it numerabili}. Il numero cardinale corrispondente si chiama $\aleph_0$ (letto {\it aleph-zero}).
Ovviamente, un insieme numerabile è in corrispondenza biunivoca con $\N$, quindi  $\aleph_0\not\in\N$ perché non può essere in corrispondenza con un sottoinsieme finito di $\N$.
Sono i sottoinsiemi {\it finiti} di $\N$ che definiscono gli elementi di $\N$. 

\Ex{sono più i numeri naturali o i numeri pari? 

[Siate certi di considerare la mappa $f:\N\arr 2\N: n\mapsto 2n$.
(È biettiva? Quindi?)]
}

\Ex{Se consideriamo $A=\N\union \{a\}$. È numerabile? Chi ha maggiore cardinalità, $\N$ o $A$?}

\Note{notate che gli assiomi di Peano definiscono i numerali naturali, dicono che 0 è il primo, che esiste sempre il successivo e che il successivo e sempre un numero nuovo, che continuano per sempre a comparire numeri nuovi e che tutti i naturali sono ottenuti come il successivo di un numero naturale. Questo definisce già un (buon) ordine di $\N$.

Poi definiamo la cardinalità dei numeri naturali che si estende agli insiemi numerabili.
}

Ora prima di definire per bene il buon ordine andiamo in vacanza in montagna all'Hilbert hotel.


\section*{Giorno 8: l'hotel di Hilbert (numerabile)}



L'hotel Hilbert è un albergo in un'amena località montana non meglio precisata che ha $\aleph_0$ stanze.
Un giorno l'albergo risulta completo al momento della cena.

Nel mezzo della notte tempestosa un nuovo cliente bussa alla porta cercando una stanza per passare la notte al sicuro.
In un primo momento il portiere risponde che sono al completo poi, mosso a compassione per il nuovo venuto, pensandoci un po' ha un'idea e trasmette in tutte le camere il seguente messaggio:

{\it Stiamo affrontando un'emergenza, chiediamo ad ogni cliente a seguire le seguenti istruzioni:
Se state occupando la nella stanza $k$, vi preghiamo di prendere la vostra roba e trasferirvi nella stanza $k+1$.
In cambio, riceverete uno sconto del 10\% sul conto.}

Dopo di ciò, la stanza $0$ è rimasta vuota visto che nessuno ci si è trasferito a fronte del suo occupante che ora dorme seneramente nella stanza $1$.
Il nuovo cliente quindi viene sistemato della stanza $0$, non prima di essersi offerta a pagare di tasca sua il mancato incasso dovuto allo sconto.
A questo proposito, il portiere ha gentilmente declinato l'offerta argomentando che malgrado le tariffe molto basse dell'albergo e lo sconto, l'incasso totale non sarebbe diminuto finché l'albergo fosse risultato completo o, per quel che conta, anche avesse avuto un numero finito di stanze vuote, o comunque con un numero numerabile di stanze occupate (ad esempio solo quelle pari).

\ms
Più tardi nella notte, un'altra infinita comitiva si presenta alla porta chiedendo una stanza per evitare la tempesta.
A quel punto il portiere trasmette il seguente messaggio:
  
{\it Mi spiace disturbarvi ancora ma stiamo affrontando una nuova emergenza.
Vi chiediamo gentilmente di alzarvi, prendere le vostre cose, e se state nella stanza $k$ trasferitevi nella stanza $2k+1$.
In cambio, riceverete un ulteriore sconto del 10\%.}

Ciò ha reso disponibili tutte le stanze pari che così hanno potuto essere destinate alla comitiva giunta nella notte.
Sembra che l'hotel di Hilbert non possa esaurire le stanze, [Ma può, come vedremo.]


\bs
Se ci pensate, tutta la storiella dice che la cosa particolare dell'albergo con $\aleph_0$ stanze è che le posso numerare coi numeri naturali $\N$.
Tuttavia le sole stanze pari, quelle dispari da sole, e tutte le stanze hanno la stessa cardinalità.
Non ci credete?

Allora controllate che la mappa $f:\N\arr P: n\mapsto 2n$, 
dove 
$$
P=\{0, 2, 4, \dots, 2k, \dots\}
$$ 
è biettiva,
così come la mappa $g:\N\arr D: n\mapsto 2n+1$, dove 
$$
D=\{1, 3, 5, \dots, 2k+1, \dots\}
$$

Ok se ci pensate, magari dite:  {\it e grazie dai sono entrambe infinite quindi sono tanti quanti}.
Se è così ripetete con me: {\it la natura è malevola e non perde occasione di rendere le cose semplici meravigliosamente complicate}.
Avete in parte ragione ma vedremo che ci sono infiniti e infiniti (se no perché la cardinalità di $\N$ l'avrei chiamata $\aleph_0$? Perché mi avanzava uno 0 
o perché alla fine mi servirà una cosa più grande $\aleph_1$, poi una più grande $\aleph_2$, \dots?).

Ma ora non possiamo occuparcene, prima dobbiamo definire i numeri interi, quelli razionali, quelli reali e vedremo che la cardinailtà dei numeri reali è più grande di quella di $\N$, cioè che i numeri reali sono una infinità non numerabile. 

E vi sembra possibile quella sia la fine della storia?
No, eh, vedete che la natura è malevola?

Prossime tappe: relazioni di ordine e buon ordine, così possiamo definire una rappresentazione (un modello) per i numeri ordinali.
Poi ci dobbiamo occupare di definire i numeri interi, i razionali e i reali, estendendo le operazioni e definendone altre (ad esempio: la sottrazione, la divisione, le radici quadrate).
A quel punto siamo arrivati ai tempi di Pitagora (che, non lui, diciamo la sua scuola, ha scoperto che $\sqrt{2}$ non è razionale).

Per questa strada estenderemo 2 volte l'hotel di Hilbert. A quel  punto il finale di Interstellar vi sembrerà banalotto.


\section*{Giorno 9: relazioni d'ordine}

Così come abbiamo visto che le relazioni di equivalenza catturano la nostra intuizione di uguaglianza sotto certi aspetti (2 persone diverse possono essere diverse ma uguali sotto la relazione d'equivalenza "sono alte uguali" oppure "pesano uguale" oppure "hanno lo stesso sesso"), le {\it relazioni d'ordine} catturano la nostra intuizione di ordine in un insieme.
A noi interessano principalmente gli ordini totali e i buoni ordini, ma ci sono diverse gradazioni di grigio e non conviene puntare dritto all'obiettivo, conviene fare lo sforzo di catalogare i diversi tipi di relazione d'ordine.

Definizione: un {\it preordine} su un insieme $A$ è una relazione $\le$ che soddisfa la proprietà  per ogni $x, y, z\in A$
$$
\Align{
&x\le x\cr
&\hbox{se   $x\le y$ e $y\le z$ allora $x\le z$ }
}\qquad
\Align{
\hbox{(riflessiva)}\cr
\hbox{(transitiva)}\cr
}
$$

\Note{le relazioni di equivalenza soddisfano la proprietà riflessiva e la proprietà transitiva, quindi sono preordini.
Non è vero il viceversa: non tutti i preordini sono relazioni di equivalenza perché non è richiesta la proprietà simmetrica.

Quindi la strada giusta dovrebbe essere definire i preordini e poi le relazioni di equivalenza come particolare preordini che hanno anche la proprietà simmetrica.
Siccome dobbiamo tenere a mente un sacco di proprietà e abbiamo un numero finito di neuroni dovremmo abituarci a riorganizzare continuamente la nostra conoscenza in questo modo.
Questa è una cosa che ci insegna la matematica anche se non ci interessa essere matematici: la conoscenza {\it deve} essere continuamente coccolata e riorganizzata mentre cresce, se no ci bastava Wikipedia.

Questa è, secondo me, {\it una} ragione per provare ad insegnare matematica per 13 anni a tutta la popolazione anche se evidentemente con scarsissimi risultati.
Tra parentesi questo secondo me indica che lo scopo della scuola pubblica, quella dell'obbligo, non è insegnare cose ai bambini. 
I bambini sono costretti ad andare a scuola, come i carcerati sono costretti a stare in carcere. Non puoi chiedere a un carcerato di stare 13 anni in carcere e pure imparare a dare il bianco alla cella, al massimo puoi {\it consentirgli} di dare il bianco alla cella. La scuola dell'obbligo ha il compito di esporre tutta la popolazione al maggior numero di cose possibile e {\it consentire} loro di sviluppare i loro interessi in qualcuna di queste materie. Possibilmente di raggiungere un livello civile di navigazione in quelle discipline che non interessano.
}

Definizione: un {\it ordine parziale} è un preordine con la proprietà antisimmetrica
$$
\hbox{se $x\le y$ e $y\le x$ allora $x=y$}
$$

\Ex{Dato un insieme $A$ e l'insieme $P(A)$ delle sue parti, diciamo che un sottoinsieme $S_1\in P(A)$  di $A$ è {\it incluso} in un  sottoinsieme $S_2\in P(A)$, e scriviamo $S_1\subset S_2$
se e solo se $S_1$ è anche un sottoinsieme di $S_2$. 

L'inclusione è un ordine parziale di $P(A)$. 

Ogni sottoinsieme è contenuto in se stesso (riflessiva).
In più se $S_1\subset S_2$ e $S_2\subset S_3$ allora $S_1\subset S_3$ (transitiva, quindi è un preordine).

Infine se $S_1\subset S_2$ significa che tutti gli elementi di $S_1$ sono anche elementi di $S_2$, e anche se $S_2\subset S_1$ significa che tutti gli elementi di $S_2$ sono anche elementi di $S_1$, allora $S_1$ e $S_2$ hanno gli stessi elementi, quindi sono lo stesso sottoinsieme (antisimmetrica).

Quindi l'inclusione è un ordine parziale.
}

Definizione: un {\it ordine totale} è un ordine parziale che in più ha la proprietà di comparazione,
dati $x, y\in A$ si ha che $x\le y$ oppure $y\le x$.

\Note{L'inclusione è una ordine parziale che non è totale (due sottoinsiemi possono essere tali da non essere inclusi né $S_1\subset S_2$ né $S_2\subset S_1$).

I numeri naturali sono ordinati totalmente dalla relazione $n_1\le n_1$ che è un ordine totale in $\N$.

Per la cronaca dobbiamo definire tutto, quindi pure cosa significa $n_1\le n_2$.
Se definiamo i numeri naturali con gli assiomi di Peano, diciamo che $n_1\le n_2$ dicendo che valgono le seguenti proprietà
$$
n\le n
\qquad\qquad
n\le s(n)
$$

Se vogliamo dimostrare che $3\le 5$ non dobbiamo fare altro che notare che $5= s(s(3))$, quindi
$$
3 \le s(3) \le s(s(3)) = 5
$$
quindi $3\le 5$ per la proprietà transitiva.
}

Un {\it buon ordine} di $A$ è un ordine totale tale che ogni sottoinsieme $S$, non-vuoto di $A$ ha un elemento minimo, cioè più piccolo (rispetto all'ordine totale che stiamo considerando su $A$) di tutti gli altri elementi di $S$.
L'insieme dei numeri naturali $\N$ con l'ordine $\le$ è ben ordinato. 
Prendete un qualunque sottoinsieme non vuoto di $\N$ in esso esiste sempre un minimo. 

\Note{Notate che non è vero che esiste sempre un massimo (abbiamo già detto che in $\N$, che è un sottoinsieme non-vuoto di $\N$, non esiste un numero più grande di tutti gli altri).

Se vi gira la testa, non vi preoccupate è come per chi vive al mare andare sulle Ande, è scarsità di ossigeno dopo un paio di giorni (o masticando foglie di coca) passa.
Sì, stiamo dicendo che {\it ogni} sottoinsieme non-vuoto di $\N$ ha un minimo anche se i sottoinsiemi di $\N$ sono in numero infinito, anche se alcuni sottoinsieme di $\N$ sono infiniti.
Ok, dovremmo dimostrarlo ma per quello hanno inventato le dimostrazioni, perché consentono di dimostrare infinite cose (nessun pari maggiore di $3$ è primo, e grazie si divide per 2 oltre che per 1 e per $n$).

Ok non abbiamo ancora detto cosa è un primo, ma lo sapete. Ma questo vi conferma che la paranoia principale del matematici è non fare ragionamenti circolari.
Non assumere cose senza dimostrazione e definizione che alla fine renda sbagliato il ragionamento. Per questo, per fare le cose per bene bisogna andare piano, dare gli assiomi dare le definizioni, dimostrare i teoremi e poi magari dare degli esempi. Non si può fare come stiamo facendo qui, dare gli esempi prima usando cose che non sono state definite. È pericoloso.

Poi quindi capite perché i matematici bestemmiano quando gli si dice {\it eh ma la matematica è astratta, non puoi parlare come mangi e fare un esempio di quello che mi stai dicendo così capisco?} 

La risposta dovrebbe essere: {\it no ora non posso darti degli esempi, prima dimostriamo i teoremi poi tra 3 giorni ti faccio un esempio! Non me ne frega nulla della tua intuizione, se potevi intuire cosa era uno spazio di Hilbert, significa che stai pensando a un esempio con certe proprietà, è pericoloso avere in mente un esempio perché facilita aggiungere proprietà che magari il tuo esempio ha ma non tutti gli spazi di Hilbert e quando dimostri i teoremi è pericoloso avere delle proprietà in mente perché si finisce per sparare stronzate.}

A proposito, sui principia matematica di Russell il teorema $1+1=2$ è marcato col numero 10mila e qualcosa. Diecimila e rotti teoremi prima di sapere che 1+1=2.
La matematica è lenta.
}




\section*{Giorno 10: odinali e albergo MultiHilbert}

Dati due insiemi $(A, \le)$ e $(B\le)$ bene ordinati, diciamo che hanno la stessa {\it lunghezza}
se esiste una mappa biettiva $f:A\arr B$ che preserva l'ordine, cioè che se $a_1\le a_2$ allora $f(a_1)\le f(a_2)$.
Questa è una relazione di equivalenza e possiamo considerare classi di equivalenza di insiemi bene ordinati che hanno la stessa liunghezza.

Ognuna di queste classi è un {\it ordinale}. 
L'ordinale degli insiemi di 3 elementi coincide con gli insiemi di cardinalità 3, quindi l'ordinale 3 e il cardinale 3 sono rappresentati dalla stessa classe di insiemi
pur avendo definizione diversa.

L'insieme dei numeri naturali $\N$ è bene ordinato quindi determina un ordinale $\om_0$ infinito, che ha cadinalità $\aleph_0$. 
Se consideriamo l'insieme $\N_{+2}=\N \union \{a, b\}$ con il buon ordine dato da  $n\in\N: n\le a\le b$, allora $\N_{+2}$ definisce un altro ordinale $\om_0+2$
(esistono mappe biettive tra $\N$ e $\N_{+2}$, ma non mappe biettive che preservino l'ordine). Tuttavia la cardinalità di $\N$ e $\N_{+2}$ è la stessa ed è sempre $\aleph_0$.
Quindi ci sono più ordinali diversi di cardinali diversi.

Un ordinale è detto {\it ordinale limite} se non ha un {\it precedente} ad esempio $\om_0$ è un ordinale limite, mentre $\om_0+1$ ha $\om_0$ come precedente perché
$s(\om_0)= \om_0+1$.

Gli ordinali soddisfano gli assiomi di Peano.
In particolare, se abbiamo un insieme che è ordinale $K=\{1, 2, \dots, k\}\subset \N$ possiamo definire il successivo $K+1=\{1, 2, \dots, k, k+1\}\subset \N$

A questo punto: plot twist!
Esiste una cosa che si chiama {\it assioma della scelta}:

\Note{
Dato un insieme $X$ e ogni famiglia $A_\al\subset X$ di sottoinsiemi etichettati da $\al\in I$, esiste sempre una funzione $\si:I\arr X: \al\mapsto \si(\al)\in A_\al$
che sceglie un elemento $\si(\al)$ in ogni sottoinsieme $A_\al$ della famiglia.

Occhio che la famiglia può essere finita, infinita (numerabile o non numerabile).
Ad esempio, se prendiamo $I=\R$ e $A_x=\{(x, y): y\in \R, x\in I\}\subset \R\times \R$ allora esiste una funzione $\si:\R\arr \R:x\mapsto \si(x)$.
Vi sembra ovvio? Beh da un lato questo è il motivo per cui è un assioma, dall'altro è perché non avete idea di quanto può essere infinito un insieme infinito.
E comunque per scegliere la funzione $\si$ ci vorrebbe infinito tempo visto che devo scegliere infiniti elementi $\si(\al)\in A_\al$ e in generale ogni $A_\al$ può essere fatto a modo suo.
}

Con l'assioma della scelta si può dimostrare che su ogni insieme esiste un buon ordine.
Questo significa che potete ordinare $\Q$ (o $\R$) in modo che ogni sottoinsieme abbia un minimo! 
E se ci pensate un attimo non avete la più pallida idea di come questo ordinamento sia fatto.

Ora se ogni insieme $X$ ha (almeno) un buon ordinamento, significa che ogni coppia $(X, \le)$ (ogni insieme ben ordinato) definisce un ordinale.
Quindi ad esempio pure $\R$ definisce un ordinale.

\ni\ms{\bf Teorema:} la cardinalità di $[0,1)$ è più grande della cardinalità di $\N$.

\Note{
Se $I=[0,1)$ avesse la cardinalità di $\N$, sarebbe numerabile, cioè esisterebbe una mappa biettiva $\la:\N\arr I$, il che significa che potrei elencare gli elementi di $I$
facendo una lista $(\la(0), \la(1), \la(2), \dots)$. Questa lista sarebbe una cosa tipo
$$
\Align{
&0 \mapsto 0.277394776559\dots \cr
&1 \mapsto 0.000473656649\dots \cr
&2 \mapsto 0.9983664756349\dots \cr
&3 \mapsto 0.1227365549000\dots \cr
&\quad\dots\cr
}
$$
ma si può sempre costruire un numero $x\in I$ che non sta nella lista (quindi la funzione $\la$ non può essere biettiva).
Basta prendere $x=0.d_1 d_2 d_3 d_4\dots$
dove $d_1$ è una cifra diversa dalla prima cifra di $\la(0)$,
dove $d_2$ è una cifra diversa dalla seconda cifra di $\la(1)$,
dove $d_3$ è una cifra diversa dalla terza cifra di $\la(2)$,
e avanti così.

Il numero così costruito è diverso da tutti i numeri nella lista per almeno una cifra.
}

Ok, ci sono delle cosucce da sistemare ma in sostanza questo si chiama dimostrazione {\it diagonale di Cantor}.
I punti di $I$ sono infiniti, ma sono di più dei punti di $\N$.
Tra l'altro non si può dimostrare né che ci siano, né che non ci siano infiniti più grandi di $\N$ ma più piccoli di $\R$.
Pure questo è un assioma della matematica. Usualmente si assume che non ce ne siano e si definisce $\aleph_1$ la cadinalità di $\R$.
Questo assioma si chiama {\it ipotesi del continuo}.

Tra l'altro poi le funzioni $f:\R\arr\R$ sono infinite di una cadinalità più grande e si assume siano $\aleph_2$ e evanti così.


Quindi voi potete contare 
$$
0\>
1\>
 2\>
  3\> 
  \dots\>
  \om_0\>
  \om_0+1\>
  \om_0+2\>
  \om_0+3\>
  \dots\>
  2\om_0\>
  \dots\>
  3\om_0\>
   \dots\>
 4 \om_0\>
  \dots\>
 (\om_0)^2
  \dots\>
 (\om_0)^3
  \dots\>
  \dots\>
$$
Ma siccome pure $\R$ definisce un ordinale prima o poi incontro un ordinale che non è più numerabile (che si chiama il {\it primo ordinale non numerabile},
che scriviamo $\om_1$,
perché gli ordinali non numerabili sono un sottoinsieme dei numerabili e quindi devono avere un minimo).
Poi si continua a contare fino ad arrivare a $\R$ (e tutti gli ordinali da $\om_0$ a $\R$ hanno necessariamnete cardinalità $\aleph_1$ per l'ipotesi del continuo.)


Mal di testa?

Ora possiamo rifare la storiella dell'albergo di MultiHilbert.
Quello di prima è solo il piano terra che ha $\om_0$ camere, poi c'è il primo piano con le camere da $\om_0+1$ a $2\om_0$.
E via così per tutti i piani numerabili.

Questo ha sempre $\aleph_0$ stanze tante quante le camere dell'albergo originale.

Poi abbiamo la dependance $\aleph_1$ che comincia con la stanza $\om_1$ e pure lei ha infiniti piani infiniti.

Poi la dependance $\om_2$ (il primo ordinale di cardinalità $\aleph_2$) e avanti così all'infinito. 

\Note{quando Cantor ha scritto sta roba (intorno al 1850), il suo maestro Kronecher, ha detto che era pazzo. 
Questi si chiamano numeri transfiniti di Cantor.
Oggi non c'è dubbio che Cantor avesse ragione. È tutto ben definito, tutto dimostrabile, tutto certificato.

Detto questo, Cantor aveva manie di persecuzione, forse era bipolare e passava 6 mesi fuori e sei mesi dentro il manicomio.}

\NewSection{Terza settimana}

intro



\EndDocument
\end


