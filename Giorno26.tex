% !TEX TS-program = latex
% !TEX spellcheck = it-IT
\documentclass[a4paper]{article}
\usepackage[utf8]{inputenc}
\usepackage[T1]{fontenc}
\usepackage[italian]{babel}
\usepackage{tikz} % LATEX
\usetikzlibrary {automata}


%%%%%%%%%%%%%%%%%%%%%%%%%%%%%%%%%%%%%%%%%%%%%%%%%%%%%%%%%%%%%%%%%%%%%%%%%%%%
%% Trim Size: 9.75in x 6.5in
%% Text Area: 8in (include Runningheads) x 5in
%% ws-ijgmmp.tex   :   2-9-08
%% Tex file to use with ws-ijgmmp.cls written in Latex2E.
%% The content, structure, format and layout of this style file is the
%% property of World Scientific Publishing Co. Pte. Ltd.
%% Copyright 1995, 2002 by World Scientific Publishing Co.
%% All rights are reserved.
%%%%%%%%%%%%%%%%%%%%%%%%%%%%%%%%%%%%%%%%%%%%%%%%%%%%%%%%%%%%%%%%%%%%%%%%%%%%
%%

%%%%%%%%%%%%%%%%%%%%%%%%%%%%%


\usepackage{pictexwd,dcpic}
\usepackage{csquotes}
\usepackage{nopageno}

\usepackage{amsmath, amssymb, amsthm}
\usepackage{pictex, dcpic}
\usepackage{color}
\usepackage{hyperref}
%\usepackage{graphicx}
%\numberwithin{equation}{section} % This line resets equation numbering when starting a new section.
%\renewcommand{\theequation}{Eq. \thesection.\arabic{equation}} % This line ads "Eq." in front of your equation numbering.



\def\http#1#2{\href{#1}{#2}}


%
%    Macros.    Version 1.2.0.beta
%    The best use is to paste all of them into the papers
%     1/8/2005
%

%%%%%%%%%%%%%%%%%%%%%%%%%%%%%%
%%%%%%			Greek		 %%%%%%
%%%%%%%%%%%%%%%%%%%%%%%%%%%%%%

\def\al{\alpha}
\def\be{\beta}
\def\de{\delta}
\def\ga{\gamma}
\def\up{\upsilon}
\def\ep{\epsilon}
\def\io{\iota}
\def\te{\theta}
\def\la{\lambda}
\def\ze{\zeta}
\def\om{\omega}
\def\si{\sigma}
\def\vp{\varphi}
\def\vpi{\varpi}
\def\ka{\kappa}
\def\vs{\varsigma}
\def\vr{\varrho}

\def\De{\Delta}
\def\Ga{\Gamma}
\def\Te{\Theta}
\def\La{\Lambda}
\def\Om{\Omega}
\def\Si{\Sigma}
\def\Up{\Upsilon}

\def\boldDe{\bf\Delta}
\def\boldGa{\bf\Gamma}
\def\boldTe{\bf\Theta}
\def\boldLa{\bf\Lambda}
\def\boldOm{\bf\Omega}
\def\boldSi{\bf\Sigma}
\def\boldUp{\bf\Upsilon}
\def\boldXi{\bf\Xi}
\def\boldPi{\bf\Pi}
\def\boldPhi{\bf\Phi}
\def\boldPsi{\bf\Psi}

\def\boldal{\bf\al}
\def\boldbe{\bf\be}
\def\boldga{\bf\ga}
\def\boldde{\bf\de}
\def\boldep{\bf\ep}
\def\boldze{\bf\ze}
\def\boldeta{\bf\eta}
\def\boldte{\bf\te}
\def\boldio{\bf\io}
\def\boldka{\bf\ka}
\def\boldla{\bf\la}
\def\boldmu{\hbox{\greekbold\char "16}}
\def\boldnu{\hbox{\greekbold\char "17}}
\def\boldxi{\hbox{\greekbold\char "18}}
\def\boldpi{\hbox{\greekbold\char "19}}
\def\boldrho{\hbox{\greekbold\char "1A}}
\def\boldsi{\bf\si}
\def\boldtau{\hbox{\greekbold\char "1C}}
\def\boldup{\hbox{\greekbold\char "1D}}
\def\boldphi{\hbox{\greekbold\char "1E}}
\def\boldchi{\hbox{\greekbold\char "1F}}
\def\boldpsi{\hbox{\greekbold\char "20}}
\def\boldom{\hbox{\greekbold\char "21}}
\def\boldvep{\hbox{\greekbold\char "22}}
\def\boldvte{\hbox{\greekbold\char "23}}
\def\boldvpi{\hbox{\greekbold\char "24}}
\def\boldvrho{\hbox{\greekbold\char "25}}
\def\boldvsi{\hbox{\greekbold\char "26}}
\def\boldvphi{\hbox{\greekbold\char "27}}

%%%%%%%%%%%%%%%%%%%%%%%%%%%%%%
%%%%%%			Cal			 %%%%%%
%%%%%%%%%%%%%%%%%%%%%%%%%%%%%%
 \def\calA{{\hbox{\cal A}}}
 \def\calU{{\hbox{\cal U}}}
 \def\calB{{\hbox{\cal B}}}
 \def\calC{{\hbox{\cal C}}}
 \def\calI{{\hbox{\cal I}}}
 \def\calQ{{\hbox{\cal Q}}}
 \def\calP{{\hbox{\cal P}}}
 \def\calL{{\hbox{\cal L}}}
 \def\calE{{\hbox{\cal E}}}
 \def\calW{{\hbox{\cal W}}}
 \def\calV{{\hbox{\cal V}}}
 \def\calS{{\hbox{\cal S}}}
 \def\calF{{\hbox{\cal F}}}
 \def\calR{{\hbox{\cal R}}}
 \def\calO{{\hbox{\cal O}}}
 \def\calM{{\hbox{\cal M}}}

%%%%%%%%%%%%%%%%%%%%%%%%%%%%%%
%%%%%%			gothic		 %%%%%%
%%%%%%%%%%%%%%%%%%%%%%%%%%%%%%
 \def\su{{\mathfrak{su}}}
 \def\gl{{\mathfrak{gl}}}
 \def\gotg{{\mathfrak{g}}}
 \def\goth{{\mathfrak{h}}}
 \def\gotm{{\mathfrak{m}}}
 \def\gotk{{\mathfrak{k}}}
 \def\spin{{\mathfrak{spin}}}
 \def\slC{{\mathfrak{sl}}}
 \def\O{{\mathfrak{O}}}
 \def\Set{{\mathfrak{Set}}}

%%%%%%%%%%%%%%%%%%%%%%%%%%%%%%
%%%%%%			Bbb			 %%%%%%
%%%%%%%%%%%%%%%%%%%%%%%%%%%%%%
 \def\one{\mathbb{I}}
 \def\A{\mathbb{A}}
 \def\B{\mathbb{B}}
 \def\C{\mathbb{C}}
 \def\D{\mathbb{D}}
 \def\E{\mathbb{E}}
 \def\F{\mathbb{F}}
 \def\G{\mathbb{G}}
 \def\H{\mathbb{H}}
 \def\J{\mathbb{J}}
 \def\K{\mathbb{K}}
 \def\I{\mathbb{I}}
 \def\L{\mathbb{L}}
 \def\M{\mathbb{M}}
 \def\N{\mathbb{N}}
 \def\O{\mathbb{O}}
 \def\P{\mathbb{P}}
 \def\Q{\mathbb{Q}}
 \def\R{\mathbb{R}}
 \def\S{\mathbb{S}}
 \def\T{\mathbb{T}}
 \def\U{\mathbb{U}}
 \def\V{\mathbb{V}}
 \def\X{\mathbb{X}}
 \def\Y{\mathbb{Y}}
 \def\W{\mathbb{W}}
 \def\Z{\mathbb{Z}}
 

%%%%%%%%%%%%%%%%%%%%%%%%%%%%%%
%%%%%%		MathRoman		 %%%%%%
%%%%%%%%%%%%%%%%%%%%%%%%%%%%%%
\def\Tr{{\hbox{Tr}}}
\def\Con{{\hbox{Con}}}
\def\Aut{{\hbox{Aut}}}
\def\Div{{\hbox{Div}}}
\def\ad{{\hbox{ad}}}
\def\Ad{{\hbox{Ad}}}
\def\Re{{\hbox{Re}}}
\def\Im{{\hbox{Im}}}
\def\Card{{\hbox{Card}}}
\def\Iso{{\hbox{Iso}}}
\def\Geo{{\hbox{Geo}}}
\def\Int{{\hbox{Int}}}
\def\Inv{{\hbox{Inv}}}
\def\Spin{{\hbox{Spin}}}
\def\SO{{\hbox{SO}}}
\def\SU{{\hbox{SU}}}
\def\SL{{\hbox{SL}}}
\def\GL{{\hbox{GL}}}
\def\det{{\hbox{det}}}
\def\Hom{{\hbox{Hom}}}
\def\End{{\hbox{End}}}
\def\Euc{{\hbox{Euc}}}
\def\Lor{{\hbox{Lor}}}
\def\Diff{{\hbox{Diff}}}
\def\di{{\hbox{d}}}
\def\id{{\hbox{\rm id}}}
\def\diag{{\hbox{diag}}}
\def\rank{{\hbox{rank}}}
\def\span{{\hbox{span}}}
\def\Sim{{\hbox{Sim}}}

\def\Obj{{\hbox{Obj}}}

%%%%%%%%%%%%%%%%%%%%%%%%%%%%%%
%%%%%%		OtherSymbols		 %%%%%%
%%%%%%%%%%%%%%%%%%%%%%%%%%%%%%
\def\ip{\hbox to4pt{\leaders\hrule height0.3pt\hfill}\vbox to8pt{\leaders\vrule width0.3pt\vfill}\kern 2pt}

% inner product
\def\QDE{\hfill\hbox{\ }\vrule height4pt width4pt depth0pt} 
\def\del{\partial}
\def\na{\nabla}
\def\inter{\cap}
\def\Vec{\mathfrak{X}}
\def\Lie{\hbox{\LieFont \$}}

\def\arr{\rightarrow}
\def\larr{\longrightarrow}
\def\harr{\hookrightarrow}
\def\hlarr{\lhook\joinrel\longrightarrow}
\def\then{\Rightarrow}
\def\semidirect{\hbox{\Bbb \char111}}
\def\binomial#1#2{\left(\Matrix{#1\cr #2\cr}\right)}

\def\barJ{\bar J}

\def\sSU{{\hbox{SU}}}
\def\so{{so}}
%\def\N{{{\mathbb N}}}

\def\file#1{{\tt #1}}
\def\calH{{{\cal H}}}
\def\calJ{{{\cal J}}}
\def\calA{{{\cal A}}}
\def\calV{{{\cal V}}}
\def\calG{{{\cal G}}}
\def\calP{{{\cal P}}}
\def\calK{{{\cal K}}}
\def\calD{{{\cal D}}}
\def\scalD{{{\cal D}}}

\def\Loop{{\hbox{Loop}}}
\def\Hoop{{\hbox{Hoop}}}
\def\Sym{{\hbox{Sym}}}
\def\sR{{\R}}
\def\dCor{dCor}
\def\epm{\hbox{$^\pm$}}
\def\dunion{\coprod}

\def\nab#1{{\buildrel #1\over \na}}
\def\frac[#1/#2]{\hbox{$#1\over#2$}}
\def\Frac[#1/#2]{{#1\over#2}}
\def\({\left(}
\def\){\right)}
\def\[{\left[}
\def\]{\right]}
\def\^#1{{}^{#1}_{\>\cdot}}
\def\_#1{{}_{#1}^{\>\cdot}}
\def\Label=#1{{\buildrel {\hbox{\fiveSerif \ShowLabel{#1}}}\over =}}
\def\<{\kern -1pt}
\def\Bar{\>|\>}
\def\Dal{\hbox{\tenRelazioni  \char003}}

\def\uvec#1{\vbox{\hbox{$\scriptstyle\rightharpoonup$}\vskip-9pt\hbox{$#1$}}}
\def\dvec#1{\vtop{\hbox{$#1$}\vskip-10pt\hbox{$\scriptstyle\rightharpoondown$}}}
\def\Cprod{\diamond}
\def\obullet{\odot}
\def\cbrt{{}^3\kern-5pt\sqrt}



%%%%%%%%%%%%			frames 				%%%%%%%%%%%%%%%%%%%

\def\red#1{{\color{red}{#1}}}
\def\blue#1{{\color{blue}{#1}}}
\def\green#1{{\color{green}{#1}}}

\def\Red{\color{red}}
\def\Blue{\color{blue}}
\def\Green{\color{green}}


\def\frame#1{\vbox{\hrule\hbox{\vrule\vbox{\kern2pt\hbox{\kern2pt#1\kern2pt}\kern2pt}\vrule}\hrule\kern-4pt}} 
\def\redframe#1{\red{\frame{#1}}} 
\def\greenframe#1{\green{\frame{#1}}} 
\def\blueframe#1{\blue{\frame{#1}}} 

\def\uline#1{\underline{#1}}
\def\uuline#1{\underline{\underline{#1}}}
\def\Box to #1#2#3{\frame{\vtop{\hbox to #1{\hfill #2 \hfill}\hbox to #1{\hfill #3 \hfill}}}}



\def\ubal{\underline{\al}\kern1pt}
\def\obal{\overline{\al}\kern1pt}

\def\ubR{\underline{R}\kern1pt}
\def\obR{\overline{R}\kern1pt}
\def\ubom{\underline{\om}\kern1pt}
\def\obxi{\overline{\xi}\kern1pt}
\def\ubu{\underline{u}\kern1pt}
\def\ube{\underline{e}\kern1pt}
\def\obe{\overline{e}\kern1pt}
\def\Limit{\>{\buildrel{r\arr\infty}\over \longrightarrow}\,}
\def\union{\cup}
\def\Emptyset{\varnothing}




\def\Uvec#1{\vbox{\mathsurround=0pt\ialign{##\crcr
     $\scriptscriptstyle\rightharpoonup$\crcr\noalign{\kern1pt\nointerlineskip}
     $\hfil\displaystyle{#1}\hfil$\crcr}}}
\def\Dvec#1{\vbox{\mathsurround=0pt\ialign{##\crcr
     $\scriptscriptstyle\rightharpoondown$\crcr\noalign{\kern-7pt\nointerlineskip}
     $\hfil\displaystyle{#1}\hfil$\crcr}}}


%   u 
\def\uvecu{\vbox{\mathsurround=0pt\ialign{##\crcr
     $\scriptscriptstyle\rightharpoonup$\crcr\noalign{\kern1pt\nointerlineskip}
     $\hfil\displaystyle{u}\hfil$\crcr}}}
\def\dvecu{\vbox{\mathsurround=0pt\ialign{##\crcr
     $\scriptscriptstyle\rightharpoondown$\crcr\noalign{\kern-7pt\nointerlineskip}
     $\hfil\displaystyle{u}\hfil$\crcr}}}
%\def\uvecu{\Uvec{u}}
%\def\dvecu{\Dvec{u}}

%   be
\def\uvecbe{\vbox{\mathsurround=0pt\ialign{##\crcr
     \kern3pt$\scriptscriptstyle\rightharpoonup$\crcr\noalign{\kern1pt\nointerlineskip}
     $\hfil\displaystyle{\be}\hfil$\crcr}}}
\def\dvecbe{\vbox{\mathsurround=0pt\ialign{##\crcr
     \kern1pt$\scriptscriptstyle\rightharpoondown$\crcr\noalign{\kern-10pt\nointerlineskip}
     $\hfil\displaystyle{\be}\hfil$\crcr}}}

%   n
\def\uvecn{\vbox{\mathsurround=0pt\ialign{##\crcr
     $\scriptscriptstyle\rightharpoonup$\crcr\noalign{\kern1pt\nointerlineskip}
     $\hfil\displaystyle{n}\hfil$\crcr}}}
\def\dvecn{\vbox{\mathsurround=0pt\ialign{##\crcr
     $\scriptscriptstyle\rightharpoondown$\crcr\noalign{\kern-7pt\nointerlineskip}
     $\hfil\displaystyle{n}\hfil$\crcr}}}

%   m
\def\uvecm{\vbox{\mathsurround=0pt\ialign{##\crcr
     $\scriptscriptstyle\rightharpoonup$\crcr\noalign{\kern1pt\nointerlineskip}
     $\hfil\displaystyle{m}\hfil$\crcr}}}
\def\dvecm{\vbox{\mathsurround=0pt\ialign{##\crcr
     $\scriptscriptstyle\rightharpoondown$\crcr\noalign{\kern-7pt\nointerlineskip}
     $\hfil\displaystyle{m}\hfil$\crcr}}}

%   N
\def\uvecN{\vbox{\mathsurround=0pt\ialign{##\crcr
     \kern3pt$\scriptscriptstyle\rightharpoonup$\crcr\noalign{\kern1pt\nointerlineskip}
     $\hfil\displaystyle{N}\hfil$\crcr}}}
\def\dvecN{\vbox{\mathsurround=0pt\ialign{##\crcr
     \kern0pt$\scriptscriptstyle\rightharpoondown$\crcr\noalign{\kern-10pt\nointerlineskip}
     $\hfil\displaystyle{N}\hfil$\crcr}}}

%   u
\def\uvecu{\vbox{\mathsurround=0pt\ialign{##\crcr
     $\scriptscriptstyle\rightharpoonup$\crcr\noalign{\kern1pt\nointerlineskip}
     $\hfil\displaystyle{u}\hfil$\crcr}}}
\def\dvecu{\vbox{\mathsurround=0pt\ialign{##\crcr
     $\scriptscriptstyle\rightharpoondown$\crcr\noalign{\kern-7pt\nointerlineskip}
     $\hfil\displaystyle{u}\hfil$\crcr}}}

%   w
\def\uvecw{\vbox{\mathsurround=0pt\ialign{##\crcr
     $\scriptscriptstyle\rightharpoonup$\crcr\noalign{\kern1pt\nointerlineskip}
     $\hfil\displaystyle{w}\hfil$\crcr}}}
\def\dvecw{\vbox{\mathsurround=0pt\ialign{##\crcr
     $\scriptscriptstyle\rightharpoondown$\crcr\noalign{\kern-7pt\nointerlineskip}
     $\hfil\displaystyle{w}\hfil$\crcr}}}

%   v
\def\uvecv{\vbox{\mathsurround=0pt\ialign{##\crcr
     $\scriptscriptstyle\rightharpoonup$\crcr\noalign{\kern1pt\nointerlineskip}
     $\hfil\displaystyle{v}\hfil$\crcr}}}
\def\dvecv{\vbox{\mathsurround=0pt\ialign{##\crcr
     $\scriptscriptstyle\rightharpoondown$\crcr\noalign{\kern-7pt\nointerlineskip}
     $\hfil\displaystyle{v}\hfil$\crcr}}}

\def\astA{{}^\ast A}
\def\circA{{}^\circ A}
\def\astk{{}^\ast k}
\def\circk{{}^\circ k}
\def\astK{{}^\ast K}
\def\circK{{}^\circ K}
\def\astL{{}^\ast L}
\def\circL{{}^\circ L}
\def\astal{{}^\ast \al}
\def\circal{{}^\circ \al}
\def\astsi{{}^\ast \si}
\def\circsi{{}^\circ \si}
\def\aste{{}^\ast e}
\def\circe{{}^\circ e}
\def\astte{{}^\ast \te}
\def\circte{{}^\circ \te}
\def\astGa{{}^\ast \Ga}
\def\circGa{{}^\circ \Ga}
\def\Lie{\pounds}

\long\def\Hide#1{}
\long\def\HideMarked#1{\hfill{$\triangleright$}}
\def\rtau{\tau}


%%%%%%%%%%%%%%%%%%%%%%%%%%%%%%%%%%%%%%%%%%%%%%%%%%%%%%%%%%%%%%%
%\def\NormalStyle{\leftskip=2cm\rightskip=0cm\normalsize\parindent=5pt\parskip=3pt\normalbaselineskip=14pt\baselineskip=\normalbaselineskip}
%\def\AbstractStyle{\leftskip=3cm\rightskip=1cm\scriptsize\parindent=0pt\parskip=0pt\normalbaselineskip=11pt\baselineskip=\normalbaselineskip}
%\def\NoteStyle{\leftskip=3cm\rightskip=1cm\scriptsize\parindent=0pt\parskip=0pt\normalbaselineskip=11pt\baselineskip=\normalbaselineskip}

\def\ShowLabel#1{\ref{#1}}

\def\Bibliography{\begin{thebibliography}{199}\footnotesize\References}
\def\EndBibliography{\end{thebibliography}}
\def\bib#1#2{\bibitem{#1}#2}
\def\Ref#1{\cite{#1}}


\def\NewSection#1{\section{#1}}
\def\NewSubSection#1{\subsection*{#1}}
\def\NewAppendix#1#2{\section*{Appendix #1: #2}}
\def\Acknowledgements{\section*{Acknowledgements}}
\def\bs{\bigskip}
\def\ms{\medskip}
\def\ss{\smallskip}
\def\ni{\noindent}

\long\def\Note#1{\blockquote{\footnotesize {\bf Nota:}~#1\par}}
\long\def\Ex#1{\blockquote{\footnotesize {\bf Esercizio:}~#1\par}}

\def\eq#1{\begin{equation}#1\end{equation}}
\def\eqLabel#1#2{\begin{equation}#1\label{#2}\end{equation}}


\def\Cases#1{\begin{cases}#1\end{cases}}
\def\Matrix#1{\begin{matrix}#1\end{matrix}}
\def\Align#1{\begin{aligned}#1\end{aligned}}

\def\eqs#1{\eq{\Align{#1}}}
\def\eqsLabel#1#2{\eq{\Align{#1}\label{#2}}}




\def\Abstract{\AbstractStyle{\bf Abstract. }}
\def\EndAbstract{\par\NormalStyle}

\def\TitleScript{}
\def\TitleLine#1{\def\TitleScript{#1}}
\def\MoreTitleLine#1{\edef\TitleScript{\TitleScript\\#1}}

\def\AuthorScript{}
\def\AuthorLine#1{\def\AuthorScript{#1}}
\def\MoreAuthorLine#1{\edef\AuthorScript{\AuthorScript\\#1}}

\def\AddressScript{}
\def\AddressLine#1{\def\AddressScript{#1}}
\def\MoreAddressLine#1{\edef\AddressScript{\AddressScript\\#1}}

\date{}

\def\BeginDocument{\begin{document}}

\def\MakeTitle{
\begin{center}
{\Large\bf\sffamily\baselineskip=2pt\TitleScript}\\
\vskip 10pt
{\small by {\it \AuthorScript}}\\
\vskip 10pt
{\small \AddressScript}
\end{center}
\ms
}

\def\EndDocument{\end{document}}

\def\Figure[#1]#2{\begin{figure}[htbp] %  figure placement: here, top, bottom, or page
   \centering
   \includegraphics[#1]{#2} }
\def\Caption#1{\caption{#1}}
\def\EndFigure{\end{figure}}

\def\Diagram#1{\eq{
\begindc{\commdiag}[10]
#1
\enddc%
}%
}

\def\Itemize#1{\begin{itemize}#1\end{itemize}}
\def\Item[#1]{\item[#1]}

\def\AllReferences{}



\def\Shrink{\small}
%%%%%%%%%%%%%%%%%%%%%%%%%%%%%%%%%%%%%%%%%%%%%%%%%
%%%%%%%%%%%%%%%%%%%%%%%%%%%%%%%%%%%%%%%%%%%%%%%%%%%

%\BeginDocument


%%%%%%%%%%%%%%%%%%%%%%%%%%%%%%%%%%%%%%%%%%%%%%%%%%%






















\parindent=5pt
\baselineskip=13pt
\parskip=5pt





\begin{document}

\AllReferences


%%%%%%%%%%%%%%%%%%%%% Publisher's Area please ignore %%%%%%%%%%%%%%%
%  Cari polli, volevo vedere se quando leggete controllate i conti o bevete qualunque boiata io scriva.
%   questo è quello giusto!
%%%%%%%%%%%%%%%%%%%%%%%%%%%%%%%%%%%%%%%%%%%%%%%%%%%%%%%%%%%%%%%%%%%%

\section*{Giorno 26: Zenone e paradossi}

Zenone è filosofo greco del quinto secolo ac.
Noto soprattutto per i paradossi omonimi atti a dimostrare l'impossibilità del moto.
Questi argomenti sono interessanti da diversi punti di vista. Mostrano che non c'era ancora una distinzione netta tra verità logica e verità fisica, oltre che i greci non sapevano contare davvero.
Hanno avuto una buona comprensione delle frazioni, un'ottima geometria, ma poca algebra, una pessima comprensione degli infiniti, per non parlare di zero e numeri negativi.
Oh, mica gli se ne fa una colpa, è solo evidenza che a quel tempo mancavano ancora gli strumenti per affrontare decentemente questi argomenti.

Non credo sia essenziale analizzare ognuno dei paradossi, ce ne basta uno, diciamo quello noto come {\it il paradosso di Achille e la tartaruga}.
Zenone argomenta che se Achille (noto velocista dell'antichità) venisse sfidato alla corsa da una tartaruga (che è considerata una velocista solo da Bruno Lauzi), e Achille divertito concedesse un certo vantaggio alla tartaruga, diciamo $10$ metri (pollici, passi, cubiti, \dots) allora non potrebbe mai raggiungere la tartaruga.

L'argomento va così: per correre i primi 10 metri Achille impiegherebbe un tempo $t_1$ e un questo tempo la tartaruga percorrerebbe uno spazio $s_1$.
Poi Achille impiegherebbe un tempo $t_2$ per percorrere lo spazio $s_1$ che lo separa dalle tartaruga, ma in quel tempo  la tartaruga percorrerebbe uno spazio $s_2$.
Quindi Achille impiegherebbe un tempo $t_3$ per percorrere lo spazio $s_2$ che lo separa dalle tartaruga, ma in quel tempo  la tartaruga percorrerebbe uno spazio $s_3$.
E via così.
Quindi il tempo impiegato da Achille per solo raggiungere la tartaruga sarebbe $T=t_1+t_2+t_3+\dots$
che sarebbe infinito (perché è evidente che la somma di infiniti numeri è infinita per Zenone) e quindi possiamo dire che Achille non raggiungerebbe mai la tartaruga, tanto meno riuscirebbe a superarla.
Quindi, conclude Zenone, nulla può muoversi!

\Note{Detto così il ragionamento è un po' balzano, ma è preciso rispetto agli argomenti che ci servono dopo.
Zenone comunque era meno cretino di quanto si potrebbe pensare. I suoi argomenti hanno ispirato Democrito e alla fine hanno portato agli atomisti che in un certo senso hanno descritto un mondo più simile a come lo descriviamo noi oggi rispetto a quello degli altri filosofi.

Anche se alla fine noi dovremmo diffidare da questi argomenti logici che pretendono di dimostrare logicamente qualcosa di reale senza riconoscere che la logica e la fisica parlan0o di 2 cose diverse. Non riconoscere questa separazione è una sorta di monoteismo logico che, come tutti i monoteismi, finisce per avere torto marcio.}

Ma qui a noi basta molto meno. Facciamo un modello più preciso. Achille corre a velocità $w$ e la tartaruga a velocità $v<w$.
abbiamo che $t_1= \frac[s_1/ w]$ e  la tartaruga in questo tempo percorre $s_2= vt_1= \frac[vs_1/ w]$.
Quindi Achille percorre lo spazio $s_2$ in un tempo $t_2=  \frac[s_2/ w] = \frac[vs_1/ w^2]$ e e  la tartaruga in questo tempo percorre $s_3= vt_2= \frac[v^2s_1/ w^2]$.
E via così $t_3=   \frac[v^2s_1/ w^3]$ e $s_4= \frac[v^3s_1/ w^3]$,
$t_4=   \frac[v^3s_1/ w^4]$ e $s_5= \frac[v^4s_1/ w^4]$.
Quindi il tempo impiegato da Achille a raggiungere la tartaruga sarebbe veramente somma di infiniti contributi
\eqs{
T=& \Frac[s_1/ w]+ \Frac[vs_1/ w^2] +  \Frac[v^2s_1/ w^3] + \Frac[v^3s_1/ w^4]+\dots + \Frac[v^{k-1}s_1/ w^k] +\dots =\cr
=&  \Frac[s_1/w] \( 1+ \Frac[v/ w] + \Frac[v^2/ w^2] +  \Frac[v^3/ w^3]+\dots + \Frac[v^{k}/ w^k] +\dots \)
}

Quello che non sta né in cielo né in terra è la conclusione che la somma di infiniti contributi sia necessariamente infinita, cosa talmente evidente che Zenone non ci spende più di una parola.

\Note{Mettiamo da parte un po' di algebra. 
Consideriamo la quantità $1-z^{k+1}$ che si annulla per $z=1$, quindi il polinomio $1-z^{k+1}$ si può scrivere come il prodotto di $(1-z)$ per un polinomio di grado $k$. 
Controllate che 
\eq{
\(1-z^{k+1}\)= (1-z)(1+z+z^2+\dots+ z^k)
}
come si vede sviluppando il prodotto. Qui non ci sono infiniti, solo $k$ termini con  $k$ qualunque ma finito.

Quindi possiamo scrivere
\eq{
\( 1+ \Frac[v/ w] + \Frac[v^2/ w^2] +  \Frac[v^3/ w^3]+\dots + \Frac[v^{k}/ w^k] \) = \Frac[{1-\(\frac[v/w]\)^{k+1}}/{1-\frac[v/w]}]
}
e se assumiamo che Achille abbia una velocità maggiore di quella della tartaruga ($w>v$) allora $0<z= \frac[v/w] <1$.
}

Quindi possiamo scrivere il tempo impiegato a percorrere $k$ passi come
\eq{
T_k= t_1+ t_2+\dots+t_k 
=  \Frac[s_1/w] \Frac[{1-z^{k+1}}/{1-z}]
}
Intanto vediamo che per ogni $k$ finito, $T_k$ è finito, siccome $z<1$ il denominatore $1-z^{k+1}$ non si annulla mai.

Ora il punto è mostrare che anche per $k$ {\it che tende ad infinito} $T_k$ resta finito.

\Note{Poniamo $z=\frac[v/w]<1$ e
mostriamo che per ogni $k$ abbiamo che  $T_k< 2\Frac[s_1/w] $.
\eqs{
& \Frac[s_1/w] \( 1+ z+ z^2 +  z^3+\dots + z^k\)  < 2\Frac[s_1/w] \cr
&1<1+ z + z^2+  z^3+\dots + z^k  < 2 \cr
&1<\Frac[{1-z^k}/1-z]< 2 \cr
}
Per la disuguaglianza a sinistra
\eq{
\Frac[{1-z^k}/1-z]-1= \Frac[{1-z^k-1+z}/1-z]=z\Frac[{1-z^{k-1}}/1-z]>0
}
che è verificata in quanto $1-z>0$, $z>0$ e $1>z>z^{k-1}$ (e quindi $1-z^{k-1}>0$).

Per la disuguaglianza a destra abbiamo
\eq{
\Frac[{1-z^k}/1-z]-2 
=\Frac[{1-z^k}-2+2z^k/1-z] 
=-\Frac[{1-z^k}/1-z] 
< 0
}
Quindi è vero che per ogni $k\in \N$ abbiamo
\eq{
1<\Frac[{1-z^k}/1-z]< 2 
}
}


Ora possiamo dire che, visto che $T_k<2$ per ogni $k$, è un po' difficile credere che la somma degli infiniti contributi $T=\sum_{k=1}^\infty t_k$
sia infinito.

E infatti non lo è, se diamo le definizioni e creiamo un contesto in cui sappiamo dimostrare qualcosa di queste somme infinite.
Cosa che facciamo tra un po'.

Va detto che questi argomenti non erano del tutto sconosciuti ai greci. Archimede (terzo secolo ac) faceva qualcosa tipo quello che abbiamo fatto qui sopra, con dimostrazioni geometriche
(ma consentendo di sommare infinite quantità). Ma diciamo che questo tipo di ragionamenti non erano condivisi da tutti a quei tempi, non rientrano in Euclide (che consentiva solo numeri finiti).

Da un punto di vista moderno, questo tipo di ragionamenti, che Archimede chiamava per {\it esaustione}, sono equivalenti alla nostra analisi e calcolo differenziale (che però sono di 1800 anni dopo). 

I famosi teoremi di Archimede sul galleggiamento e stabilità delle coppe in acqua sono qualcosa che noi facciamo facendo qualche integrale che infatti è definito (da Riemann) come somma di infiniti contributi infinitesimi.

Tra l'altro se guardate la \http{https://maddmaths.simai.eu/news-2/due-liceali-americane-hanno-trovato-una-nuova-dimostrazione-del-teorema-di-pitagora/}{notizia}
e provate a ripercorrere la dimostrazione, vedrete che non si tratta di una dimostrazione nel senso di Euclide, ma contiene un processo di esaustione come avrebbe fatto Archimede.

\ms
Prima di procedere con la teoria generale diamo un altro esempio. Considerate la somma infinita
\eq{
s=1+ \Frac[1/2]+ \Frac[1/4] +\dots  + \Frac[1/2^k] + \dots
}
e leggetela così. 
Camminate per una misura 1, poi per una misura $\frac[1/2]$ (cioè $\frac[1/2]$ meno di quello che serve per arrivare a 2),
 poi per una misura $\frac[1/4]$ (cioè $\frac[1/4]$ meno di quello che serve per arrivare a 2),
 poi per una misura $\frac[1/8]$ (cioè $\frac[1/8]$ meno di quello che serve per arrivare a 2),
 e via così.

Così è più evidente che al passo $k$ manca $\frac[1/2^k]$ per arrivare a 2 e quindi non raggiungete mai $2$, ci si avvicina indefinitamente a 2 sempre percorrendo 
ad ogni passo metà dello spazio quello che manca per arrivare a $2$.

Come nel caso precedente, questo mostra che è ragionevole asepttarsi che in qualche senso da precisare la somma di infiniti numeri sempre più piccoli possa restare finito.

\EndDocument
\end


