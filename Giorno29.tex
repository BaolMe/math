% !TEX TS-program = latex
% !TEX spellcheck = it-IT
\documentclass[a4paper]{article}
\usepackage[utf8]{inputenc}
\usepackage[T1]{fontenc}
\usepackage[italian]{babel}
\usepackage{tikz} % LATEX
\usetikzlibrary {automata}


%%%%%%%%%%%%%%%%%%%%%%%%%%%%%%%%%%%%%%%%%%%%%%%%%%%%%%%%%%%%%%%%%%%%%%%%%%%%
%% Trim Size: 9.75in x 6.5in
%% Text Area: 8in (include Runningheads) x 5in
%% ws-ijgmmp.tex   :   2-9-08
%% Tex file to use with ws-ijgmmp.cls written in Latex2E.
%% The content, structure, format and layout of this style file is the
%% property of World Scientific Publishing Co. Pte. Ltd.
%% Copyright 1995, 2002 by World Scientific Publishing Co.
%% All rights are reserved.
%%%%%%%%%%%%%%%%%%%%%%%%%%%%%%%%%%%%%%%%%%%%%%%%%%%%%%%%%%%%%%%%%%%%%%%%%%%%
%%

%%%%%%%%%%%%%%%%%%%%%%%%%%%%%


\usepackage{pictexwd,dcpic}
\usepackage{csquotes}
\usepackage{nopageno}

\usepackage{amsmath, amssymb, amsthm}
\usepackage{pictex, dcpic}
\usepackage{color}
%\usepackage{graphicx}
%\numberwithin{equation}{section} % This line resets equation numbering when starting a new section.
%\renewcommand{\theequation}{Eq. \thesection.\arabic{equation}} % This line ads "Eq." in front of your equation numbering.




%
%    Macros.    Version 1.2.0.beta
%    The best use is to paste all of them into the papers
%     1/8/2005
%

%%%%%%%%%%%%%%%%%%%%%%%%%%%%%%
%%%%%%			Greek		 %%%%%%
%%%%%%%%%%%%%%%%%%%%%%%%%%%%%%

\def\al{\alpha}
\def\be{\beta}
\def\de{\delta}
\def\ga{\gamma}
\def\up{\upsilon}
\def\ep{\epsilon}
\def\io{\iota}
\def\te{\theta}
\def\la{\lambda}
\def\ze{\zeta}
\def\om{\omega}
\def\si{\sigma}
\def\vp{\varphi}
\def\vpi{\varpi}
\def\ka{\kappa}
\def\vs{\varsigma}
\def\vr{\varrho}

\def\De{\Delta}
\def\Ga{\Gamma}
\def\Te{\Theta}
\def\La{\Lambda}
\def\Om{\Omega}
\def\Si{\Sigma}
\def\Up{\Upsilon}

\def\boldDe{\bf\Delta}
\def\boldGa{\bf\Gamma}
\def\boldTe{\bf\Theta}
\def\boldLa{\bf\Lambda}
\def\boldOm{\bf\Omega}
\def\boldSi{\bf\Sigma}
\def\boldUp{\bf\Upsilon}
\def\boldXi{\bf\Xi}
\def\boldPi{\bf\Pi}
\def\boldPhi{\bf\Phi}
\def\boldPsi{\bf\Psi}

\def\boldal{\bf\al}
\def\boldbe{\bf\be}
\def\boldga{\bf\ga}
\def\boldde{\bf\de}
\def\boldep{\bf\ep}
\def\boldze{\bf\ze}
\def\boldeta{\bf\eta}
\def\boldte{\bf\te}
\def\boldio{\bf\io}
\def\boldka{\bf\ka}
\def\boldla{\bf\la}
\def\boldmu{\hbox{\greekbold\char "16}}
\def\boldnu{\hbox{\greekbold\char "17}}
\def\boldxi{\hbox{\greekbold\char "18}}
\def\boldpi{\hbox{\greekbold\char "19}}
\def\boldrho{\hbox{\greekbold\char "1A}}
\def\boldsi{\bf\si}
\def\boldtau{\hbox{\greekbold\char "1C}}
\def\boldup{\hbox{\greekbold\char "1D}}
\def\boldphi{\hbox{\greekbold\char "1E}}
\def\boldchi{\hbox{\greekbold\char "1F}}
\def\boldpsi{\hbox{\greekbold\char "20}}
\def\boldom{\hbox{\greekbold\char "21}}
\def\boldvep{\hbox{\greekbold\char "22}}
\def\boldvte{\hbox{\greekbold\char "23}}
\def\boldvpi{\hbox{\greekbold\char "24}}
\def\boldvrho{\hbox{\greekbold\char "25}}
\def\boldvsi{\hbox{\greekbold\char "26}}
\def\boldvphi{\hbox{\greekbold\char "27}}

%%%%%%%%%%%%%%%%%%%%%%%%%%%%%%
%%%%%%			Cal			 %%%%%%
%%%%%%%%%%%%%%%%%%%%%%%%%%%%%%
 \def\calA{{\hbox{\cal A}}}
 \def\calU{{\hbox{\cal U}}}
 \def\calB{{\hbox{\cal B}}}
 \def\calC{{\hbox{\cal C}}}
 \def\calI{{\hbox{\cal I}}}
 \def\calQ{{\hbox{\cal Q}}}
 \def\calP{{\hbox{\cal P}}}
 \def\calL{{\hbox{\cal L}}}
 \def\calE{{\hbox{\cal E}}}
 \def\calW{{\hbox{\cal W}}}
 \def\calV{{\hbox{\cal V}}}
 \def\calS{{\hbox{\cal S}}}
 \def\calF{{\hbox{\cal F}}}
 \def\calR{{\hbox{\cal R}}}
 \def\calO{{\hbox{\cal O}}}
 \def\calM{{\hbox{\cal M}}}

%%%%%%%%%%%%%%%%%%%%%%%%%%%%%%
%%%%%%			gothic		 %%%%%%
%%%%%%%%%%%%%%%%%%%%%%%%%%%%%%
 \def\su{{\mathfrak{su}}}
 \def\gl{{\mathfrak{gl}}}
 \def\gotg{{\mathfrak{g}}}
 \def\goth{{\mathfrak{h}}}
 \def\gotm{{\mathfrak{m}}}
 \def\gotk{{\mathfrak{k}}}
 \def\spin{{\mathfrak{spin}}}
 \def\slC{{\mathfrak{sl}}}
 \def\O{{\mathfrak{O}}}
 \def\Set{{\mathfrak{Set}}}

%%%%%%%%%%%%%%%%%%%%%%%%%%%%%%
%%%%%%			Bbb			 %%%%%%
%%%%%%%%%%%%%%%%%%%%%%%%%%%%%%
 \def\one{\mathbb{I}}
 \def\A{\mathbb{A}}
 \def\B{\mathbb{B}}
 \def\C{\mathbb{C}}
 \def\D{\mathbb{D}}
 \def\E{\mathbb{E}}
 \def\F{\mathbb{F}}
 \def\G{\mathbb{G}}
 \def\H{\mathbb{H}}
 \def\J{\mathbb{J}}
 \def\K{\mathbb{K}}
 \def\I{\mathbb{I}}
 \def\L{\mathbb{L}}
 \def\M{\mathbb{M}}
 \def\N{\mathbb{N}}
 \def\O{\mathbb{O}}
 \def\P{\mathbb{P}}
 \def\Q{\mathbb{Q}}
 \def\R{\mathbb{R}}
 \def\S{\mathbb{S}}
 \def\T{\mathbb{T}}
 \def\U{\mathbb{U}}
 \def\V{\mathbb{V}}
 \def\X{\mathbb{X}}
 \def\Y{\mathbb{Y}}
 \def\W{\mathbb{W}}
 \def\Z{\mathbb{Z}}
 

%%%%%%%%%%%%%%%%%%%%%%%%%%%%%%
%%%%%%		MathRoman		 %%%%%%
%%%%%%%%%%%%%%%%%%%%%%%%%%%%%%
\def\Tr{{\hbox{Tr}}}
\def\Con{{\hbox{Con}}}
\def\Aut{{\hbox{Aut}}}
\def\Div{{\hbox{Div}}}
\def\ad{{\hbox{ad}}}
\def\Ad{{\hbox{Ad}}}
\def\Re{{\hbox{Re}}}
\def\Im{{\hbox{Im}}}
\def\Card{{\hbox{Card}}}
\def\Iso{{\hbox{Iso}}}
\def\Geo{{\hbox{Geo}}}
\def\Int{{\hbox{Int}}}
\def\Inv{{\hbox{Inv}}}
\def\Spin{{\hbox{Spin}}}
\def\SO{{\hbox{SO}}}
\def\SU{{\hbox{SU}}}
\def\SL{{\hbox{SL}}}
\def\GL{{\hbox{GL}}}
\def\det{{\hbox{det}}}
\def\Hom{{\hbox{Hom}}}
\def\End{{\hbox{End}}}
\def\Euc{{\hbox{Euc}}}
\def\Lor{{\hbox{Lor}}}
\def\Diff{{\hbox{Diff}}}
\def\di{{\hbox{d}}}
\def\id{{\hbox{\rm id}}}
\def\diag{{\hbox{diag}}}
\def\rank{{\hbox{rank}}}
\def\span{{\hbox{span}}}
\def\Sim{{\hbox{Sim}}}

\def\Obj{{\hbox{Obj}}}

%%%%%%%%%%%%%%%%%%%%%%%%%%%%%%
%%%%%%		OtherSymbols		 %%%%%%
%%%%%%%%%%%%%%%%%%%%%%%%%%%%%%
\def\ip{\hbox to4pt{\leaders\hrule height0.3pt\hfill}\vbox to8pt{\leaders\vrule width0.3pt\vfill}\kern 2pt}

% inner product
\def\QDE{\hfill\hbox{\ }\vrule height4pt width4pt depth0pt} 
\def\del{\partial}
\def\na{\nabla}
\def\inter{\cap}
\def\Vec{\mathfrak{X}}
\def\Lie{\hbox{\LieFont \$}}

\def\arr{\rightarrow}
\def\larr{\longrightarrow}
\def\harr{\hookrightarrow}
\def\hlarr{\lhook\joinrel\longrightarrow}
\def\then{\Rightarrow}
\def\semidirect{\hbox{\Bbb \char111}}
\def\binomial#1#2{\left(\Matrix{#1\cr #2\cr}\right)}

\def\barJ{\bar J}

\def\sSU{{\hbox{SU}}}
\def\so{{so}}
%\def\N{{{\mathbb N}}}

\def\file#1{{\tt #1}}
\def\calH{{{\cal H}}}
\def\calJ{{{\cal J}}}
\def\calA{{{\cal A}}}
\def\calV{{{\cal V}}}
\def\calG{{{\cal G}}}
\def\calP{{{\cal P}}}
\def\calK{{{\cal K}}}
\def\calD{{{\cal D}}}
\def\scalD{{{\cal D}}}

\def\Loop{{\hbox{Loop}}}
\def\Hoop{{\hbox{Hoop}}}
\def\Sym{{\hbox{Sym}}}
\def\sR{{\R}}
\def\dCor{dCor}
\def\epm{\hbox{$^\pm$}}
\def\dunion{\coprod}

\def\nab#1{{\buildrel #1\over \na}}
\def\frac[#1/#2]{\hbox{$#1\over#2$}}
\def\Frac[#1/#2]{{#1\over#2}}
\def\({\left(}
\def\){\right)}
\def\[{\left[}
\def\]{\right]}
\def\^#1{{}^{#1}_{\>\cdot}}
\def\_#1{{}_{#1}^{\>\cdot}}
\def\Label=#1{{\buildrel {\hbox{\fiveSerif \ShowLabel{#1}}}\over =}}
\def\<{\kern -1pt}
\def\Bar{\>|\>}
\def\Dal{\hbox{\tenRelazioni  \char003}}

\def\uvec#1{\vbox{\hbox{$\scriptstyle\rightharpoonup$}\vskip-9pt\hbox{$#1$}}}
\def\dvec#1{\vtop{\hbox{$#1$}\vskip-10pt\hbox{$\scriptstyle\rightharpoondown$}}}
\def\Cprod{\diamond}
\def\obullet{\odot}
\def\cbrt{{}^3\kern-5pt\sqrt}



%%%%%%%%%%%%			frames 				%%%%%%%%%%%%%%%%%%%

\def\red#1{{\color{red}{#1}}}
\def\blue#1{{\color{blue}{#1}}}
\def\green#1{{\color{green}{#1}}}

\def\Red{\color{red}}
\def\Blue{\color{blue}}
\def\Green{\color{green}}


\def\frame#1{\vbox{\hrule\hbox{\vrule\vbox{\kern2pt\hbox{\kern2pt#1\kern2pt}\kern2pt}\vrule}\hrule\kern-4pt}} 
\def\redframe#1{\red{\frame{#1}}} 
\def\greenframe#1{\green{\frame{#1}}} 
\def\blueframe#1{\blue{\frame{#1}}} 

\def\uline#1{\underline{#1}}
\def\uuline#1{\underline{\underline{#1}}}
\def\Box to #1#2#3{\frame{\vtop{\hbox to #1{\hfill #2 \hfill}\hbox to #1{\hfill #3 \hfill}}}}



\def\ubal{\underline{\al}\kern1pt}
\def\obal{\overline{\al}\kern1pt}

\def\ubR{\underline{R}\kern1pt}
\def\obR{\overline{R}\kern1pt}
\def\ubom{\underline{\om}\kern1pt}
\def\obxi{\overline{\xi}\kern1pt}
\def\ubu{\underline{u}\kern1pt}
\def\ube{\underline{e}\kern1pt}
\def\obe{\overline{e}\kern1pt}
\def\Limit{\>{\buildrel{r\arr\infty}\over \longrightarrow}\,}
\def\union{\cup}
\def\Emptyset{\varnothing}




\def\Uvec#1{\vbox{\mathsurround=0pt\ialign{##\crcr
     $\scriptscriptstyle\rightharpoonup$\crcr\noalign{\kern1pt\nointerlineskip}
     $\hfil\displaystyle{#1}\hfil$\crcr}}}
\def\Dvec#1{\vbox{\mathsurround=0pt\ialign{##\crcr
     $\scriptscriptstyle\rightharpoondown$\crcr\noalign{\kern-7pt\nointerlineskip}
     $\hfil\displaystyle{#1}\hfil$\crcr}}}


%   u 
\def\uvecu{\vbox{\mathsurround=0pt\ialign{##\crcr
     $\scriptscriptstyle\rightharpoonup$\crcr\noalign{\kern1pt\nointerlineskip}
     $\hfil\displaystyle{u}\hfil$\crcr}}}
\def\dvecu{\vbox{\mathsurround=0pt\ialign{##\crcr
     $\scriptscriptstyle\rightharpoondown$\crcr\noalign{\kern-7pt\nointerlineskip}
     $\hfil\displaystyle{u}\hfil$\crcr}}}
%\def\uvecu{\Uvec{u}}
%\def\dvecu{\Dvec{u}}

%   be
\def\uvecbe{\vbox{\mathsurround=0pt\ialign{##\crcr
     \kern3pt$\scriptscriptstyle\rightharpoonup$\crcr\noalign{\kern1pt\nointerlineskip}
     $\hfil\displaystyle{\be}\hfil$\crcr}}}
\def\dvecbe{\vbox{\mathsurround=0pt\ialign{##\crcr
     \kern1pt$\scriptscriptstyle\rightharpoondown$\crcr\noalign{\kern-10pt\nointerlineskip}
     $\hfil\displaystyle{\be}\hfil$\crcr}}}

%   n
\def\uvecn{\vbox{\mathsurround=0pt\ialign{##\crcr
     $\scriptscriptstyle\rightharpoonup$\crcr\noalign{\kern1pt\nointerlineskip}
     $\hfil\displaystyle{n}\hfil$\crcr}}}
\def\dvecn{\vbox{\mathsurround=0pt\ialign{##\crcr
     $\scriptscriptstyle\rightharpoondown$\crcr\noalign{\kern-7pt\nointerlineskip}
     $\hfil\displaystyle{n}\hfil$\crcr}}}

%   m
\def\uvecm{\vbox{\mathsurround=0pt\ialign{##\crcr
     $\scriptscriptstyle\rightharpoonup$\crcr\noalign{\kern1pt\nointerlineskip}
     $\hfil\displaystyle{m}\hfil$\crcr}}}
\def\dvecm{\vbox{\mathsurround=0pt\ialign{##\crcr
     $\scriptscriptstyle\rightharpoondown$\crcr\noalign{\kern-7pt\nointerlineskip}
     $\hfil\displaystyle{m}\hfil$\crcr}}}

%   N
\def\uvecN{\vbox{\mathsurround=0pt\ialign{##\crcr
     \kern3pt$\scriptscriptstyle\rightharpoonup$\crcr\noalign{\kern1pt\nointerlineskip}
     $\hfil\displaystyle{N}\hfil$\crcr}}}
\def\dvecN{\vbox{\mathsurround=0pt\ialign{##\crcr
     \kern0pt$\scriptscriptstyle\rightharpoondown$\crcr\noalign{\kern-10pt\nointerlineskip}
     $\hfil\displaystyle{N}\hfil$\crcr}}}

%   u
\def\uvecu{\vbox{\mathsurround=0pt\ialign{##\crcr
     $\scriptscriptstyle\rightharpoonup$\crcr\noalign{\kern1pt\nointerlineskip}
     $\hfil\displaystyle{u}\hfil$\crcr}}}
\def\dvecu{\vbox{\mathsurround=0pt\ialign{##\crcr
     $\scriptscriptstyle\rightharpoondown$\crcr\noalign{\kern-7pt\nointerlineskip}
     $\hfil\displaystyle{u}\hfil$\crcr}}}

%   w
\def\uvecw{\vbox{\mathsurround=0pt\ialign{##\crcr
     $\scriptscriptstyle\rightharpoonup$\crcr\noalign{\kern1pt\nointerlineskip}
     $\hfil\displaystyle{w}\hfil$\crcr}}}
\def\dvecw{\vbox{\mathsurround=0pt\ialign{##\crcr
     $\scriptscriptstyle\rightharpoondown$\crcr\noalign{\kern-7pt\nointerlineskip}
     $\hfil\displaystyle{w}\hfil$\crcr}}}

%   v
\def\uvecv{\vbox{\mathsurround=0pt\ialign{##\crcr
     $\scriptscriptstyle\rightharpoonup$\crcr\noalign{\kern1pt\nointerlineskip}
     $\hfil\displaystyle{v}\hfil$\crcr}}}
\def\dvecv{\vbox{\mathsurround=0pt\ialign{##\crcr
     $\scriptscriptstyle\rightharpoondown$\crcr\noalign{\kern-7pt\nointerlineskip}
     $\hfil\displaystyle{v}\hfil$\crcr}}}

\def\astA{{}^\ast A}
\def\circA{{}^\circ A}
\def\astk{{}^\ast k}
\def\circk{{}^\circ k}
\def\astK{{}^\ast K}
\def\circK{{}^\circ K}
\def\astL{{}^\ast L}
\def\circL{{}^\circ L}
\def\astal{{}^\ast \al}
\def\circal{{}^\circ \al}
\def\astsi{{}^\ast \si}
\def\circsi{{}^\circ \si}
\def\aste{{}^\ast e}
\def\circe{{}^\circ e}
\def\astte{{}^\ast \te}
\def\circte{{}^\circ \te}
\def\astGa{{}^\ast \Ga}
\def\circGa{{}^\circ \Ga}
\def\Lie{\pounds}

\long\def\Hide#1{}
\long\def\HideMarked#1{\hfill{$\triangleright$}}
\def\rtau{\tau}


%%%%%%%%%%%%%%%%%%%%%%%%%%%%%%%%%%%%%%%%%%%%%%%%%%%%%%%%%%%%%%%
%\def\NormalStyle{\leftskip=2cm\rightskip=0cm\normalsize\parindent=5pt\parskip=3pt\normalbaselineskip=14pt\baselineskip=\normalbaselineskip}
%\def\AbstractStyle{\leftskip=3cm\rightskip=1cm\scriptsize\parindent=0pt\parskip=0pt\normalbaselineskip=11pt\baselineskip=\normalbaselineskip}
%\def\NoteStyle{\leftskip=3cm\rightskip=1cm\scriptsize\parindent=0pt\parskip=0pt\normalbaselineskip=11pt\baselineskip=\normalbaselineskip}

\def\ShowLabel#1{\ref{#1}}

\def\Bibliography{\begin{thebibliography}{199}\footnotesize\References}
\def\EndBibliography{\end{thebibliography}}
\def\bib#1#2{\bibitem{#1}#2}
\def\Ref#1{\cite{#1}}


\def\NewSection#1{\section{#1}}
\def\NewSubSection#1{\subsection*{#1}}
\def\NewAppendix#1#2{\section*{Appendix #1: #2}}
\def\Acknowledgements{\section*{Acknowledgements}}
\def\bs{\bigskip}
\def\ms{\medskip}
\def\ss{\smallskip}
\def\ni{\noindent}

\long\def\Note#1{\blockquote{\footnotesize {\bf Nota:}~#1\par}}
\long\def\Ex#1{\blockquote{\footnotesize {\bf Esercizio:}~#1\par}}

\def\eq#1{\begin{equation}#1\end{equation}}
\def\eqLabel#1#2{\begin{equation}#1\label{#2}\end{equation}}


\def\Cases#1{\begin{cases}#1\end{cases}}
\def\Matrix#1{\begin{matrix}#1\end{matrix}}
\def\Align#1{\begin{aligned}#1\end{aligned}}

\def\eqs#1{\eq{\Align{#1}}}
\def\eqsLabel#1#2{\eq{\Align{#1}\label{#2}}}




\def\Abstract{\AbstractStyle{\bf Abstract. }}
\def\EndAbstract{\par\NormalStyle}

\def\TitleScript{}
\def\TitleLine#1{\def\TitleScript{#1}}
\def\MoreTitleLine#1{\edef\TitleScript{\TitleScript\\#1}}

\def\AuthorScript{}
\def\AuthorLine#1{\def\AuthorScript{#1}}
\def\MoreAuthorLine#1{\edef\AuthorScript{\AuthorScript\\#1}}

\def\AddressScript{}
\def\AddressLine#1{\def\AddressScript{#1}}
\def\MoreAddressLine#1{\edef\AddressScript{\AddressScript\\#1}}

\date{}

\def\BeginDocument{\begin{document}}

\def\MakeTitle{
\begin{center}
{\Large\bf\sffamily\baselineskip=2pt\TitleScript}\\
\vskip 10pt
{\small by {\it \AuthorScript}}\\
\vskip 10pt
{\small \AddressScript}
\end{center}
\ms
}

\def\EndDocument{\end{document}}

\def\Figure[#1]#2{\begin{figure}[htbp] %  figure placement: here, top, bottom, or page
   \centering
   \includegraphics[#1]{#2} }
\def\Caption#1{\caption{#1}}
\def\EndFigure{\end{figure}}

\def\Diagram#1{\eq{
\begindc{\commdiag}[10]
#1
\enddc%
}%
}

\def\Itemize#1{\begin{itemize}#1\end{itemize}}
\def\Item[#1]{\item[#1]}

\def\AllReferences{}



\def\Shrink{\small}
%%%%%%%%%%%%%%%%%%%%%%%%%%%%%%%%%%%%%%%%%%%%%%%%%
%%%%%%%%%%%%%%%%%%%%%%%%%%%%%%%%%%%%%%%%%%%%%%%%%%%

%\BeginDocument


%%%%%%%%%%%%%%%%%%%%%%%%%%%%%%%%%%%%%%%%%%%%%%%%%%%






















\parindent=5pt
\baselineskip=13pt
\parskip=5pt





\begin{document}

\AllReferences


%%%%%%%%%%%%%%%%%%%%% Publisher's Area please ignore %%%%%%%%%%%%%%%
%
%
%%%%%%%%%%%%%%%%%%%%%%%%%%%%%%%%%%%%%%%%%%%%%%%%%%%%%%%%%%%%%%%%%%%%

\section*{Giorno 29: sezioni di Dedekind}

Una {\it sezione di Dedekind} è un sottoinsieme $A\subset\Q$, $A$ non vuoto e non $A=\Q$, tale che se $a\in A$ allora ogni $a'<a$ è in $a'\in A$ pure lui
e, infine, $A$ non ha massimo, cioè non esiste un $a\in A$ tale che $a'\le a$ per ogni $a'\in A$.

Ogni sezione di Dedekind corrisponde ad un numero reale. 
Ad esempio $A=\{q\in \Q: q^2\le 2\}$ è una sezione di Dedekind e corrisponde al numero irrazionale $\sqrt{2}$.
Anche per ogni $p\in \Q$ possiamo definire $A_p=\{q\in \Q: q< p\}$ che è una sezione di Dedekind e corrisponde al numero razionale $\Q$.

Quindi definiamo $\R$ l'insieme delle sezioni di Dedekind.
Abbiamo $\Q\subset \R$, abbiamo che $r_1\le r_2$ se e solo se  $r_1\subset r_2$.

Date 2 sezioni di Dedekind $r_1$ e $r_2$ possiamo definire $A=\{q= q_1+q_2: q_1\in r_1, q_2\in r_2\}$ che è una sezione di Dedekind che chiamiamo $r_1+r_2$.
Questo definisce la somma di 2 numeri reali estendendo la somma dei razionali.

Date 2 sezioni di Dedekind $r_1$ e $r_2$ possiamo definire $A=\{q= q_1q_2: q_1\in r_1, q_2\in r_2\}$ che è una sezione di Dedekind che chiamiamo $r_1r_2$.
Questo definisce il prodotto di 2 numeri reali estendendo il prodotto dei razionali.

A questo punto possiamo estendere le potenze alle basi reali, ma soprattutto agli esponenti reali.
Prendiamo $r\in \R$, una successione $a:\N\arr \Q$ con $\lim_{n\arr\infty}=r$ e definiamo
\eq{
b^r= \lim_{n\arr \infty} b^{a_n}
} 
dove $b\in \R$ è un reale strettamente positivo.

Tutte queste operazioni ereditano per costruzione le proprietà delle operazioni in $\Q$.

Un insieme con 2 operazioni (somma e prodotto), entrambe associative, entrambe commutative, entrambe con elemento neutro ($0$ e $1$)
entrambe dotate di inverso (l'inverso di $r\in \R$ rispetto alla somma è l'opposto $-a$, rispetto al prodotto è il reciproco $\frac[1/a]$ con $a$ non nullo)
e col prodotto che è distributivo rispetto alla somma si chiama un {\it campo}.
Quindi $\Q$ e $\R$ sono campi. Siccome quando fate i conti (ad esempio risolvete un'equazione o semplificate una espressione) usate solo le proprietà elencate sopra,
risolvere un'equazione in $\Q$, in $\R$ o in qualunque altro campo è la medesima cosa.

 \Note{Se abbiamo un insieme con 2 operazioni con tutte le proprietà per essere un campo tranne che il prodotto non è commutativo, abbiamo un {\it corpo}.
 risolvere un'equazione in un corpo è quasi come risolverla in un campo ma bisogna mantenere l'ordine dei fattori nei prodotti.
 Ad esempio
 \eq{
 ax+b=0
\qquad
 ax=-b
\qquad
 x=-a^{-1}b
 }
 e non $x=-ba^{-1}$. Nei corpi non ci sono le frazioni, perché $ba^{-1}\not=a^{-1}b$ mentre entrambe sarebbero la frazione $\frac[b/a]$.
 }

Quindi ora possiamo aggiungere un pezzo di discorso al perché ci piace astrarre.
Abbiamo costruito i numeri razionali e i numeri reali, abbiamo definito le operazioni abbiamo {\it dimostrato} le proprietà delle operazioni, e con queste proprietà abbiamo imparato a risolvere delle classi di equazioni (ad esempio lineari e quadratiche). Questo è ciò che facciamo in un {\it modello}. 
In un certo senso è una costruzione concreta, sappiamo che un numero reale è una sezione di Dedekind (o una successione di Cauchy in $\Q$) e sappiamo cosa significa fare le somme e i prodotti.

\Note{Ok, i modelli possono non essere unici. Le successioni di Cauchy e le sezioni di Dedekind sono 2 modelli diversi dei numeri reali, ma le operazioni definite nei modelli si corrispondono e 
possiamo passare da una all'altra rappresentazione quando ci va. Un numero reale come successione è in realtà una  {\it classe} di successioni equivalenti, 
mentre lo stesso numero reale come sezione di Dedekind è {\it una} sezione di Dedekind.
Quindi può essere che vi piaccia di più il modello delle sezioni di Dedekind, che vi sembri più concreto.}

Anche per i naturali, gli interi e i razionali abbiamo usato modelli. Gli interi sono stati definiti come classi di insiemi finiti. 
Poi abbiamo definito gli interi come classi di coppie di naturali e i razionali come classi di coppie di interi.
Questo ha il vantaggio che se sapete cosa sono i naturali, quello che è definito con i naturali è ben definito comunque. Non dovete più capire gli insiemi per capire i razionali.

 Poi ad ogni livello possiamo {\it invece del modello} dare una descrizione assiomatica in cui non ci occupiamo più di cosa siano gli oggetti ma deriviamo (o assumiamo) le proprietà.
 Se valgono le proprietà allora sappiamo risolvere le equazioni. Il punto è che molte proprietà sono in comune tra numeri naturali, interi, razionali e reali.
 In particolare le proprietà di reali e razionali sono le stesse perché sono entrambi campi. E quindi io so risolvere le equazioni in qualunque campo.
 
 Quindi {\it conviene} definire un campo in astratto senza specificare se si parla di numeri reali o razionali (o complessi, o faremo $\Z_p$) perché nel momento stesso che so che è un campo
 so che posso risolvere le equazioni (o altri problemi) nello stesso modo. Gli algoritmi che sviluppiamo per calcolare, non dipendono da cosa siamo gli oggetti che manipoliamo, dipendono dalle proprietà delle operazioni che definiamo. 
 
 \Note{Se tutti i frutti rossi fossero velenosi non avrebbe tanto senso ricordare che ci sono mele rosse e mele gialle, invece di definire mele avrebbe più senso definire frutti rossi e frutti gialli.}

Dal punto di vista iniziale abbiamo che concreto è bello. All'inizio piacciono di sicuro di più i modelli che i sistemi assiomatici.
Ora la situazione, sotto un'altro aspetto, astratto e bello. È molto più importante sapere che sto in un campo invece che sapere se sto maneggiando numeri razionali o reali.

Se so che sono in un campo posso risolvere equazioni lineari come
\eq{
ax=-bx+c
\qquad
ax+bx=c
\qquad
(a+b)x=c
}
e se $a+b\not=0$ allora $x=\frac[c/a+b]$.

\Note{Notate che $\frac[c/a+b]$ è una frazione, non necessariamente un numero razionale. Per quanto ne sappiamo $c$ e $a+b$ possono essere reali, complessi e non interi.
Possono essere conigli se sui conigli abbiamo 2 operazioni che rendono l'insieme  dei conigli un campo.}

Questa attitudine mentale, porta anche un'altra conseguenza.
Quando astraiamo non ci importa più del vero e del falso, ci importa se le cose possono essere derivate dalle proprietà che assumiamo come assiomi dei campi.
Un algoritmo non è vero o falso. È vero o falso {\it in un campo}.

\Note{
Ora se siamo in un campo, possiamo usare le proprietà che sono vere in un campo. Non interessa che
\eq{
\frac[1\not6/\not64]=\frac[1/4]
}
sia vero, non è una cosa che potete usare in un campo (tanto è vero che non credo sia vero in base 8, lo è?).

È parte essenziale di quest'attitudine astratta che uno non può fare quello che vuole tranne quando ciò è falso, al contrario può fare solo quello che può dimostrare.
Quando definite un sistema assiomatico, sapete poche cose e siete vincolati a usare solo gli assiomi. Mano mano che dimostrate teoremi aumentano le cose che sapete 
e siete più liberi di fare. I teoremi non sono lì per darvi cose da ricordare e dover dimostrare, sono lì per rendervi liberi.
}


Infine notiamo che per ora la nostra strada verso l'astrazione è lastricata dei aumentare i problemi (le equazioni) che sappiamo risolvere.
Siamo passati agli interi per poter risolvere le equazioni tipo $3x+3=0$, ai razionali per poter risolvere $6x+3=0$, ai reali per poter risolvere $x^2=2$.

Poi discuteremo $x^2+1=0$.

Per ora introduciamo altri campi e poi le algebre.
\EndDocument
\end


