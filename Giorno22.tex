% !TEX TS-program = latex
% !TEX spellcheck = it-IT
\documentclass[a4paper]{article}
\usepackage[utf8]{inputenc}
\usepackage[T1]{fontenc}
\usepackage[italian]{babel}
\usepackage{tikz} % LATEX
\usetikzlibrary {automata}


%%%%%%%%%%%%%%%%%%%%%%%%%%%%%%%%%%%%%%%%%%%%%%%%%%%%%%%%%%%%%%%%%%%%%%%%%%%%
%% Trim Size: 9.75in x 6.5in
%% Text Area: 8in (include Runningheads) x 5in
%% ws-ijgmmp.tex   :   2-9-08
%% Tex file to use with ws-ijgmmp.cls written in Latex2E.
%% The content, structure, format and layout of this style file is the
%% property of World Scientific Publishing Co. Pte. Ltd.
%% Copyright 1995, 2002 by World Scientific Publishing Co.
%% All rights are reserved.
%%%%%%%%%%%%%%%%%%%%%%%%%%%%%%%%%%%%%%%%%%%%%%%%%%%%%%%%%%%%%%%%%%%%%%%%%%%%
%%

%%%%%%%%%%%%%%%%%%%%%%%%%%%%%


\usepackage{pictexwd,dcpic}
\usepackage{csquotes}
\usepackage{nopageno}

\usepackage{amsmath, amssymb, amsthm}
\usepackage{pictex, dcpic}
\usepackage{color}
%\usepackage{graphicx}
%\numberwithin{equation}{section} % This line resets equation numbering when starting a new section.
%\renewcommand{\theequation}{Eq. \thesection.\arabic{equation}} % This line ads "Eq." in front of your equation numbering.




%
%    Macros.    Version 1.2.0.beta
%    The best use is to paste all of them into the papers
%     1/8/2005
%

%%%%%%%%%%%%%%%%%%%%%%%%%%%%%%
%%%%%%			Greek		 %%%%%%
%%%%%%%%%%%%%%%%%%%%%%%%%%%%%%

\def\al{\alpha}
\def\be{\beta}
\def\de{\delta}
\def\ga{\gamma}
\def\up{\upsilon}
\def\ep{\epsilon}
\def\io{\iota}
\def\te{\theta}
\def\la{\lambda}
\def\ze{\zeta}
\def\om{\omega}
\def\si{\sigma}
\def\vp{\varphi}
\def\vpi{\varpi}
\def\ka{\kappa}
\def\vs{\varsigma}
\def\vr{\varrho}

\def\De{\Delta}
\def\Ga{\Gamma}
\def\Te{\Theta}
\def\La{\Lambda}
\def\Om{\Omega}
\def\Si{\Sigma}
\def\Up{\Upsilon}

\def\boldDe{\bf\Delta}
\def\boldGa{\bf\Gamma}
\def\boldTe{\bf\Theta}
\def\boldLa{\bf\Lambda}
\def\boldOm{\bf\Omega}
\def\boldSi{\bf\Sigma}
\def\boldUp{\bf\Upsilon}
\def\boldXi{\bf\Xi}
\def\boldPi{\bf\Pi}
\def\boldPhi{\bf\Phi}
\def\boldPsi{\bf\Psi}

\def\boldal{\bf\al}
\def\boldbe{\bf\be}
\def\boldga{\bf\ga}
\def\boldde{\bf\de}
\def\boldep{\bf\ep}
\def\boldze{\bf\ze}
\def\boldeta{\bf\eta}
\def\boldte{\bf\te}
\def\boldio{\bf\io}
\def\boldka{\bf\ka}
\def\boldla{\bf\la}
\def\boldmu{\hbox{\greekbold\char "16}}
\def\boldnu{\hbox{\greekbold\char "17}}
\def\boldxi{\hbox{\greekbold\char "18}}
\def\boldpi{\hbox{\greekbold\char "19}}
\def\boldrho{\hbox{\greekbold\char "1A}}
\def\boldsi{\bf\si}
\def\boldtau{\hbox{\greekbold\char "1C}}
\def\boldup{\hbox{\greekbold\char "1D}}
\def\boldphi{\hbox{\greekbold\char "1E}}
\def\boldchi{\hbox{\greekbold\char "1F}}
\def\boldpsi{\hbox{\greekbold\char "20}}
\def\boldom{\hbox{\greekbold\char "21}}
\def\boldvep{\hbox{\greekbold\char "22}}
\def\boldvte{\hbox{\greekbold\char "23}}
\def\boldvpi{\hbox{\greekbold\char "24}}
\def\boldvrho{\hbox{\greekbold\char "25}}
\def\boldvsi{\hbox{\greekbold\char "26}}
\def\boldvphi{\hbox{\greekbold\char "27}}

%%%%%%%%%%%%%%%%%%%%%%%%%%%%%%
%%%%%%			Cal			 %%%%%%
%%%%%%%%%%%%%%%%%%%%%%%%%%%%%%
 \def\calA{{\hbox{\cal A}}}
 \def\calU{{\hbox{\cal U}}}
 \def\calB{{\hbox{\cal B}}}
 \def\calC{{\hbox{\cal C}}}
 \def\calI{{\hbox{\cal I}}}
 \def\calQ{{\hbox{\cal Q}}}
 \def\calP{{\hbox{\cal P}}}
 \def\calL{{\hbox{\cal L}}}
 \def\calE{{\hbox{\cal E}}}
 \def\calW{{\hbox{\cal W}}}
 \def\calV{{\hbox{\cal V}}}
 \def\calS{{\hbox{\cal S}}}
 \def\calF{{\hbox{\cal F}}}
 \def\calR{{\hbox{\cal R}}}
 \def\calO{{\hbox{\cal O}}}
 \def\calM{{\hbox{\cal M}}}

%%%%%%%%%%%%%%%%%%%%%%%%%%%%%%
%%%%%%			gothic		 %%%%%%
%%%%%%%%%%%%%%%%%%%%%%%%%%%%%%
 \def\su{{\mathfrak{su}}}
 \def\gl{{\mathfrak{gl}}}
 \def\gotg{{\mathfrak{g}}}
 \def\goth{{\mathfrak{h}}}
 \def\gotm{{\mathfrak{m}}}
 \def\gotk{{\mathfrak{k}}}
 \def\spin{{\mathfrak{spin}}}
 \def\slC{{\mathfrak{sl}}}
 \def\O{{\mathfrak{O}}}
 \def\Set{{\mathfrak{Set}}}

%%%%%%%%%%%%%%%%%%%%%%%%%%%%%%
%%%%%%			Bbb			 %%%%%%
%%%%%%%%%%%%%%%%%%%%%%%%%%%%%%
 \def\one{\mathbb{I}}
 \def\A{\mathbb{A}}
 \def\B{\mathbb{B}}
 \def\C{\mathbb{C}}
 \def\D{\mathbb{D}}
 \def\E{\mathbb{E}}
 \def\F{\mathbb{F}}
 \def\G{\mathbb{G}}
 \def\H{\mathbb{H}}
 \def\J{\mathbb{J}}
 \def\K{\mathbb{K}}
 \def\I{\mathbb{I}}
 \def\L{\mathbb{L}}
 \def\M{\mathbb{M}}
 \def\N{\mathbb{N}}
 \def\O{\mathbb{O}}
 \def\P{\mathbb{P}}
 \def\Q{\mathbb{Q}}
 \def\R{\mathbb{R}}
 \def\S{\mathbb{S}}
 \def\T{\mathbb{T}}
 \def\U{\mathbb{U}}
 \def\V{\mathbb{V}}
 \def\X{\mathbb{X}}
 \def\Y{\mathbb{Y}}
 \def\W{\mathbb{W}}
 \def\Z{\mathbb{Z}}
 

%%%%%%%%%%%%%%%%%%%%%%%%%%%%%%
%%%%%%		MathRoman		 %%%%%%
%%%%%%%%%%%%%%%%%%%%%%%%%%%%%%
\def\Tr{{\hbox{Tr}}}
\def\Con{{\hbox{Con}}}
\def\Aut{{\hbox{Aut}}}
\def\Div{{\hbox{Div}}}
\def\ad{{\hbox{ad}}}
\def\Ad{{\hbox{Ad}}}
\def\Re{{\hbox{Re}}}
\def\Im{{\hbox{Im}}}
\def\Card{{\hbox{Card}}}
\def\Iso{{\hbox{Iso}}}
\def\Geo{{\hbox{Geo}}}
\def\Int{{\hbox{Int}}}
\def\Inv{{\hbox{Inv}}}
\def\Spin{{\hbox{Spin}}}
\def\SO{{\hbox{SO}}}
\def\SU{{\hbox{SU}}}
\def\SL{{\hbox{SL}}}
\def\GL{{\hbox{GL}}}
\def\det{{\hbox{det}}}
\def\Hom{{\hbox{Hom}}}
\def\End{{\hbox{End}}}
\def\Euc{{\hbox{Euc}}}
\def\Lor{{\hbox{Lor}}}
\def\Diff{{\hbox{Diff}}}
\def\di{{\hbox{d}}}
\def\id{{\hbox{\rm id}}}
\def\diag{{\hbox{diag}}}
\def\rank{{\hbox{rank}}}
\def\span{{\hbox{span}}}
\def\Sim{{\hbox{Sim}}}

\def\Obj{{\hbox{Obj}}}

%%%%%%%%%%%%%%%%%%%%%%%%%%%%%%
%%%%%%		OtherSymbols		 %%%%%%
%%%%%%%%%%%%%%%%%%%%%%%%%%%%%%
\def\ip{\hbox to4pt{\leaders\hrule height0.3pt\hfill}\vbox to8pt{\leaders\vrule width0.3pt\vfill}\kern 2pt}

% inner product
\def\QDE{\hfill\hbox{\ }\vrule height4pt width4pt depth0pt} 
\def\del{\partial}
\def\na{\nabla}
\def\inter{\cap}
\def\Vec{\mathfrak{X}}
\def\Lie{\hbox{\LieFont \$}}

\def\arr{\rightarrow}
\def\larr{\longrightarrow}
\def\harr{\hookrightarrow}
\def\hlarr{\lhook\joinrel\longrightarrow}
\def\then{\Rightarrow}
\def\semidirect{\hbox{\Bbb \char111}}
\def\binomial#1#2{\left(\Matrix{#1\cr #2\cr}\right)}

\def\barJ{\bar J}

\def\sSU{{\hbox{SU}}}
\def\so{{so}}
%\def\N{{{\mathbb N}}}

\def\file#1{{\tt #1}}
\def\calH{{{\cal H}}}
\def\calJ{{{\cal J}}}
\def\calA{{{\cal A}}}
\def\calV{{{\cal V}}}
\def\calG{{{\cal G}}}
\def\calP{{{\cal P}}}
\def\calK{{{\cal K}}}
\def\calD{{{\cal D}}}
\def\scalD{{{\cal D}}}

\def\Loop{{\hbox{Loop}}}
\def\Hoop{{\hbox{Hoop}}}
\def\Sym{{\hbox{Sym}}}
\def\sR{{\R}}
\def\dCor{dCor}
\def\epm{\hbox{$^\pm$}}
\def\dunion{\coprod}

\def\nab#1{{\buildrel #1\over \na}}
\def\frac[#1/#2]{\hbox{$#1\over#2$}}
\def\Frac[#1/#2]{{#1\over#2}}
\def\({\left(}
\def\){\right)}
\def\[{\left[}
\def\]{\right]}
\def\^#1{{}^{#1}_{\>\cdot}}
\def\_#1{{}_{#1}^{\>\cdot}}
\def\Label=#1{{\buildrel {\hbox{\fiveSerif \ShowLabel{#1}}}\over =}}
\def\<{\kern -1pt}
\def\Bar{\>|\>}
\def\Dal{\hbox{\tenRelazioni  \char003}}

\def\uvec#1{\vbox{\hbox{$\scriptstyle\rightharpoonup$}\vskip-9pt\hbox{$#1$}}}
\def\dvec#1{\vtop{\hbox{$#1$}\vskip-10pt\hbox{$\scriptstyle\rightharpoondown$}}}
\def\Cprod{\diamond}
\def\obullet{\odot}



%%%%%%%%%%%%			frames 				%%%%%%%%%%%%%%%%%%%

\def\red#1{{\color{red}{#1}}}
\def\blue#1{{\color{blue}{#1}}}
\def\green#1{{\color{green}{#1}}}

\def\Red{\color{red}}
\def\Blue{\color{blue}}
\def\Green{\color{green}}


\def\frame#1{\vbox{\hrule\hbox{\vrule\vbox{\kern2pt\hbox{\kern2pt#1\kern2pt}\kern2pt}\vrule}\hrule\kern-4pt}} 
\def\redframe#1{\red{\frame{#1}}} 
\def\greenframe#1{\green{\frame{#1}}} 
\def\blueframe#1{\blue{\frame{#1}}} 

\def\uline#1{\underline{#1}}
\def\uuline#1{\underline{\underline{#1}}}
\def\Box to #1#2#3{\frame{\vtop{\hbox to #1{\hfill #2 \hfill}\hbox to #1{\hfill #3 \hfill}}}}



\def\ubal{\underline{\al}\kern1pt}
\def\obal{\overline{\al}\kern1pt}

\def\ubR{\underline{R}\kern1pt}
\def\obR{\overline{R}\kern1pt}
\def\ubom{\underline{\om}\kern1pt}
\def\obxi{\overline{\xi}\kern1pt}
\def\ubu{\underline{u}\kern1pt}
\def\ube{\underline{e}\kern1pt}
\def\obe{\overline{e}\kern1pt}
\def\Limit{\>{\buildrel{r\arr\infty}\over \longrightarrow}\,}
\def\union{\cup}
\def\Emptyset{\varnothing}




\def\Uvec#1{\vbox{\mathsurround=0pt\ialign{##\crcr
     $\scriptscriptstyle\rightharpoonup$\crcr\noalign{\kern1pt\nointerlineskip}
     $\hfil\displaystyle{#1}\hfil$\crcr}}}
\def\Dvec#1{\vbox{\mathsurround=0pt\ialign{##\crcr
     $\scriptscriptstyle\rightharpoondown$\crcr\noalign{\kern-7pt\nointerlineskip}
     $\hfil\displaystyle{#1}\hfil$\crcr}}}


%   u 
\def\uvecu{\vbox{\mathsurround=0pt\ialign{##\crcr
     $\scriptscriptstyle\rightharpoonup$\crcr\noalign{\kern1pt\nointerlineskip}
     $\hfil\displaystyle{u}\hfil$\crcr}}}
\def\dvecu{\vbox{\mathsurround=0pt\ialign{##\crcr
     $\scriptscriptstyle\rightharpoondown$\crcr\noalign{\kern-7pt\nointerlineskip}
     $\hfil\displaystyle{u}\hfil$\crcr}}}
%\def\uvecu{\Uvec{u}}
%\def\dvecu{\Dvec{u}}

%   be
\def\uvecbe{\vbox{\mathsurround=0pt\ialign{##\crcr
     \kern3pt$\scriptscriptstyle\rightharpoonup$\crcr\noalign{\kern1pt\nointerlineskip}
     $\hfil\displaystyle{\be}\hfil$\crcr}}}
\def\dvecbe{\vbox{\mathsurround=0pt\ialign{##\crcr
     \kern1pt$\scriptscriptstyle\rightharpoondown$\crcr\noalign{\kern-10pt\nointerlineskip}
     $\hfil\displaystyle{\be}\hfil$\crcr}}}

%   n
\def\uvecn{\vbox{\mathsurround=0pt\ialign{##\crcr
     $\scriptscriptstyle\rightharpoonup$\crcr\noalign{\kern1pt\nointerlineskip}
     $\hfil\displaystyle{n}\hfil$\crcr}}}
\def\dvecn{\vbox{\mathsurround=0pt\ialign{##\crcr
     $\scriptscriptstyle\rightharpoondown$\crcr\noalign{\kern-7pt\nointerlineskip}
     $\hfil\displaystyle{n}\hfil$\crcr}}}

%   m
\def\uvecm{\vbox{\mathsurround=0pt\ialign{##\crcr
     $\scriptscriptstyle\rightharpoonup$\crcr\noalign{\kern1pt\nointerlineskip}
     $\hfil\displaystyle{m}\hfil$\crcr}}}
\def\dvecm{\vbox{\mathsurround=0pt\ialign{##\crcr
     $\scriptscriptstyle\rightharpoondown$\crcr\noalign{\kern-7pt\nointerlineskip}
     $\hfil\displaystyle{m}\hfil$\crcr}}}

%   N
\def\uvecN{\vbox{\mathsurround=0pt\ialign{##\crcr
     \kern3pt$\scriptscriptstyle\rightharpoonup$\crcr\noalign{\kern1pt\nointerlineskip}
     $\hfil\displaystyle{N}\hfil$\crcr}}}
\def\dvecN{\vbox{\mathsurround=0pt\ialign{##\crcr
     \kern0pt$\scriptscriptstyle\rightharpoondown$\crcr\noalign{\kern-10pt\nointerlineskip}
     $\hfil\displaystyle{N}\hfil$\crcr}}}

%   u
\def\uvecu{\vbox{\mathsurround=0pt\ialign{##\crcr
     $\scriptscriptstyle\rightharpoonup$\crcr\noalign{\kern1pt\nointerlineskip}
     $\hfil\displaystyle{u}\hfil$\crcr}}}
\def\dvecu{\vbox{\mathsurround=0pt\ialign{##\crcr
     $\scriptscriptstyle\rightharpoondown$\crcr\noalign{\kern-7pt\nointerlineskip}
     $\hfil\displaystyle{u}\hfil$\crcr}}}

%   w
\def\uvecw{\vbox{\mathsurround=0pt\ialign{##\crcr
     $\scriptscriptstyle\rightharpoonup$\crcr\noalign{\kern1pt\nointerlineskip}
     $\hfil\displaystyle{w}\hfil$\crcr}}}
\def\dvecw{\vbox{\mathsurround=0pt\ialign{##\crcr
     $\scriptscriptstyle\rightharpoondown$\crcr\noalign{\kern-7pt\nointerlineskip}
     $\hfil\displaystyle{w}\hfil$\crcr}}}

%   v
\def\uvecv{\vbox{\mathsurround=0pt\ialign{##\crcr
     $\scriptscriptstyle\rightharpoonup$\crcr\noalign{\kern1pt\nointerlineskip}
     $\hfil\displaystyle{v}\hfil$\crcr}}}
\def\dvecv{\vbox{\mathsurround=0pt\ialign{##\crcr
     $\scriptscriptstyle\rightharpoondown$\crcr\noalign{\kern-7pt\nointerlineskip}
     $\hfil\displaystyle{v}\hfil$\crcr}}}

\def\astA{{}^\ast A}
\def\circA{{}^\circ A}
\def\astk{{}^\ast k}
\def\circk{{}^\circ k}
\def\astK{{}^\ast K}
\def\circK{{}^\circ K}
\def\astL{{}^\ast L}
\def\circL{{}^\circ L}
\def\astal{{}^\ast \al}
\def\circal{{}^\circ \al}
\def\astsi{{}^\ast \si}
\def\circsi{{}^\circ \si}
\def\aste{{}^\ast e}
\def\circe{{}^\circ e}
\def\astte{{}^\ast \te}
\def\circte{{}^\circ \te}
\def\astGa{{}^\ast \Ga}
\def\circGa{{}^\circ \Ga}
\def\Lie{\pounds}

\long\def\Hide#1{}
\long\def\HideMarked#1{\hfill{$\triangleright$}}
\def\rtau{\tau}


%%%%%%%%%%%%%%%%%%%%%%%%%%%%%%%%%%%%%%%%%%%%%%%%%%%%%%%%%%%%%%%
%\def\NormalStyle{\leftskip=2cm\rightskip=0cm\normalsize\parindent=5pt\parskip=3pt\normalbaselineskip=14pt\baselineskip=\normalbaselineskip}
%\def\AbstractStyle{\leftskip=3cm\rightskip=1cm\scriptsize\parindent=0pt\parskip=0pt\normalbaselineskip=11pt\baselineskip=\normalbaselineskip}
%\def\NoteStyle{\leftskip=3cm\rightskip=1cm\scriptsize\parindent=0pt\parskip=0pt\normalbaselineskip=11pt\baselineskip=\normalbaselineskip}

\def\ShowLabel#1{\ref{#1}}

\def\Bibliography{\begin{thebibliography}{199}\footnotesize\References}
\def\EndBibliography{\end{thebibliography}}
\def\bib#1#2{\bibitem{#1}#2}
\def\Ref#1{\cite{#1}}


\def\NewSection#1{\section{#1}}
\def\NewSubSection#1{\subsection*{#1}}
\def\NewAppendix#1#2{\section*{Appendix #1: #2}}
\def\Acknowledgements{\section*{Acknowledgements}}
\def\bs{\bigskip}
\def\ms{\medskip}
\def\ss{\smallskip}
\def\ni{\noindent}

\long\def\Note#1{\blockquote{\footnotesize {\bf Nota:}~#1\par}}
\long\def\Ex#1{\blockquote{\footnotesize {\bf Esercizio:}~#1\par}}

\def\eq#1{\begin{equation}#1\end{equation}}
\def\eqLabel#1#2{\begin{equation}#1\label{#2}\end{equation}}


\def\Cases#1{\begin{cases}#1\end{cases}}
\def\Matrix#1{\begin{matrix}#1\end{matrix}}
\def\Align#1{\begin{aligned}#1\end{aligned}}

\def\eqs#1{\eq{\Align{#1}}}
\def\eqsLabel#1#2{\eq{\Align{#1}\label{#2}}}




\def\Abstract{\AbstractStyle{\bf Abstract. }}
\def\EndAbstract{\par\NormalStyle}

\def\TitleScript{}
\def\TitleLine#1{\def\TitleScript{#1}}
\def\MoreTitleLine#1{\edef\TitleScript{\TitleScript\\#1}}

\def\AuthorScript{}
\def\AuthorLine#1{\def\AuthorScript{#1}}
\def\MoreAuthorLine#1{\edef\AuthorScript{\AuthorScript\\#1}}

\def\AddressScript{}
\def\AddressLine#1{\def\AddressScript{#1}}
\def\MoreAddressLine#1{\edef\AddressScript{\AddressScript\\#1}}

\date{}

\def\BeginDocument{\begin{document}}

\def\MakeTitle{
\begin{center}
{\Large\bf\sffamily\baselineskip=2pt\TitleScript}\\
\vskip 10pt
{\small by {\it \AuthorScript}}\\
\vskip 10pt
{\small \AddressScript}
\end{center}
\ms
}

\def\EndDocument{\end{document}}

\def\Figure[#1]#2{\begin{figure}[htbp] %  figure placement: here, top, bottom, or page
   \centering
   \includegraphics[#1]{#2} }
\def\Caption#1{\caption{#1}}
\def\EndFigure{\end{figure}}

\def\Diagram#1{\eq{
\begindc{\commdiag}[10]
#1
\enddc%
}%
}

\def\Itemize#1{\begin{itemize}#1\end{itemize}}
\def\Item[#1]{\item[#1]}

\def\AllReferences{}



\def\Shrink{\small}
%%%%%%%%%%%%%%%%%%%%%%%%%%%%%%%%%%%%%%%%%%%%%%%%%
%%%%%%%%%%%%%%%%%%%%%%%%%%%%%%%%%%%%%%%%%%%%%%%%%%%

%\BeginDocument


%%%%%%%%%%%%%%%%%%%%%%%%%%%%%%%%%%%%%%%%%%%%%%%%%%%






















\parindent=5pt
\baselineskip=13pt
\parskip=5pt





\begin{document}

\AllReferences


%%%%%%%%%%%%%%%%%%%%% Publisher's Area please ignore %%%%%%%%%%%%%%%
%
%
%%%%%%%%%%%%%%%%%%%%%%%%%%%%%%%%%%%%%%%%%%%%%%%%%%%%%%%%%%%%%%%%%%%%

\section*{Giorno 22: lettere, incognite e parametri}

Una volta che abbiamo imparato a fare le operazioni con i numeri razionali, che contengono pure gli interi e i naturali come casi particolari,
in molte situazioni vogliamo scrivere identità che valgono per ogni valore dei numeri coinvolti.
Per questa ragione abbiamo inventato i parametri $a, b, c, \dots\in \Q$.
Quando ad esempio scriviamo le proprietà delle operazioni
\eq{
a(b+c)= ab+ac
}
intendiamo che questa vale per ogni valore di $a, b, c\in \Q$. 

Tra breve introdurremo le incognite che di solito indichiamo con $x$ che hanno una semantica diversa. Quando scriviamo 
\eq{
3x+3=1-2x
}
intendiamo chiedere se esiste un valore di $x$ che rende uguali i 2 lati dell'equazione.
I {\it parametri} sottintendono un quantificatore universale (per ogni) mentre le {\it incognite} sottintendono un quantificatore esistenziale.
Se vi aiuta, le incognite hanno un valore interrogativo aggiunto: esiste un valore di $x$ per cui è verificata l'equazione?

\Note{Quindi quando scriviamo 
\eq{
\frac[a+b/a]=b
}
non basta vedere che per $b=2$ e $a=2$ è soddisfatta per accettarla come formula valida, 
mentre basta vedere che non vale per $a=1$ e $b=2$ per dire che l'identità non è corretta.
}

Ora se state pensando che distinguere tra parametri e incognite è facile, se vedo $x, y, z, \dots$ sono incognite se vedo $a, b, c, \dots$ sono parametri anche questo non è sempre vero.
Se vi chiedo di scrivere l'equazione della retta generica mi scrivete $ax+by=c$ intendendo che per ogni valore di $[a,b,c]$ che sono parametri,
i punti $(x,y)$  che soddisfano l'equazione si distribuiscono lungo una retta, quale retta è identificata dai parametri, mentre $x, y$ compaiono come incognite.

Se poi chiediamo quali rette passano per il punto $(2, 3)$, dobbiamo sostituire a $(x, y)=(2, 3)$ ottenendo $2a+3b=c$ e a questo punto dobbiamo determinare $(a, b,c)$ in modo che questa identità sia valida. Le lettere $a,b,c$ che nel paragrafo precedente erano parametri, ora diventano incognite. Allora si determina $c=2a+3b$ e si sostituisca
\eq{
ax+by=2a+3b
\qquad
a(x-2)+b(y-3)=0
}
che di nuovo, per ogni valore $[a,b]$ determinano una retta che passa per il punto $(2,3)$.
Quindi $a, b$ tornano parametri e $x, y$ tornano incignite.

In pratica è il lettore in base a quello che vuole fare che decide cosa è parametro e cosa è incognita.


È ovvio che per fare i conti con frazioni qualunque non serve fare i conti coi numeri ma bisogna vedere il calcolo come manipolazione di formule seguendo delle regole 
che derivano dalle proprietà delle operazioni
\eq{
\frac[a/b]+ \frac[c/d]=\frac[ad+cb/bd]
\qquad\qquad
\frac[a/b] \frac[c/d]=\frac[ac/bd]
\qquad\qquad
\frac[{\frac[a/b] }/{  \frac[c/d]}]= \frac[a/b]\frac[d/c]
}

\Note{Queste sono le stesse regole che si usano nel caso delle frazioni numeriche, tranne il fatto che per evitare di dover semplificare alla fine (cosa che peraltro può succedere comunque)
quando sommiamo le frazioni si cerca nel caso numerico il minimo comune denominatore, mentre nel caso letterale, visto che non sappiamo i valori non possiamo di certo sapere il minimo comune denominatore e quindi lo facciamo prendendo come denominatore semplicemente il prodotto. 
}

Anche questo passaggio dai numeri alle lettere è un passaggio delicato e qualcuno non lo passa mai.
Intanto dovete scordarvi che sia
\eq{
\frac[a+b/a]=b
\qquad\o\qquad
\frac[a+b/a]= 1+b
}
Potete facilmente convincervi che queste formule non sono valide, fatelo.

La seconda cosa importante sono le regole di precedenza tra le operazioni. 
Tutto nasce dal fatto che per specificare l'ordine con cui vengono applicate le operazioni, bisognerebbe scrivere
\eq{
\frac[1+3+4/8] = (1+(3+4))/8
}
Prima si svolgono i conti nelle parentesi più interne quindi si calcola
\eq{
\frac[1+3+4/8] = (1+(3+4))/8
=  (1+7)/8
= 8/8
=1
}

Ciò comporta che, non fosse per la proprietà associativa della somma, non si potrebbe neanche scrivere $1+3+4$ perché sarebbe ambiguo distinguere tra $(1+3)+4$ e $1+(3+4)$.
È solo in virtù della proprietà associativa che  $(a+b)+c=a+(b+c)$ e quindi, non essendoci differenza tra le 2 scritture, possiamo scrivere ancora peggio $a+b+c$.
La stessa cosa per il prodotto $abc$.

Ma il problema emerge comunque quando si mischiano somme e prodotti. Se scriviamo $a+bc$ intendiamo
$a+(bc)$ o $(a+b)c$?

\Note{Notate che sui libri $a+bc=a+(bc)$ mentre sulle calcolatrici
$a+bc$ calcola $(a+b)*c$. 
Le calcolatrici calcolano le funzioni nel valore che compare sullo schermo, calcolano e operazioni tra il valore sullo schermo e il successivo valore digitato.

Almeno di norma perché poi esistono calcolatrici bellissime in polacco inverso in cui per calcolare 5+3 si digita 3 5 + (come in forth o nel linguaggio che si usa per i compilatori)
o in polacco diretto in cui si digita + 5 3 (come in lisp in ps o in pdf).
In più oggi la potenza di calcolo è sufficiente per implementare sulle calcolatrici le convenzioni dei libri con la possibilità di scrivere una formula su una riga con tutte le parentesi necessarie
e poi interpretarla e calcolare il risultato. Questo succede su molti smartphones ma ad esempio le calcolatrici di sistema di windows implementano le convenzioni standard delle calcolatrici standard.
}

Non so perché i lettori siano allergici alle parentesi, ma per evitarle si conviene che il prodotto abbia precedenza sulle somme, quindi
\eq{
a+bc=a+(bc)
}
mentre se uno vuole può scrivere $(a+b)c$ in cui la parentesi è necessaria se prima si vuole fare la somma e poi il prodotto.

La stessa  cosa le potenze hanno la precedenza sui prodotti. di norma si interpreta $xy^2= x\cdot (y^2)$ non come $(xy)^2$.

Onestamente sono cose che abbiamo assunto col latte materno e farei fatica a fare una lista delle precedenze adottate in matematica.
Le abbiamo assunto facendo una serie interminabili di esercizi tipo semplificare l'espressione
\eq{
\Frac[(x-y)(x+y)/xy]+ \Frac[2y/x]+2-\Frac[(x+y)^2/xy]
}
in cui tra l'altro $x, y$ compaiono come parametri.
Semplificare una espressione significa riscriverla in modo più semplice, riscrivendo sempre espressioni uguali.

\Note{
Semplificare l'espressione 
\eq{
\Align{
&\Frac[1/x^2-4x+4]- \Frac[2/x^2-2x] +\Frac[1/x^2-x-2]=\cr
&\quad=\Frac[1/(x-2)^2]- \Frac[2/x(x-2)] +\Frac[1/ (x-2)(x+1)]=\cr
&\quad=\Frac[x(x+1)/x(x+1)(x-2)^2]+\Frac[-2(x+1)(x-2)/x(x+1)(x-2)^2]+ \Frac[x(x-2)/x(x+1)(x-2)^2]=\cr
&\quad=\Frac[x(x+1)  -2(x+1)(x-2)	 +x(x-2)		/x(x+1)(x-2)^2]=\cr
&\quad=\Frac[x^2+x  -2x^2+2x+4+x^2-2x 		/x(x+1)(x-2)^2]=\cr
&\quad=\Frac[  x+4		/x(x+1)(x-2)^2]\cr
}
}
L'esercizio prende la forma di una catena di uguaglianze tra espressioni equivalenti piano piano più ``semplici''.
}

Assicuratevi di saperla semplificare, saperla calcolare ponendo $x=7$ e $y=-4$.
Questo è un buon posto per fermarsi se non siete super-convinti di saperlo fare. 
Se cercate {\it esercizi espressioni algebriche letterali} trovate un miliardo di esercizi da svolgere.

Esercitatevi, scoprite dove siete soliti fare errori, condizionatevi a controllare mentre fate i conti di non aver fatto errori. 
È una forma di meditazione e solo così potete ottenere di riuscire a eliminare gli errori che condizionano il risultato.
Se l'espressione si allunga e siete esposti a fare un errore ogni 100 passaggi, da una certa complessità in sù sarete pressoché certi di ottenere un risultato sbagliato, e non potete permettervelo.

Dice il saggio: 
\blockquote{\footnotesize
It was never really a battle for me to win, \\
it was an eternal dance\\
And like a dance, the more rigid I became, the harder it got\\
The more I cursed my clumsy footsteps, the more I struggled

So I got older and I learned to relax\\
And I learned to soften and that dance got easier

It is this eternal dance that separates human beings\\
From angels, from demons, from gods\\
And I must not forget, we must not forget\\
That we are human beings\\
}

Può essere fatto, potete diventare sicuri del conto che avete fatto, anche se è lungo una pagina.



\EndDocument
\end


