% !TEX TS-program = latex
% !TEX spellcheck = it-IT
\documentclass[a4paper]{article}
\usepackage[utf8]{inputenc}
\usepackage[T1]{fontenc}
\usepackage[italian]{babel}

%%%%%%%%%%%%%%%%%%%%%%%%%%%%%%%%%%%%%%%%%%%%%%%%%%%%%%%%%%%%%%%%%%%%%%%%%%%%
%% Trim Size: 9.75in x 6.5in
%% Text Area: 8in (include Runningheads) x 5in
%% ws-ijgmmp.tex   :   2-9-08
%% Tex file to use with ws-ijgmmp.cls written in Latex2E.
%% The content, structure, format and layout of this style file is the
%% property of World Scientific Publishing Co. Pte. Ltd.
%% Copyright 1995, 2002 by World Scientific Publishing Co.
%% All rights are reserved.
%%%%%%%%%%%%%%%%%%%%%%%%%%%%%%%%%%%%%%%%%%%%%%%%%%%%%%%%%%%%%%%%%%%%%%%%%%%%
%%

%%%%%%%%%%%%%%%%%%%%%%%%%%%%%


\usepackage{pictexwd,dcpic}
\usepackage{csquotes}
\usepackage{nopageno}

\usepackage{amsmath, amssymb, amsthm}
\usepackage{pictex, dcpic}
\usepackage{color}
%\usepackage{graphicx}
%\numberwithin{equation}{section} % This line resets equation numbering when starting a new section.
%\renewcommand{\theequation}{Eq. \thesection.\arabic{equation}} % This line ads "Eq." in front of your equation numbering.




%
%    Macros.    Version 1.2.0.beta
%    The best use is to paste all of them into the papers
%     1/8/2005
%

%%%%%%%%%%%%%%%%%%%%%%%%%%%%%%
%%%%%%			Greek		 %%%%%%
%%%%%%%%%%%%%%%%%%%%%%%%%%%%%%

\def\al{\alpha}
\def\be{\beta}
\def\de{\delta}
\def\ga{\gamma}
\def\up{\upsilon}
\def\ep{\epsilon}
\def\io{\iota}
\def\te{\theta}
\def\la{\lambda}
\def\ze{\zeta}
\def\om{\omega}
\def\si{\sigma}
\def\vp{\varphi}
\def\vpi{\varpi}
\def\ka{\kappa}
\def\vs{\varsigma}
\def\vr{\varrho}

\def\De{\Delta}
\def\Ga{\Gamma}
\def\Te{\Theta}
\def\La{\Lambda}
\def\Om{\Omega}
\def\Si{\Sigma}
\def\Up{\Upsilon}

\def\boldDe{\bf\Delta}
\def\boldGa{\bf\Gamma}
\def\boldTe{\bf\Theta}
\def\boldLa{\bf\Lambda}
\def\boldOm{\bf\Omega}
\def\boldSi{\bf\Sigma}
\def\boldUp{\bf\Upsilon}
\def\boldXi{\bf\Xi}
\def\boldPi{\bf\Pi}
\def\boldPhi{\bf\Phi}
\def\boldPsi{\bf\Psi}

\def\boldal{\bf\al}
\def\boldbe{\bf\be}
\def\boldga{\bf\ga}
\def\boldde{\bf\de}
\def\boldep{\bf\ep}
\def\boldze{\bf\ze}
\def\boldeta{\bf\eta}
\def\boldte{\bf\te}
\def\boldio{\bf\io}
\def\boldka{\bf\ka}
\def\boldla{\bf\la}
\def\boldmu{\hbox{\greekbold\char "16}}
\def\boldnu{\hbox{\greekbold\char "17}}
\def\boldxi{\hbox{\greekbold\char "18}}
\def\boldpi{\hbox{\greekbold\char "19}}
\def\boldrho{\hbox{\greekbold\char "1A}}
\def\boldsi{\bf\si}
\def\boldtau{\hbox{\greekbold\char "1C}}
\def\boldup{\hbox{\greekbold\char "1D}}
\def\boldphi{\hbox{\greekbold\char "1E}}
\def\boldchi{\hbox{\greekbold\char "1F}}
\def\boldpsi{\hbox{\greekbold\char "20}}
\def\boldom{\hbox{\greekbold\char "21}}
\def\boldvep{\hbox{\greekbold\char "22}}
\def\boldvte{\hbox{\greekbold\char "23}}
\def\boldvpi{\hbox{\greekbold\char "24}}
\def\boldvrho{\hbox{\greekbold\char "25}}
\def\boldvsi{\hbox{\greekbold\char "26}}
\def\boldvphi{\hbox{\greekbold\char "27}}

%%%%%%%%%%%%%%%%%%%%%%%%%%%%%%
%%%%%%			Cal			 %%%%%%
%%%%%%%%%%%%%%%%%%%%%%%%%%%%%%
 \def\calA{{\hbox{\cal A}}}
 \def\calU{{\hbox{\cal U}}}
 \def\calB{{\hbox{\cal B}}}
 \def\calC{{\hbox{\cal C}}}
 \def\calI{{\hbox{\cal I}}}
 \def\calQ{{\hbox{\cal Q}}}
 \def\calP{{\hbox{\cal P}}}
 \def\calL{{\hbox{\cal L}}}
 \def\calE{{\hbox{\cal E}}}
 \def\calW{{\hbox{\cal W}}}
 \def\calS{{\hbox{\cal S}}}
 \def\calF{{\hbox{\cal F}}}
 \def\calR{{\hbox{\cal R}}}
 \def\calO{{\hbox{\cal O}}}
 \def\calM{{\hbox{\cal M}}}

%%%%%%%%%%%%%%%%%%%%%%%%%%%%%%
%%%%%%			gothic		 %%%%%%
%%%%%%%%%%%%%%%%%%%%%%%%%%%%%%
 \def\su{{\mathfrak{su}}}
 \def\gl{{\mathfrak{gl}}}
 \def\gotg{{\mathfrak{g}}}
 \def\goth{{\mathfrak{h}}}
 \def\gotm{{\mathfrak{m}}}
 \def\gotk{{\mathfrak{k}}}
 \def\spin{{\mathfrak{spin}}}
 \def\slC{{\mathfrak{sl}}}
 \def\O{{\mathfrak{O}}}
 \def\Set{{\mathfrak{Set}}}

%%%%%%%%%%%%%%%%%%%%%%%%%%%%%%
%%%%%%			Bbb			 %%%%%%
%%%%%%%%%%%%%%%%%%%%%%%%%%%%%%
 \def\one{\mathbb{I}}
 \def\A{\mathbb{A}}
 \def\B{\mathbb{B}}
 \def\C{\mathbb{C}}
 \def\D{\mathbb{D}}
 \def\E{\mathbb{E}}
 \def\F{\mathbb{F}}
 \def\G{\mathbb{G}}
 \def\H{\mathbb{H}}
 \def\J{\mathbb{J}}
 \def\K{\mathbb{K}}
 \def\I{\mathbb{I}}
 \def\L{\mathbb{L}}
 \def\M{\mathbb{M}}
 \def\N{\mathbb{N}}
 \def\O{\mathbb{O}}
 \def\P{\mathbb{P}}
 \def\Q{\mathbb{Q}}
 \def\R{\mathbb{R}}
 \def\S{\mathbb{S}}
 \def\T{\mathbb{T}}
 \def\U{\mathbb{U}}
 \def\V{\mathbb{V}}
 \def\X{\mathbb{X}}
 \def\Y{\mathbb{Y}}
 \def\W{\mathbb{W}}
 \def\Z{\mathbb{Z}}
 

%%%%%%%%%%%%%%%%%%%%%%%%%%%%%%
%%%%%%		MathRoman		 %%%%%%
%%%%%%%%%%%%%%%%%%%%%%%%%%%%%%
\def\Tr{{\hbox{Tr}}}
\def\Con{{\hbox{Con}}}
\def\Aut{{\hbox{Aut}}}
\def\Div{{\hbox{Div}}}
\def\ad{{\hbox{ad}}}
\def\Ad{{\hbox{Ad}}}
\def\Re{{\hbox{Re}}}
\def\Im{{\hbox{Im}}}
\def\Iso{{\hbox{Iso}}}
\def\Geo{{\hbox{Geo}}}
\def\Int{{\hbox{Int}}}
\def\Inv{{\hbox{Inv}}}
\def\Spin{{\hbox{Spin}}}
\def\SO{{\hbox{SO}}}
\def\SU{{\hbox{SU}}}
\def\SL{{\hbox{SL}}}
\def\GL{{\hbox{GL}}}
\def\det{{\hbox{det}}}
\def\Hom{{\hbox{Hom}}}
\def\End{{\hbox{End}}}
\def\Euc{{\hbox{Euc}}}
\def\Lor{{\hbox{Lor}}}
\def\Diff{{\hbox{Diff}}}
\def\di{{\hbox{d}}}
\def\id{{\hbox{\rm id}}}
\def\diag{{\hbox{diag}}}
\def\rank{{\hbox{rank}}}
\def\span{{\hbox{span}}}
\def\Sim{{\hbox{Sim}}}

\def\Obj{{\hbox{Obj}}}

%%%%%%%%%%%%%%%%%%%%%%%%%%%%%%
%%%%%%		OtherSymbols		 %%%%%%
%%%%%%%%%%%%%%%%%%%%%%%%%%%%%%
\def\ip{\hbox to4pt{\leaders\hrule height0.3pt\hfill}\vbox to8pt{\leaders\vrule width0.3pt\vfill}\kern 2pt}

% inner product
\def\QDE{\hfill\hbox{\ }\vrule height4pt width4pt depth0pt} 
\def\del{\partial}
\def\na{\nabla}
\def\inter{\cap}
\def\Vec{\mathfrak{X}}
\def\Lie{\hbox{\LieFont \$}}

\def\arr{\rightarrow}
\def\larr{\longrightarrow}
\def\harr{\hookrightarrow}
\def\hlarr{\lhook\joinrel\longrightarrow}
\def\then{\Rightarrow}
\def\semidirect{\hbox{\Bbb \char111}}
\def\binomial#1#2{\left(\Matrix{#1\cr #2\cr}\right)}

\def\barJ{\bar J}

\def\sSU{{\hbox{SU}}}
\def\so{{so}}
%\def\N{{{\mathbb N}}}

\def\file#1{{\tt #1}}
\def\calH{{{\cal H}}}
\def\calJ{{{\cal J}}}
\def\calG{{{\cal G}}}
\def\calK{{{\cal K}}}
\def\calD{{{\cal D}}}
\def\scalD{{{\cal D}}}

\def\Loop{{\hbox{Loop}}}
\def\Hoop{{\hbox{Hoop}}}
\def\Sym{{\hbox{Sym}}}
\def\sR{{\R}}
\def\dCor{dCor}
\def\epm{\hbox{$^\pm$}}
\def\dunion{\coprod}

\def\nab#1{{\buildrel #1\over \na}}
\def\frac[#1/#2]{\hbox{$#1\over#2$}}
\def\Frac[#1/#2]{{#1\over#2}}
\def\({\left(}
\def\){\right)}
\def\[{\left[}
\def\]{\right]}
\def\^#1{{}^{#1}_{\>\cdot}}
\def\_#1{{}_{#1}^{\>\cdot}}
\def\Label=#1{{\buildrel {\hbox{\fiveSerif \ShowLabel{#1}}}\over =}}
\def\<{\kern -1pt}
\def\Bar{\>|\>}
\def\Dal{\hbox{\tenRelazioni  \char003}}

\def\uvec#1{\vbox{\hbox{$\scriptstyle\rightharpoonup$}\vskip-9pt\hbox{$#1$}}}
\def\dvec#1{\vtop{\hbox{$#1$}\vskip-10pt\hbox{$\scriptstyle\rightharpoondown$}}}
\def\Cprod{\diamond}
\def\obullet{\odot}



%%%%%%%%%%%%			frames 				%%%%%%%%%%%%%%%%%%%

\def\red#1{{\color{red}{#1}}}
\def\blue#1{{\color{blue}{#1}}}
\def\green#1{{\color{green}{#1}}}

\def\Red{\color{red}}
\def\Blue{\color{blue}}
\def\Green{\color{green}}


\def\frame#1{\vbox{\hrule\hbox{\vrule\vbox{\kern2pt\hbox{\kern2pt#1\kern2pt}\kern2pt}\vrule}\hrule\kern-4pt}} 
\def\redframe#1{\red{\frame{#1}}} 
\def\greenframe#1{\green{\frame{#1}}} 
\def\blueframe#1{\blue{\frame{#1}}} 

\def\uline#1{\underline{#1}}
\def\uuline#1{\underline{\underline{#1}}}
\def\Box to #1#2#3{\frame{\vtop{\hbox to #1{\hfill #2 \hfill}\hbox to #1{\hfill #3 \hfill}}}}



\def\ubal{\underline{\al}\kern1pt}
\def\obal{\overline{\al}\kern1pt}

\def\ubR{\underline{R}\kern1pt}
\def\obR{\overline{R}\kern1pt}
\def\ubom{\underline{\om}\kern1pt}
\def\obxi{\overline{\xi}\kern1pt}
\def\ubu{\underline{u}\kern1pt}
\def\ube{\underline{e}\kern1pt}
\def\obe{\overline{e}\kern1pt}
\def\Limit{\>{\buildrel{r\arr\infty}\over \longrightarrow}\,}
\def\union{\cup}
\def\Emptyset{\varnothing}




\def\Uvec#1{\vbox{\mathsurround=0pt\ialign{##\crcr
     $\scriptscriptstyle\rightharpoonup$\crcr\noalign{\kern1pt\nointerlineskip}
     $\hfil\displaystyle{#1}\hfil$\crcr}}}
\def\Dvec#1{\vbox{\mathsurround=0pt\ialign{##\crcr
     $\scriptscriptstyle\rightharpoondown$\crcr\noalign{\kern-7pt\nointerlineskip}
     $\hfil\displaystyle{#1}\hfil$\crcr}}}


%   u 
\def\uvecu{\vbox{\mathsurround=0pt\ialign{##\crcr
     $\scriptscriptstyle\rightharpoonup$\crcr\noalign{\kern1pt\nointerlineskip}
     $\hfil\displaystyle{u}\hfil$\crcr}}}
\def\dvecu{\vbox{\mathsurround=0pt\ialign{##\crcr
     $\scriptscriptstyle\rightharpoondown$\crcr\noalign{\kern-7pt\nointerlineskip}
     $\hfil\displaystyle{u}\hfil$\crcr}}}
%\def\uvecu{\Uvec{u}}
%\def\dvecu{\Dvec{u}}

%   be
\def\uvecbe{\vbox{\mathsurround=0pt\ialign{##\crcr
     \kern3pt$\scriptscriptstyle\rightharpoonup$\crcr\noalign{\kern1pt\nointerlineskip}
     $\hfil\displaystyle{\be}\hfil$\crcr}}}
\def\dvecbe{\vbox{\mathsurround=0pt\ialign{##\crcr
     \kern1pt$\scriptscriptstyle\rightharpoondown$\crcr\noalign{\kern-10pt\nointerlineskip}
     $\hfil\displaystyle{\be}\hfil$\crcr}}}

%   n
\def\uvecn{\vbox{\mathsurround=0pt\ialign{##\crcr
     $\scriptscriptstyle\rightharpoonup$\crcr\noalign{\kern1pt\nointerlineskip}
     $\hfil\displaystyle{n}\hfil$\crcr}}}
\def\dvecn{\vbox{\mathsurround=0pt\ialign{##\crcr
     $\scriptscriptstyle\rightharpoondown$\crcr\noalign{\kern-7pt\nointerlineskip}
     $\hfil\displaystyle{n}\hfil$\crcr}}}

%   m
\def\uvecm{\vbox{\mathsurround=0pt\ialign{##\crcr
     $\scriptscriptstyle\rightharpoonup$\crcr\noalign{\kern1pt\nointerlineskip}
     $\hfil\displaystyle{m}\hfil$\crcr}}}
\def\dvecm{\vbox{\mathsurround=0pt\ialign{##\crcr
     $\scriptscriptstyle\rightharpoondown$\crcr\noalign{\kern-7pt\nointerlineskip}
     $\hfil\displaystyle{m}\hfil$\crcr}}}

%   N
\def\uvecN{\vbox{\mathsurround=0pt\ialign{##\crcr
     \kern3pt$\scriptscriptstyle\rightharpoonup$\crcr\noalign{\kern1pt\nointerlineskip}
     $\hfil\displaystyle{N}\hfil$\crcr}}}
\def\dvecN{\vbox{\mathsurround=0pt\ialign{##\crcr
     \kern0pt$\scriptscriptstyle\rightharpoondown$\crcr\noalign{\kern-10pt\nointerlineskip}
     $\hfil\displaystyle{N}\hfil$\crcr}}}

%   u
\def\uvecu{\vbox{\mathsurround=0pt\ialign{##\crcr
     $\scriptscriptstyle\rightharpoonup$\crcr\noalign{\kern1pt\nointerlineskip}
     $\hfil\displaystyle{u}\hfil$\crcr}}}
\def\dvecu{\vbox{\mathsurround=0pt\ialign{##\crcr
     $\scriptscriptstyle\rightharpoondown$\crcr\noalign{\kern-7pt\nointerlineskip}
     $\hfil\displaystyle{u}\hfil$\crcr}}}

%   w
\def\uvecw{\vbox{\mathsurround=0pt\ialign{##\crcr
     $\scriptscriptstyle\rightharpoonup$\crcr\noalign{\kern1pt\nointerlineskip}
     $\hfil\displaystyle{w}\hfil$\crcr}}}
\def\dvecw{\vbox{\mathsurround=0pt\ialign{##\crcr
     $\scriptscriptstyle\rightharpoondown$\crcr\noalign{\kern-7pt\nointerlineskip}
     $\hfil\displaystyle{w}\hfil$\crcr}}}

%   v
\def\uvecv{\vbox{\mathsurround=0pt\ialign{##\crcr
     $\scriptscriptstyle\rightharpoonup$\crcr\noalign{\kern1pt\nointerlineskip}
     $\hfil\displaystyle{v}\hfil$\crcr}}}
\def\dvecv{\vbox{\mathsurround=0pt\ialign{##\crcr
     $\scriptscriptstyle\rightharpoondown$\crcr\noalign{\kern-7pt\nointerlineskip}
     $\hfil\displaystyle{v}\hfil$\crcr}}}

\def\astA{{}^\ast A}
\def\circA{{}^\circ A}
\def\astk{{}^\ast k}
\def\circk{{}^\circ k}
\def\astK{{}^\ast K}
\def\circK{{}^\circ K}
\def\astL{{}^\ast L}
\def\circL{{}^\circ L}
\def\astal{{}^\ast \al}
\def\circal{{}^\circ \al}
\def\astsi{{}^\ast \si}
\def\circsi{{}^\circ \si}
\def\aste{{}^\ast e}
\def\circe{{}^\circ e}
\def\astte{{}^\ast \te}
\def\circte{{}^\circ \te}
\def\astGa{{}^\ast \Ga}
\def\circGa{{}^\circ \Ga}
\def\Lie{\pounds}

\long\def\Hide#1{}
\long\def\HideMarked#1{\hfill{$\triangleright$}}
\def\rtau{\tau}


%%%%%%%%%%%%%%%%%%%%%%%%%%%%%%%%%%%%%%%%%%%%%%%%%%%%%%%%%%%%%%%
%\def\NormalStyle{\leftskip=2cm\rightskip=0cm\normalsize\parindent=5pt\parskip=3pt\normalbaselineskip=14pt\baselineskip=\normalbaselineskip}
%\def\AbstractStyle{\leftskip=3cm\rightskip=1cm\scriptsize\parindent=0pt\parskip=0pt\normalbaselineskip=11pt\baselineskip=\normalbaselineskip}
%\def\NoteStyle{\leftskip=3cm\rightskip=1cm\scriptsize\parindent=0pt\parskip=0pt\normalbaselineskip=11pt\baselineskip=\normalbaselineskip}

\def\ShowLabel#1{\ref{#1}}

\def\Bibliography{\begin{thebibliography}{199}\footnotesize\References}
\def\EndBibliography{\end{thebibliography}}
\def\bib#1#2{\bibitem{#1}#2}
\def\Ref#1{\cite{#1}}


\def\NewSection#1{\section{#1}}
\def\NewSubSection#1{\subsection*{#1}}
\def\NewAppendix#1#2{\section*{Appendix #1: #2}}
\def\Acknowledgements{\section*{Acknowledgements}}
\def\bs{\bigskip}
\def\ms{\medskip}
\def\ss{\smallskip}
\def\ni{\noindent}

\long\def\Note#1{\blockquote{\footnotesize {\bf Nota:}~#1\par}}
\long\def\Ex#1{\blockquote{\footnotesize {\bf Esercizio:}~#1\par}}

\def\eq#1{\begin{equation}#1\end{equation}}
\def\eqLabel#1#2{\begin{equation}#1\label{#2}\end{equation}}


\def\Cases#1{\begin{cases}#1\end{cases}}
\def\Matrix#1{\begin{matrix}#1\end{matrix}}
\def\Align#1{\begin{aligned}#1\end{aligned}}

\def\eqs#1{\eq{\Align{#1}}}
\def\eqsLabel#1#2{\eq{\Align{#1}\label{#2}}}




\def\Abstract{\AbstractStyle{\bf Abstract. }}
\def\EndAbstract{\par\NormalStyle}

\def\TitleScript{}
\def\TitleLine#1{\def\TitleScript{#1}}
\def\MoreTitleLine#1{\edef\TitleScript{\TitleScript\\#1}}

\def\AuthorScript{}
\def\AuthorLine#1{\def\AuthorScript{#1}}
\def\MoreAuthorLine#1{\edef\AuthorScript{\AuthorScript\\#1}}

\def\AddressScript{}
\def\AddressLine#1{\def\AddressScript{#1}}
\def\MoreAddressLine#1{\edef\AddressScript{\AddressScript\\#1}}

\date{}

\def\BeginDocument{\begin{document}}

\def\MakeTitle{
\begin{center}
{\Large\bf\sffamily\baselineskip=2pt\TitleScript}\\
\vskip 10pt
{\small by {\it \AuthorScript}}\\
\vskip 10pt
{\small \AddressScript}
\end{center}
\ms
}

\def\EndDocument{\end{document}}

\def\Figure[#1]#2{\begin{figure}[htbp] %  figure placement: here, top, bottom, or page
   \centering
   \includegraphics[#1]{#2} }
\def\Caption#1{\caption{#1}}
\def\EndFigure{\end{figure}}

\def\Diagram#1{\eq{
\begindc{\commdiag}[10]
#1
\enddc%
}%
}

\def\Itemize#1{\begin{itemize}#1\end{itemize}}
\def\Item[#1]{\item[#1]}

\def\AllReferences{}



\def\Shrink{\small}
%%%%%%%%%%%%%%%%%%%%%%%%%%%%%%%%%%%%%%%%%%%%%%%%%
%%%%%%%%%%%%%%%%%%%%%%%%%%%%%%%%%%%%%%%%%%%%%%%%%%%

%\BeginDocument


%%%%%%%%%%%%%%%%%%%%%%%%%%%%%%%%%%%%%%%%%%%%%%%%%%%






















\parindent=5pt
\baselineskip=13pt
\parskip=5pt





\begin{document}

\AllReferences


%%%%%%%%%%%%%%%%%%%%% Publisher's Area please ignore %%%%%%%%%%%%%%%
%
%
%%%%%%%%%%%%%%%%%%%%%%%%%%%%%%%%%%%%%%%%%%%%%%%%%%%%%%%%%%%%%%%%%%%%

\section*{Giorno 9: relazioni d'ordine}

Così come abbiamo visto che le relazioni di equivalenza catturano la nostra intuizione di uguaglianza sotto certi aspetti (2 persone diverse possono essere diverse ma uguali sotto la relazione d'equivalenza "sono alte uguali" oppure "pesano uguale" oppure "hanno lo stesso sesso"), le {\it relazioni d'ordine} catturano la nostra intuizione di ordine in un insieme.
A noi interessano principalmente gli ordini totali e i buoni ordini, ma ci sono diverse gradazioni di grigio e non conviene puntare dritto all'obiettivo, conviene fare lo sforzo di catalogare i diversi tipi di relazione d'ordine.

Definizione: un {\it preordine} su un insieme $A$ è una relazione $\le$ che soddisfa la proprietà  per ogni $x, y, z\in A$
$$
\Align{
&x\le x\cr
&\hbox{se   $x\le y$ e $y\le z$ allora $x\le z$ }
}\qquad
\Align{
\hbox{(riflessiva)}\cr
\hbox{(transitiva)}\cr
}
$$

\Note{le relazioni di equivalenza soddisfano la proprietà riflessiva e la proprietà transitiva, quindi sono preordini.
Non è vero il viceversa: non tutti i preordini sono relazioni di equivalenza perché non è richiesta la proprietà simmetrica.

Quindi la strada giusta dovrebbe essere definire i preordini e poi le relazioni di equivalenza come particolare preordini che hanno anche la proprietà simmetrica.
Siccome dobbiamo tenere a mente un sacco di proprietà e abbiamo un numero finito di neuroni dovremmo abituarci a riorganizzare continuamente la nostra conoscenza in questo modo.
Questa è una cosa che ci insegna la matematica anche se non ci interessa essere matematici: la conoscenza {\it deve} essere continuamente coccolata e riorganizzata mentre cresce, se no ci bastava Wikipedia.

Questa è, secondo me, {\it una} ragione per provare ad insegnare matematica per 13 anni a tutta la popolazione anche se evidentemente con scarsissimi risultati.
Tra parentesi questo secondo me indica che lo scopo della scuola pubblica, quella dell'obbligo, non è insegnare cose ai bambini. 
I bambini sono costretti ad andare a scuola, come i carcerati sono costretti a stare in carcere. Non puoi chiedere a un carcerato di stare 13 anni in carcere e pure imparare a dare il bianco alla cella, al massimo puoi {\it consentirgli} di dare il bianco alla cella. La scuola dell'obbligo ha il compito di esporre tutta la popolazione al maggior numero di cose possibile e {\it consentire} loro di sviluppare i loro interessi in qualcuna di queste materie. Possibilmente di raggiungere un livello civile di navigazione in quelle discipline che non interessano.
}

Definizione: un {\it ordine parziale} è un preordine con la proprietà antisimmetrica
$$
\hbox{se $x\le y$ e $y\le x$ allora $x=y$}
$$

\Ex{Dato un insieme $A$ e l'insieme $P(A)$ delle sue parti, diciamo che un sottoinsieme $S_1\in P(A)$  di $A$ è {\it incluso} in un  sottoinsieme $S_2\in P(A)$, e scriviamo $S_1\subset S_2$
se e solo se $S_1$ è anche un sottoinsieme di $S_2$. 

L'inclusione è un ordine parziale di $P(A)$. 

Ogni sottoinsieme è contenuto in se stesso (riflessiva).
In più se $S_1\subset S_2$ e $S_2\subset S_3$ allora $S_1\subset S_3$ (transitiva, quindi è un preordine).

Infine se $S_1\subset S_2$ significa che tutti gli elementi di $S_1$ sono anche elementi di $S_2$, e anche se $S_2\subset S_1$ significa che tutti gli elementi di $S_2$ sono anche elementi di $S_1$, allora $S_1$ e $S_2$ hanno gli stessi elementi, quindi sono lo stesso sottoinsieme (antisimmetrica).

Quindi l'inclusione è un ordine parziale.
}

Definizione: un {\it ordine totale} è un ordine parziale che in più ha la proprietà di comparazione,
dati $x, y\in A$ si ha che $x\le y$ oppure $y\le x$.

\Note{L'inclusione è una ordine parziale che non è totale (due sottoinsiemi possono essere tali da non essere inclusi né $S_1\subset S_2$ né $S_2\subset S_1$).

I numeri naturali sono ordinati totalmente dalla relazione $n_1\le n_1$ che è un ordine totale in $\N$.

Per la cronaca dobbiamo definire tutto, quindi pure cosa significa $n_1\le n_2$.
Se definiamo i numeri naturali con gli assiomi di Peano, diciamo che $n_1\le n_2$ dicendo che valgono le seguenti proprietà
$$
n\le n
\qquad\qquad
n\le s(n)
$$

Se vogliamo dimostrare che $3\le 5$ non dobbiamo fare altro che notare che $5= s(s(3))$, quindi
$$
3 \le s(3) \le s(s(3)) = 5
$$
quindi $3\le 5$ per la proprietà transitiva.
}

Un {\it buon ordine} di $A$ è un ordine totale tale che ogni sottoinsieme $S$, non-vuoto di $A$ ha un elemento minimo, cioè più piccolo (rispetto all'ordine totale che stiamo considerando su $A$) di tutti gli altri elementi di $S$.
L'insieme dei numeri naturali $\N$ con l'ordine $\le$ è ben ordinato. 
Prendete un qualunque sottoinsieme non vuoto di $\N$ in esso esiste sempre un minimo. 

\Note{Notate che non è vero che esiste sempre un massimo (abbiamo già detto che in $\N$, che è un sottoinsieme non-vuoto di $\N$, non esiste un numero più grande di tutti gli altri).

Se vi gira la testa, non vi preoccupate è come per chi vive al mare andare sulle Ande, è scarsità di ossigeno dopo un paio di giorni (o masticando foglie di coca) passa.
Sì, stiamo dicendo che {\it ogni} sottoinsieme non-vuoto di $\N$ ha un minimo anche se i sottoinsiemi di $\N$ sono in numero infinito, anche se alcuni sottoinsieme di $\N$ sono infiniti.
Ok, dovremmo dimostrarlo ma per quello hanno inventato le dimostrazioni, perché consentono di dimostrare infinite cose (nessun pari maggiore di $3$ è primo, e grazie si divide per 2 oltre che per 1 e per $n$).

Ok non abbiamo ancora detto cosa è un primo, ma lo sapete. Ma questo vi conferma che la paranoia principale del matematici è non fare ragionamenti circolari.
Non assumere cose senza dimostrazione e definizione che alla fine renda sbagliato il ragionamento. Per questo, per fare le cose per bene bisogna andare piano, dare gli assiomi dare le definizioni, dimostrare i teoremi e poi magari dare degli esempi. Non si può fare come stiamo facendo qui, dare gli esempi prima usando cose che non sono state definite. È pericoloso.

Poi quindi capite perché i matematici bestemmiano quando gli si dice {\it eh ma la matematica è astratta, non puoi parlare come mangi e fare un esempio di quello che mi stai dicendo così capisco?} 

La risposta dovrebbe essere: {\it no ora non posso darti degli esempi, prima dimostriamo i teoremi poi tra 3 giorni ti faccio un esempio! Non me ne frega nulla della tua intuizione, se potevi intuire cosa era uno spazio di Hilbert, significa che stai pensando a un esempio con certe proprietà, è pericoloso avere in mente un esempio perché facilita aggiungere proprietà che magari il tuo esempio ha ma non tutti gli spazi di Hilbert e quando dimostri i teoremi è pericoloso avere delle proprietà in mente perché si finisce per sparare stronzate.}

A proposito, sui principia matematica di Russell il teorema $1+1=2$ è marcato col numero 10mila e qualcosa. Diecimila e rotti teoremi prima di sapere che 1+1=2.
La matematica è lenta.
}

\EndDocument
\end


